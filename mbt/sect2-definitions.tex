\section{Network Model and Definitions} \label{sec:def}

The communication network is represented by a connected graph $G=(V,E)$ and a subset $S\subseteq V$ referred to as the set of sources, where $|V|=n$. 
% Let $V\left[F\right]=\{v\in V:\text{ some edge in } F \text{ is incident to } v\}, F\subseteq E$.
% For disjoint node sets $V_1,V_S\subseteq V$, we denote by $G\left[V_1,V_2\right]$ the bipartite graph $(V_1\cup V_2,E')$, where  $E'=\{\{u,v\}\in E: u\in V_1,v\in V_2\}$. 

\begin{definition} \label{def:broadcasttime}
The \emph{broadcast time} $\tau(G,S)$ of a node set $S\subseteq V$ in $G$ is defined as the smallest integer $t\geq 0$ for which there exist
a sequence $V_0\subseteq\dots\subseteq V_t$ of node sets and a function $\pi:V\setminus S\to V$, such that:
\begin{enumerate}
  \item $V_0=S$ and $V_t=V$, \label{def:boundary}
  \item for all $v\in V\setminus S$, $\{v,\pi(v)\}\in E$, \label{def:edge}
  \item for all $i=1,\ldots,t$ and all $v\in V_i$, $\pi(v)\in V_{i-1}$, and \label{def:parent}
  \item for all $u,v\in V_i\setminus V_{i-1}$, $\pi(u)=\pi(v)$ only if $u=v$. \label{def:unique}
\end{enumerate}
% % \begin{equation*}
% \tau(G,S)=
% \begin{cases}
% 	0, S=V,\\
% 	1+\min\{\tau(G,S\cup V\left[M\right]):M \text{ is a matching in } G\left[S, V\setminus S\right]\}, S\neq V.
% \end{cases}
% \label{eq:btime}
% \end{equation*}
\end{definition}
% %Broadcasting is defined as a sequence of sets $(S=V_0\subseteq\dots\subseteq V_k = V)$ where each $V_i$ represents the nodes informed after time step $i$, $0\leq i\leq t$.
% %For each node $v\in V_i\setminus V_{i-1}$, there exists a single node $\pi(v)\in V_{i-1}$ adjacent to $v$, which forwarded the signal to $v$.
% %Also, for every two nodes $u,v\in V_i\setminus V_{i-1}$ we have $\pi(u)=\pi(v)$.
% %The value $k$ is referred to as \emph{delay}.

Referring to Section \ref{intro}, the node set $V_i$ is the set of nodes that are informed in time step $i$.
Initially, only the sources are informed ($V_0=S$), whereas all nodes are informed after $t$ time steps ($V_t=V$),
and the set of informed nodes is monotonously non-decreasing ($V_{i-1}\subseteq V_i$ for $i=1,\ldots,t$).
The parent function $\pi$ maps each node to the node from which it received the signal.
Conditions \ref{def:edge}--\ref{def:parent} of Definition \ref{def:broadcasttime} thus reflect that the sender is a neighbor node in $G$,
and that it is informed at an earlier time step than the recipient node.
Because each node can send to at most one neighbor node in each time step,
condition \ref{def:unique} states that $\pi$ maps the set of nodes becoming informed in step $i$ to distinct parent nodes.

The optimization problem in question is formulated as follows:
\begin{problem}[\textsc{Minimum Broadcast Time}]\label{prob:min}
Given $G=(V,E)$ and $S\subseteq V$, find $\tau(G,S)$.
% a sequence $(S=V_0\subseteq\dots\subseteq V_{\tau(G,S)}=V)$ and a function $\pi:V\setminus S\to V$, such that for each $v\in V\setminus S:\{v,\pi(v)\}\in E$, and for each $u,v\in V_i\setminus V_{i-1}: \pi(u)=\pi(v)\Leftrightarrow u=v$.
\end{problem}

Directed versions of Defintion \ref{def:broadcasttime} and Problem \ref{prob:min} are defined analogously.
For a digraph $D=(V,A)$, condition \ref{def:edge} is modified to state that $(\pi(v),v)\in A$ for all $v\in V\setminus S$.
We denote by $\tau(D,S)$ the broadcast time of $S$ in $D$.

\begin{definition} \label{def:broadcastgraph}
For any $V_0,\ldots,V_t$ and $\pi$ satisfying the conditions of Definition \ref{def:broadcasttime},
the corresponding \emph{communication forest} is
the digraph $D=(V,A)$, where $A=\left\{\left(v,\pi(v)\right): v\in V\right\}$.
Each connected component of $D$ is a \emph{communication tree}.
\end{definition}

\noindent
It is easily verified that the communication trees are indeed arborescences, rooted at distinct sources, with arcs pointing away from the source.
Let $T_s=\left(V_{T_s},A_{V_s}\right)$ denote the communication tree rooted at source $s\in S$, and define
the function $\alpha:V\to S$ such that $v\in V_{T_{\alpha(v)}}$ for all $v\in V$.
That is, $\alpha$ associates a node $v$ with the source at which the communication tree containing $v$ is rooted.
For any communication tree $T_s$, let $T^i_s$ be the subtree of $T_s$ induced by the set of nodes with distance at most $i$ edges from $s$.
Analogously, let $D^i=\left\{T_s^i: s\in S\right\}$.

% Given a feasible sequence of node sets and $\pi$, we derive the following broadcast trees \cite{grigni91}:
% For $s\in S$, a broadcast tree $T_s=(V_{T_s},E_{T_s})$ is a time-labeled directed subgraph of $G$ describing a broadcast originated in $s$ by the rules:
% \begin{enumerate}
% \item $T_s$ is rooted at $s$ with arcs directed towards the leaves.
% \item each node $v\in V_i$ is labeled with an integer $\tau(v)=i$,
% \item for $v\in V\setminus S, v\in V_{T_s}\Leftrightarrow \pi(v)\in V_{T_s}$, 
% \item $E_{T_s}=\{\{u,v\}: u,v\in V_{T_s}, \pi(v)=u\}$, and
% \item for two sources $s,s'\in S, s\neq s'$ we have that $V_{T_s}\cap V_{T_{s'}}=\emptyset$. 
% \end{enumerate}

% Any feasible sequence of node sets and $\pi$ yield a set of trees $T=\{T_s:s\in S\}$ that forms a partition of $G$ into trees referred to as \emph{broadcast forest}.

% Given a broadcast forest $T$, the delay is determined as $\max_{v\in V}\{\tau(v)\}$.
%A set of trees $T=\{T_s:\cup_{s\in S}V_{T_s}=V\}$ forms a partition of $G$ into trees referred to as \emph{broadcast forest}.
%$T$ can be derived from a given sequence of sets of nodes defining broadcasting and the mapping $\pi$.
% For any integer $i$, let $T^i_s$ be the subtree of $T_s$ obtained by pruning all nodes $v\in V_{T_s}$ with $\tau(v)>i$.
% Analogously, we define $T^i=\{T^i_s:s\in S\}$. 


%We also define the set $A=\{(i,j),(j,i):\{i,j\}\in E\}$ that consists of all arcs that can be derived by directing edges in $E$.
The degree of node $v$ in graph $G$ is denoted by $\delta_G(v)$.
For the diameter of $G$ we use the symbol $\Delta_G$.
The set of neighbors of $v\in V$ in $G$ is denoted by $N_G(v)$.
Whenever there is no risk of confusion, the subscript $G$ is omitted.
For a given subset $U\subseteq V$ of nodes, we define $N(U)=\bigcup_{v\in U}N(v)$.

Likewise, $\delta^-_D(v)$ and $\delta^+_D(v)$ denote, respectively, the in-degree and the out-degree of digraph $D$,
and $N_D^+(v)=\left\{u\in V:(v,u)\in A\right\}$ and $N_D^-(v)=\left\{u\in V:(u,v)\in A\right\}$ denote the neighbor sets of node $v$ in $D$.
The node set $L(D)=\left\{v\in V: N_D^+(v)=\emptyset\right\}$ is referred to as the \emph{leaves} in $D$.

%For convenience, we consider the following definition of broadcast trees \cite{grigni91}:
%For $s\in S$, a broadcast tree $T_s$ with node set $V_{T_s}$ is a time-labeled directed subgraph of $G$ describing a broadcast originated by $s$ by the following rules:
%\begin{enumerate}
%\item $T_s$ is rooted at $s$ with arcs directed towards the leaves.
%\item Each node $v$ is labeled with an integer $\tau(v)$, where $\tau(s)=0$.
%\item Whenever $v$ is a parent of $u$ in $T_s$ $\tau(v)<\tau(u)$.
%\item Whenever $v$ and $u$ are siblings in $T_s$, $\tau(v)\neq \tau(u)$. 
%\end{enumerate}
%The number of leaves in $T_s$ is denoted by $L(T_s)$.

