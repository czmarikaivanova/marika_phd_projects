\section{Exact methods} \label{sec:exact}

Problem \ref{prob:min} can be formulated as an integer linear program (ILP). % and solved to optimality. 
In this section, we present an ILP model and possible solving strategies. 

\subsection{Broadcast time model}
The studied model is a straightforward formulation of the problem.
Consider variables 
$$ x_{uv}^k=
\begin{cases} 
1, \text{ if } v\in V_k \text{ and } \pi(v)=u,\\ 
0, \text{ otherwise},
\end{cases}
z_{k}=\begin{cases}
1, \text{ if } k\leq\tau(G,S),\\
0, \text{ otherwise}.
\end{cases}
$$
The worst case scenario is when $G$ is a path $v_1,\dots,v_n$ with $S=\{v_1\}$. 
In such an instance, the necessary number of time steps is $n-1$, which gives a trivial upper bound $\bar{t}=n-s$ on the broadcast time.
Problem \ref{prob:min} is then formulated as follows: 
\begin{subequations}\label{mod:basic}
\begin{align}
\label{mod:basic:obj} \min \sum\limits_{k=1}^{\bar{t}}z_k \\ 
\text{s. t.~~~} \label{mod:basic:singlein} \sum\limits_{k=1}^{\bar{t}}\sum\limits_{v\in N(u)}x_{vu}^k & = 1 & u\in V \setminus S,\\
%\label{mod:basic:uniqueTout} \sum\limits_{v\in N(u)}x_{uv}^k & \leq 1  & u\in V,k=1,\dots,\bar{t},\\
\label{mod:basic:tIncreases} \sum\limits_{v\in N(u)}x_{uv}^k &\leq\sum\limits_{\ell=1}^{k-1}\sum\limits_{w\in N(u)} x_{wu}^{\ell}  & u\in V\setminus S, k=2,\dots,\bar{t},\\
%\label{mod:basic:tIncreases} x_{uv}^k &\leq\sum\limits_{\ell=1}^{k-1}\sum\limits_{w\in N(u)\setminus\{v\}} x_{wu}^{\ell}  & \{u,v\}\in E, u\not\in S, k=2,\dots,\bar{t},\\
%\label{mod:basic:tcrel} \sum\limits_{k=1}^{\bar{t}}k\cdot x_{uv}^k & \leq t^* &  (u,v)\in A,\\
\label{mod:basic:tcrel} \sum\limits_{v\in N(u)}x_{uv}^k & \leq z_k &  u\in V,k=1,\dots,\bar{t},\\
\label{mod:basic:timing} z_k & \leq z_{k-1} &  k=2,\dots,\bar{t},\\
%\label{mod:basic:tcrel} \sum\limits_{t=1}^{n-1}t\sum\limits_{j\in N(i)}x_{ij}^k & \leq c &  i\in V,\\
\label{mod:basic:positiveCost}x_{uv}^1 & = 0 & (u,v)\in A, u \in V\setminus S,\\
\label{mod:basic:dim}x \in \{0,1\}^{A\times \{1,\dots,\bar{t}\}},z&\in\{0,1\}^{\bar{t}}.&
\end{align}~
\end{subequations}
By \eqref{mod:basic:singlein}, every non-source node $u$ receives the signal from exactly one adjacent node $v$ in some time step $k$.
%The requirement that a non-source node has a neighbor $v\in V_k$ such that $\pi(v)=u$ only if there exists a node $w\in V_{k-1}$ such that $\pi(u)=w$ is modeled by \eqref{mod:basic:tIncreases}. 
The requirement that a non-source node $u$ informs a neighbor $v$ in the $k$-th time step only if $u$ is informed by some adjacent node $w$ in an earlier time step is modeled by \eqref{mod:basic:tIncreases}. 
%Constraints \eqref{mod:basic:uniqueTout} enforce that for each node $u\in V$ and each subset $V_k$, there is at most one adjacent node $v\in V_k$ with $\pi(v)=u$.
Constraints \eqref{mod:basic:tcrel} enforce that each node $u\in V$ forwards the signal to at most one adjacent node $v$ in each time step.
It also sets correct values to the $z$-variable that appear in the objective function.
%The requirement that only informed nodes can relay a signal is modeled by \eqref{mod:basic:tIncreases}. 
%The maximum time step at which any transmission takes place is captured by \eqref{mod:basic:tcrel}, and finally, \eqref{mod:basic:positiveCost} states that a node that is not a source never transmits in the first time step.
%The length of the sequence of subsets is captured by \eqref{mod:basic:tcrel}, and finally, \eqref{mod:basic:positiveCost} state that if $\pi(v)\not\in S$ for some $j\in V$, then $v\not\in V_1$.
The valid inequality \ref{mod:basic:timing} states that $k$ is no greater than broadcast time of the given node set only if the same holds for $k-1$.
Lastly, \eqref{mod:basic:positiveCost} state that non-source nodes do not transmit in the first time step.

\subsubsection{Decision version}
\label{sec:decbasic}
The nature of MBT suggests another modelling approach derived from Model \ref{mod:basic}. 
For a given maximum time period $t$, we maximize the number of nodes $v$ that receive a signal from some neighbor $u$ within $t$ time steps.
If the optimal value is $n-s$, it is clear that we found a spanning broadcast tree with broadcast time $t$.
In case the optimal value is less than $n-s$, the given $t$ is insufficient, and we have to try solving the problem with an increased time limit.
Clearly, an additional knowledge of upper and lower bound spares some computations. 
A lower bound is the initial value of $t$. 
Similarly, if an upper bound $\bar{t}$ is known and $t=\bar{t}-1$, the process can be terminated even when the objective function is less then $n-s$, because $\bar{t}$ is than the optimal value.

The decision model \ref{mod:basic:dec}  has the form
\begin{subequations}\label{mod:basic:dec}
\begin{align}
\label{mod:basic:dec:obj} \max \sum\limits_{v \in V\setminus S}\sum\limits_{u \in N(v)} \sum\limits_{k=1}^{t}x_{uv}^k \\ 
\text{s. t.~~~} \label{mod:basic:dec:atMost1in} \sum\limits_{k=1}^{t}\sum\limits_{v\in N(u)}x_{vu}^k & \leq 1 & u\in V \setminus S,\\
\label{mod:basic:dec:0toSource} \sum\limits_{k=1}^{t}\sum\limits_{v\in N(u)}x_{vu}^k & = 0  & u\in S,\\
\eqref{mod:basic:tIncreases},\notag \eqref{mod:basic:timing},  \eqref{mod:basic:positiveCost},\\
\label{mod:basic:dec:dim}x &\in \{0,1\}^{A\times \{1,\dots,t\}}.&
\end{align}~
\end{subequations}

Constraint \eqref{mod:basic:singlein} is replaced by \eqref{mod:basic:dec:atMost1in} and \eqref{mod:basic:dec:0toSource}. 
The former is an inequality, because not all nodes are necessarily reached within the given time limit.
The latter makes sure that sources are not reached by any signal. 
This was filtered out by optimality in \eqref{mod:basic}, but it must be forbidden explicitly in \eqref{mod:basic:dec} due to the changed objective function.
The actual target function is the minimum time required for spreading the signal to all nodes, and so we are looking for the smallest value of $t$.

