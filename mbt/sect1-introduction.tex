\section{Introduction}
\label{intro}
The minimum broadcast time (MBT) problem is identified by a graph and a subset of its nodes,
referred to as source nodes.
Each node in the graph corresponds to a communication unit.
The task is to disseminate a signal from the source nodes to all other nodes in a shortest possible time (broadcast time), while abiding by communication rules.
A node is said to be \emph{informed} at a given time if it is a source, or it already has received the signal from some other node.
Otherwise, the node is said to be \emph{uninformed}.
Consequently, the set of informed nodes is initially exactly the set of sources.
Reflecting the fact that communication can be established only between pairs of nodes that are located within a sufficiently close vicinity of each other,
the edge set of the graph consists of potential communication links along which the signal can be transmitted.

The time is divided into a finite number of steps.
Every informed node can, in each time step, forward the signal to at most one uninformed neighbor node.
Therefore, the number of informed nodes can at most be doubled from one step to the next.
This communication protocol appears in various practical application such as communication among computer processors or telephone networks.
Situations where the signals have to cover large distances typically assume sending the signal to one neighbor at the time.
Inter-satellite communication networks thus constitute a prominent application area \cite{chu17}.
In particular, the MBT problem arises when one or a few satellites need to broadcast data quickly by means of time-division multiplexing.

The current literature on MBT offers some theoretical results, including complexity and approximability theorems.
Although inexact solution methods also have been proposed, few attempts seem to be made on order to compute the exact optimum,
or to find lower bounds on the minimum broadcast time.
The goal of the current text is to fill this gap, and we make the following contributions in that direction:

First, a compact integer programming model is developed.
While the model targets the exact minimum in instances of moderate size, its continuous relaxation is suitable for computation of lower bounds in larger instances.
Second, we derive lower bounding techniques, both of an analytical nature and in terms of a combinatorial relaxation of MBT,
that do not rely on linear programming.
Third, we devise an upper bounding algorithm, which in combination with the lower bounds is able to close the optimality gap in a wide range of instances.

The remainder of the paper is organized as follows:
Next, we review the current scientific literature on MBT and related problems,
and in Section \ref{sec:def}, a concise problem definition is provided.
The integer program is formulated and discussed in Section \ref{sec:exact}.
Lower and upper bounding methods are derived in Section \ref{sec:lb} and \ref{sec:ub}, respectively.
Computational experiments are reported in Section \ref{sec:exp}, before the work is concluded by Section \ref{sec:conc}.

\subsection{Literature overview} \label{sec:litrev}

Deciding whether an instance of MBT has a solution with broadcast time at most $t$ has been shown to be NP-complete \cite{slater81}. 
For bipartite planar graphs with maximum degree 3, NP-completeness persists even if $t=2$ or if there is only one source \cite{jansen95}.
When $t=2$, the problem also remains NP-complete for cubic planar graphs \cite{middendorf93}, grid graphs with maximum degree 3,
complete grid graphs, chordal graphs, and for split graphs \cite{jansen95}. 
The single-source variant of the decision version of MBT is NP-complete for grid graphs with maximum degree 4, and for chordal graphs \cite{jansen95}.
The problem is known to be polynomial in trees \cite{slater81}.
Whether the problem is NP-complete for split graphs with a single source was stated as an open questions in \cite{jansen95}, and has to the best of our knowledge not been answered yet.

A number of inexact methods, for both general and special graph classes, have been proposed in the literature during the last three decades.
One of the first works of this category \cite{scheuermann84} 
introduces a dynamic programming algorithm based on generating all maximum matchings in an induced bipartite graph.
Additional contributions of \cite{scheuermann84} are heuristic approaches for near optimal broadcasting.
From more recent works we mention \cite{hasson04}, which describes a meta heuristic algorithm for MBT, and provides a comparison with other existing methods.
The communication model is considered in an existing satellite navigation system in \cite{chu17}, where a greedy inexact method is proposed together with a mathematical programming model.
Examples of additional efficient heuristics can be found e.g. in \cite{harutyunyan06,harutyunyan14,wang10,jimborean13}.

Approximation algorithms for MBT are studied in \cite{kortsarz95}. 
The authors argue that methods presented in \cite{scheuermann84} provide no guarantee on the performance, and show that a wheel-graph is an example of an unfavourable instance.
They introduce an $\mathcal{O}(\sqrt{n})$-additive approximation algorithm for broadcasting in general graphs with $n$ nodes.
They further provide approximation algorithms for several graph classes with small separators with approximation ratio proportional to the separator size times $\log n$.
Throughout the text, the symbol $\log$ refers to the logarithm of base 2.
Another algorithm with $\mathcal{O}\left(\frac{\log n}{\log \log n}\right)$-approximation ratio is given in~\cite{elkin03}.
Most of the works cited above consider a single source.

A related problem extensively studied in the literature is the minimum broadcast graph problem \cite{grigni91,mcgarvey16}. 
A broadcast graph supports a broadcast from any node to all other nodes in optimal time $\lceil\log n\rceil$.
For a given integer $n$, a variant of the problem is to find a broadcast graph of $n$ nodes such that the number of edges in the graph is minimized.
In another variant, the maximum node degree rather than the edge cardinality is subject to minimization.
MgGarvey et al.\ \cite{mcgarvey16} study integer linear programming (ILP) models for $c$-broadcast graphs, which is a generalization where signal transmission to at most $c$ neighbours is allowed in a single time step.

Despite a certain resemblance with MBT, the minimum broadcast graph problem is clearly distinguished from our problem,
and will consequently not be considered further in the current work.
