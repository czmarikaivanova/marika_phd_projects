\section{Lower bounds} \label{sec:lb}
Strong lower bounds on the minimum objective function value are of vital importance to combinatorial optimization algorithms.
In this section, we study three types of lower bounds on the broadcast time $\tau(G,S)$.

\subsection{Analytical lower bounds} \label{sec:lbanalyt}
Any solution $\left(V_0,\ldots,V_t,\pi\right)$ satisfying conditions \ref{def:boundary}--\ref{def:unique} of Definition \ref{def:broadcasttime},
also satisfies $\left|V_{i+1}\right|\leq 2\left|V_i\right|$ for all $i\geq 1$.
Because $G$ is connected, $N_G\left(V_i\right)$ intersects $V\setminus V_i$ as long as $V_i\neq V$.
This yields the following bounds:
\begin{observation}
For all instances $(G,S)$ of Problem \ref{prob:min},
\begin{equation}
\left\lceil\log\frac{n}{\sigma}\right\rceil\leq \tau(G,S) \leq n-\sigma.
\label{eq:loglb}
\end{equation}
\label{obs:loglb}
\end{observation}

Consider the $m$-step Fibonacci numbers $\left\{f^{m}_i\right\}_{i=1,2,\ldots}$ \cite{noe05}, a generalization of the well-known (2-step) Fibonacci numbers, defined by
$f^{m}_i=0$ for $i\leq 0$, $f^{m}_1=1$, and 
other terms according to the linear recurrence relation 
\begin{align*}
f^{m}_i &=\sum\limits_{j=1}^m f^{m}_{i-j}, &\text{ for } i\geq 2.
\end{align*}

The generalized Fibonacci numbers are instrumental in the derivation of a lower bound on $\tau(G,S)$,
depending on the maximum node degree $d=\max\left\{\delta_G(v): v\in V\right\}$ in $G$.
The idea behind the bound is that the broadcast time can be no shorter than what is achieved if
the following ideal, but not necessarily feasible, criteria are met:
Every source transmits the signal to a neighbor node in each of the periods $1,\ldots,d$,
and every node $u\in V\setminus S$
transmits the signal to a neighbor node in each of the first $d-1$ periods following the period when $u$ gets informed.
An exception possibly occurs in the last period, as there may be fewer nodes left to be informed than there are nodes available to inform them.

\begin{proposition}
\begin{equation*}
\label{lem:lbreg1}
	\tau(G,S)\geq\min\left\{t:2\sigma\sum\limits_{j=1}^tf^{d-1}_j\geq n\right\}.
\end{equation*}
\label{prop:lbfib}
\end{proposition}
\begin{proof}

Consider a solution $\left(V_0,\ldots,V_t,\pi\right)$ with associated broadcast graph $D$, such that $V_{t-1}\neq V_t$, 
\begin{itemize}
  \item conditions \ref{def:boundary} and \ref{def:parent}--\ref{def:unique} of Definition \ref{def:broadcasttime} are satisfied,
  \item for each source $u\in S$ and each $j=1,\ldots,\min\{d,t-1\}$, there exists a node $v_j\in V_j\setminus V_{j-1}$ such that $\pi\left(v_j\right)=u$, and
  \item for each $i\in\{1,\ldots,t-2\}$, each node $u\in V_i\setminus V_{i-1}$, and each $j=i+1,\ldots,\min\{i+d-1,t-1\}$,
        there exists a node $v_j\in V_j\setminus V_{j-1}$ such that $\pi\left(v_j\right)=u$.
\end{itemize}
\noindent
Clearly, such a solution exists, and $\left(V_0,\ldots,V_t,\pi\right)$ is optimal if $\pi$ also satisfies condition \ref{def:edge}.
We thus have $t\leq\tau(G,S)$.
It remains to prove that $2\sigma\sum_{i=1}^{t-1}f_i^{d-1}<n\leq 2\sigma\sum_{i=1}^tf_i^{d-1}$.

For $i=1,\ldots,t$, let $L_i=\left\{v\in V_i:\delta_{D_i}(v)=1\right\}$ denote the set of nodes with exactly one out- or in-neighbor in $D_i$,
and let $L_i=\emptyset$ for $i\leq 0$.
That is, for $i>1$, $L_i$ is the set of nodes that receive the signal in period $i$, whereas $L_1$ consists of all nodes informed in period 1, including the sources $S$.
Hence, $L_1,\ldots,L_{t-1}$ are disjoint sets (but $V_t$ may intersect $V_{t-1}$), and $V_i=L_1\cup\cdots\cup L_i$ for all $i=1,\ldots,t$.

Consider a period $i\in\{2,\ldots,t-1\}$.
The assumptions on $\left(V_0,\ldots,V_t,\pi\right)$ imply that $\pi$ is a bijection from $L_i$ to $L_{i-1}\cup\cdots\cup L_{i-d+1}$.
Thus, $\left|L_i\right|=\sum_{j=1}^{d-1}\left|L_{i-j}\right|$.
Since also $\left|L_1\right|=2\sigma=2\sigma f_1^{d-1}$ and $\left|L_j\right|=f_j^{d-1}=0$ for $j\leq 0$,
we get $\left|L_i\right|=2\sigma f_i^{d-1}$.
Further, $\left|L_t\right|\leq\sum_{j=1}^{d-1}\left|L_{t-j}\right|=2\sigma f_t^{d-1}$.
It follows that $2\sigma\sum_{i=1}^{t-1}f_i^{d-1}=\sum_{i=1}^{t-1}\left|L_i\right|=\left|V_{t-1}\right|<n=\left|V_t\right|\leq\sum_{i=1}^t\left|L_i\right|\leq 2\sigma\sum_{i=1}^tf_i^{d-1}$,
which completes the proof.
\end{proof}

\subsection{Continuous relaxations of integer programming models} \label{sec:lblprel}

For $t\in\mathbb{Z}_+$, define $\Omega(t)\subseteq[0,1]^{A\times\{1,\ldots,t\}}$ as the set of feasible solutions to the continuous relaxation of
\eqref{mod:basic:dec:obj}--\eqref{mod:basic:dec:dim},
and let
\[
 \Omega^=(t) = \left\{x\in\Omega(t): \sum\limits_{u \in N(v)} \sum\limits_{k=1}^tx_{uv}^k = 1 ~~(v\in V\setminus S)\right\}.
\]
Let $t^{\ast}=\min\left\{t\in\mathbb{Z}_+: \Omega^=\neq\emptyset\right\}$ be the smallest value of $t$ for which the optimal objective function value in
the relaxation equals $n-\sigma$.
Existence of $t^{\ast}$ follows directly from $\Omega^=\left(\tau(G,S)\right)\neq\emptyset$.

% Consider a modification of model \eqref{mod:basic:dec:obj}--\eqref{mod:basic:dec:dim} where the coverage constraint
% $\sum\limits_{u \in N(v)} \sum\limits_{k=1}^tx_{uv}^k = 1$ is imposed on all non-source nodes $v\in V\setminus S$, and the objective is to minimize the communication time.
The continuous relaxation of \eqref{mod:basic:obj}--\eqref{mod:basic:dim} is feasible for sufficiently large $t$.
We denote its optimal objective function value by $\zeta(t)$.
% $\zeta(t)=\min\left\{\sum\limits_{k=1}^tz_k: x\in\Omega^=(t), z_k\geq\sum\limits_{u \in N(v)}x_{uv}^k ~~(k=1,\ldots,t, v\in V\setminus S)\right\}$.

\begin{proposition} \label{prop:lpweak}
For all $t\in\mathbb{Z}_+$ such that $\Omega^=(t)\neq\emptyset$, $\zeta(t)\leq t^{\ast}\leq\tau(G,S)$.
\end{proposition}
\begin{proof}
We first prove that $\zeta(t)$ is non-increasing with increasing $t$:
Let $\Gamma(t)$ denote the set of feasible solutions to the continuous relaxation of \eqref{mod:basic:obj}--\eqref{mod:basic:dim}, and assume $(x,z)\in\Gamma(t)$.
Define $\hat{x}\in[0,1]^{A\times\{1,\ldots,t+1\}}$ such that
for all $(u,v)\in A$, $\hat{x}_{uv}^k=x_{uv}^k$ ($k\leq t$) and $\hat{x}_{uv}^{t+1}=0$.
An analogous extension of $z$ to $\hat{z}\in[0,1]^{\{1,\ldots,t+1\}}$ yields $(\hat{x},\hat{z})\in\Gamma(t+1)$,
and $\sum_{i=1}^{t+1}\hat{z}_i=\sum_{i=1}^tz_i$ proves that $\zeta(t+1)\leq\zeta(t)$.

For $t\in\mathbb{Z}_+$ such that $\Omega^=(t)\neq\emptyset$, $t\geq t^{\ast}$ thus implies $\zeta(t)\leq\zeta(t^{\ast})$.
Since the only lower bounds on $z_k$ in \eqref{mod:basic:obj}--\eqref{mod:basic:dim} are $\max_{v\in V\setminus S}\sum\limits_{u \in N(v)}x_{uv}^k\leq 1$ and $z_{k-1}\leq 1$,
we get $\zeta(t)\leq\zeta(t^{\ast})\leq t^{\ast}$.
The proof is complete by observing that $t^{\ast}\leq\tau(G,S)$ follows from $\Omega^=\left(\tau(G,S)\right)\neq\emptyset$.
\end{proof}

\begin{remark} \label{rem:lpweak}
Proposition \ref{prop:lpweak} suggests to solve a sequence of instances of the continuous relaxation of problem \eqref{mod:basic:dec:obj}--\eqref{mod:basic:dec:dim},
and stop by the first value of $t$ for which the optimal objective function value is $n-\sigma$.
Such an approach yields a lower bound ($t^{\ast}$) on $\tau(G,S)$,
which is no weaker than the bound achieved by solving the continuous relaxation of \eqref{mod:basic:obj}--\eqref{mod:basic:dim}.
\end{remark}

\begin{remark} \label{rem:otheropt}
Remark \ref{rem:lpweak} applies to a reformulation of \eqref{mod:basic:obj}--\eqref{mod:basic:dim}, where a unique integer variable $y$ replaces $z_1,\ldots,z_t$,
and the objective is to minimize $y$ subject to the constraints
$y\geq\sum\limits_{k=1}^tk\sum\limits_{u \in N(v)}x_{uv}^k ~~(v\in V\setminus S)$, \eqref{mod:basic:singlein}--\eqref{mod:basic:tIncreases},
and \eqref{mod:basic:positiveCost}--\eqref{mod:basic:dim}.
\end{remark}

\subsection{Combinatorial relaxations} \label{sec:lbcombrel}

Lower bounds on the broadcast time $\tau(G,S)$ are obtained by omitting one or more of the conditions imposed in Definition \ref{def:broadcasttime}.
For the purpose of strongest possible bounds, the relaxations thus constructed can be supplied with conditions that are redundant in the problem definition.

Recall from Section \ref{sec:def} that $D=(V,A)$ denotes the communication forest corresponding to $\left(V_0,\ldots,V_t,\pi\right)$.
Conditions \ref{def:boundary}--\ref{def:unique} of Definition \ref{def:broadcasttime} imply that
\begin{enumerate}
\setcounter{enumi}{4}
  \item for all $v\in V$, $\delta_D^+(v)+\delta_D^-(v)\leq\delta_G(v)$. \label{def:degree}
\end{enumerate}

\noindent
A lower bound on $\tau(G,S)$ is then given by the solution to:
\begin{problem}[\textsc{Node Degree Relaxation}]\label{prob:degree}
Find the smallest integer $t\geq 0$ for which there exist
a sequence $V_0\subseteq\dots\subseteq V_t$ of node sets and a function $\pi:V\setminus S\to V$,
satisfying conditions \ref{def:boundary} and \ref{def:parent}--\ref{def:degree}.
\end{problem}

Observe that the bound given in Proposition \ref{prop:lbfib} is obtained by exploiting the lower-bounding capabilities of the \textsc{Node Degree Relaxation}.
By considering the degree of all nodes $v\in V$, rather than just the maximum degree, stronger bounds may be achieved in instances where $G$ is not regular
($\min_{v\in V}\delta_G(v)<\max_{v\in V}\delta_G(v)$).

% In order to develop an exact algorithm for the \textsc{Node Degree Relaxation}, we derive some optimality conditions.
Denote the source nodes $S=\left\{v_1,\dots,v_{\sigma}\right\}$ and the non-source nodes $V\setminus S=\left\{v_{\sigma+1},\ldots,v_n\right\}$, where $\delta_G(v_{\sigma+1})\geq\delta_G(v_{\sigma+2})\geq\dots\geq\delta_G(v_n)$,
and let $d_i=\delta_G(v_i)$ ($i=1,\ldots,n$).
Thus, $\left\{d_1,\ldots,d_n\right\}$ resembles the conventional definition of a non-increasing degree sequence of $G$,
with the difference that only the final $n-\sigma$ elements are required to be non-increasing.

Given a graph with a degree sequence 
$Q=(d_1,\dots,d_n)$, we determine a lower bound by finding a smallest broadcast time among all possible graphs with degree sequence $Q$.
A modified verision of Problem \ref{prob:degree} takes a fixed $t$ and asks for a sequence $V_0,\dots,V_t$ minimizing $|V_t|$ while satisfying \ref{def:boundary} and \ref{def:parent}--\ref{def:degree}.
As the requirement \ref{def:edge} is relaxed, Problem~\ref{prob:degree} can be solved by iterative solution of its modified version for an increasing $t$.
As soon as $|V_t|=n$, $t$ is the optimal solution to Problem~\ref{prob:degree}.
Lemma \ref{lemma:degorder} states a sufficient optimality condition condition for the modified Problem~\ref{prob:degree}, and thereby also for Problem~\ref{prob:degree} itself.

For a given $t\in\mathbb{Z}_+$, consider the problem of finding $\left(V_0,\ldots,V_t,\pi\right)$ such that $V_0=S$,
conditions \ref{def:parent}--\ref{def:degree} are satisfied, and $\left|V_t\right|$ is maximized.
The smallest value of $t$ for which the maximum equals $n$ is obviously the solution to Problem~\ref{prob:degree}.

The algorithm for Problem~\ref{prob:degree}, to follow later in the section, utilizes that the maximum value of $\left|V_t\right|$
is achieved by transmitting the signal to nodes in non-increasing order of their degrees.
Observe that, contrary to the case of Problem~\ref{prob:min}, transmissions to non-neighbors are allowed in the relaxed problem.
Any instance of Problem~\ref{prob:degree} thus has an optimal solution where, for $i=1,\ldots,t-1$,
$u\in V_i\setminus V_{i-1}$ and $v\in V_{i+1}\setminus V_i$ implies $\delta_G(u)\geq\delta_G(v)$.

More rigorously, this is stated as follows:
\begin{lemma}
\label{lemma:degorder}
The maximum value of $\left|V_t\right|$ over all $\left(V_0,\ldots,V_t,\pi\right)$ satisfying $V_0=S$ and
conditions \ref{def:parent}--\ref{def:degree}, is attained by some
$\left(V_0,\ldots,V_t,\pi\right)$ where \\ $\max_{v\in V_{i-1}}\delta_G(v)\leq\min_{v\in V_i}\delta_G(v)$ ($i=2,\ldots,t$).
\end{lemma}
\begin{proof}

A node with degree $d_i$ informed in the $j$-th time step can inform at most $\min\{d_i-1,t-j\}$ nodes.
Consider integers  $k$ and $l$ such that $\sigma< k < \ell\leq n$ and nodes $v_k$, $v_\ell$ with degrees $d_k$, $d_\ell$, respectively.
Assume without loss of generality, that $v_k$ and $v_l$ are informed in the $k'$-th and $\ell'$-th time step, respectively, and that $k'\leq\ell'$.
Then, $d_k-1\geq d_\ell-1$ and $t-k' \geq t-\ell'$.
Let $K_1$ and $L_1$ be the number of nodes that can be informed by $v_k$ and $v_l$, respectively:
$$
K_1=\min\{d_k-1,t-k'\}, ~~~ L_1=\min\{d_\ell-1,t-\ell'\}.
$$
If the order in which $v_k$ and $v_l$ are informed is switched, i.e. when $v_k$ and $v_\ell$ is informed in the $\ell'$-th and $k'$-th time step, respectively, 
we use $K_2$ and $L_2$ to denote the maximum number of nodes informed by $v_k$ and $v_\ell$:
$$
K_2=\min\{d_k-1,t-\ell'\}, ~~~ L_2=\min\{d_\ell-1,t-k'\}.
$$
We now investigate what values the difference 
\begin{equation}
\label{eq:degbounds}
(K_1+L_1)-(K_2+L_2)
\end{equation}
can attain.
Observe that $K_1\geq L_2$ and $K_2\geq L_1$. 
If $K_2>L_1$, there are two distinct cases:
Either $K_2=d_k-1=K_1$, and then also $L_2=d_l-1=L_1$.
	Or $K_2=t-\ell'\leq K_1$ implying $L_1 =d_\ell-1$, and since $d_\ell-1\leq t-\ell'\leq t-k'$, we have $L_2=d_\ell-1$ as well.
We conclude that \eqref{eq:degbounds} always takes a non-negative value.
%Therefore, if the nodes are informed in the order of their decreasing degrees,
%the number of informed nodes within the give time step is no worse than when the nodes are informed in any other order.
Assume that the nodes are informed in an arbitrary order.
As long as it is possible to swap the order in which two nodes are informed so that a node with a higher degree becomes informed before a node with a lower degree,
thus whenever such $k$ and $l$ as above exist, swapping the order can not decrease the number of informed nodes within the given time limit.
The desired order of nodes can always be achieved by a sequence of swaps.
\end{proof}
For the relaxed problem, informing nodes in the desired order is clearly possible.
If $\mathcal{Q}$ is the set of graphs with degree sequence $Q$, a straightforward algorithm determining
$\min\{\tau(G,S): G\in \mathcal{Q}\}$,
and thus an optimal solution to Problem~\ref{prob:degree} can be developed.

Alg. \ref{alg:dreg} operates with the set $I(t)$ that consists of indices of nodes informed within the first $t$ time steps.
The function $a_i(t)$ determines the number of nodes informed by node $v_i$ within $t$ time steps, and finally,
the set $F(t)$ consists of nodes that in period $t$ can inform other nodes.
That is, $v\in F(t)$ if $v$ informed less than $\delta_G(v)$ other nodes, and $v$ itself is also informed.

\begin{algorithm}
\KwData{$I(0)=\{1,\dots,\sigma\}, a_1(0)=\dots = a_{n}(0)=0.$}
\For{$t=1,2,\dots$} {
	$F(t)=\{i\in I(t-1):a_i(t-1)<d_i\},$\\
	$I(t)=\{1,\dots,|I(t-1)|+|F(t)|\},$\\
	$a_i(t)=a_i(t-1)+
	\begin{cases}
		1, i\in F(t)\cup \left(I(t)\setminus I(t-1)\right) \\
		0, \text{ otherwise. }\\
	\end{cases}$\\
	\If{$|I(t)|=n$} {\Return $t$}
		
}
\caption{Lower bound exploiting distribution of degrees}
\label{alg:dreg}
\end{algorithm}


\begin{proposition}
Alg.~\ref{alg:dreg} returns a lower bound on $\tau(G,S)$.
\label{cor:deg}
\end{proposition}
\begin{proof}
		Follows from Lemma \ref{lemma:degorder} and the subsequent discussion. 
\end{proof}

