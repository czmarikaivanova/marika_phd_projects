\section{Lower bounds} \label{sec:lb}
Strong lower bounds on the minimum objective function value are of vital importance to combinatorial optimization algorithms.
In this section, we study three types of lower bounds on the broadcast time $\tau(G,S)$.

\subsection{Analytical lower bounds} \label{sec:lbanalyt}
Because $\left|V_{i+1}\right|\leq 2\left|V_i\right|$ for all $i\geq 1$, and $N_G\left(V_i\right)\neq\emptyset$, the following bounds are obvious:
\begin{observation}
For an instance $(G,S)$ of Problem \ref{prob:min},
\begin{equation}
\left\lceil\log\frac{n}{\sigma}\right\rceil\leq \tau(G,S) \leq n-\sigma.
\label{eq:loglb}
\end{equation}
\label{obs:loglb}
\end{observation}

Consider the $m$-step Fibonacci numbers $\left\{f^{(m)}_k\right\}_{k=1,2,\ldots}$ \cite{noe05}, a generalization of the well-known (2-step) Fibonacci numbers, defined by
$f^{(m)}_k=0$ for $k\leq 0$, $f^{(m)}_1=1$, and 
other terms according to the linear recurrence relation 
\begin{align*}
f^{(m)}_k &=\sum\limits_{i=1}^m f^{(m)}_{k-i}, &\text{ for } k\geq 2.
\end{align*}

Assume that $G$ is a $d$-regular graph with a unique source $s$.
Any broadcast forest consists of a single tree $T_s$ rooted at $s$.
We investigate the number of leaves in $T_s$, and exploit this number to derive a lower bound on $\tau\{G,\{s\}\}$.

If the orientation of arcs in $T_s$ is disregarded, $|L(T^1_s)|=|L(T^2_s)|=2$.
For $i\geq 3$, $|L(T^i_s)|$ is no more than the number of nodes with degree below $d$ in $T^{i-1}_s$, 
because in a $d$-regular graph, only nodes with degree lower than $d$ can inform new uninformed nodes.
It can also be interpreted as the sum of the number of leaves in $T^{i-d+1}_s,\dots,T^{i-1}_s$, which leads to %the following formula
\begin{equation*}
\label{eq:leafrec}
|L(T^i_s)|=\sum\limits_{j=i-d+1}^{i-1} |L(T^j_s)|.
\end{equation*}  
This formula equals to the recursive definition of Fibonacci sequence of order $d-1$.
As each of the two base cases, $|L(T^1_s)|$ and $|L(T^2_s)|$, equals the double of the base cases of the Fibonacci sequence, the number of leaves in time step $i$ is calculated as
\begin{equation*}
\label{eq:fibleaf}
|L(T^i_s)|=2 F^{(d-1)}_i.
\end{equation*}  
The number of nodes in $T^i_s$ can be expressed as the sum of nodes newly informed in time steps $1,\dots,i$, that is, the sum of leaves in $T^1_s,\dots,T^i_s$. Thus,
\begin{equation}
\label{eq:fibcnt}
|V_{T^i_s}|=|V_i|=2\sum\limits_{j=1}^i F^{(d-1)}_j.
\end{equation}

\begin{proposition}
	If $\delta_G(v)=d$ for all $v\in V$, then 
\begin{equation*}
\label{lem:lbreg1}
	\tau(G,S)\geq\min\left\{i:2\sigma\sum\limits_{j=1}^kF^{(d-1)}_j\geq n\right\}.
\end{equation*}
\label{prop:lbfib}
\end{proposition}
\begin{proof}
In order to inform as many nodes as possible within a given time step $t$,
we assume that every informed node $v$ in each time step $i<t$ with $\delta_{T_i}(v)<d$ informs a new, not yet informed node.
Maximum number of nodes informed within $i$ time steps in one tree in the broadcast forest is given by \eqref{eq:fibcnt}.
For the entire broadcast forest, the maximum number of informed nodes becomes $2\sigma\sum\limits_{j=1}^i F^{(d-1)}_j$, because the trees can be completely independent.
We are therefore looking for the smallest $i$, for which the summation exceeds $n$.
%In order to inform $n$ nodes in the best possible scenario, the signal has to be relayed sufficiently many time steps so that the maximum possible number of informed nodes becomes $n$.
%The maximum number of nodes informed within $i$ time steps is given by \eqref{eq:fibcnt}.
%We therefore need to set the upper limit of the summation in \eqref{eq:fibcnt} so that the right-hand side exceeds $n$.
%The reason why the result is divided by $|S|$ is that the best case scenario with several source nodes assumes that the signals initiated in individual sources are spread evenly.
\end{proof}

\subsection{Continuous relaxations of integer programming models} \label{sec:lblprel}

For $t\in\mathbb{Z}_+$, define $\Omega(t)\subseteq[0,1]^{A\times\{1,\ldots,t\}}$ as the set of solutions satisfying
\eqref{mod:basic:tIncreases}, \eqref{mod:basic:positiveCost}, and \eqref{mod:basic:dec:atMost1in},
and let $\Omega^=(t) = \left\{x\in\Omega(t): \sum\limits_{u \in N(v)} \sum\limits_{k=1}^tx_{uv}^k = 1 ~~(v\in V\setminus S)\right\}$.
Let $t^{\ast}=\min\left\{t\in\mathbb{Z}_+: \Omega^=\neq\emptyset\right\}$ be the smallest value of $t$, for which the optimal objective function value in
the relaxation of \eqref{mod:basic:dec:obj}--\eqref{mod:basic:dec:dim} equals $V\setminus S$.
Existence of $t^{\ast}$ follows directly from $\Omega^=\left(\tau(G,S)\right)\neq\emptyset$.

Consider a modification of model \eqref{mod:basic:dec:obj}--\eqref{mod:basic:dec:dim} where the coverage constraint
$\sum\limits_{u \in N(v)} \sum\limits_{k=1}^tx_{uv}^k = 1$ is imposed on all non-source nodes $v\in V\setminus S$, and the objective is to minimize the communication time.
Its continuous relaxation is feasible for sufficently large $t$, and has optimal objective function value defined as
$\zeta(t)=\min\left\{\sum\limits_{k=1}^tz_k: x\in\Omega^=(t), z_k\geq\sum\limits_{u \in N(v)}x_{uv}^k ~~(k=1,\ldots,t, v\in V\setminus S)\right\}$.

\begin{proposition} \label{prop:lpweak}
For all $t\in\mathbb{Z}_+$ such that $\Omega^=(t)\neq\emptyset$, $\zeta(t)\leq t^{\ast}\leq\tau(G,S)$.
\end{proposition}
\begin{proof}
We first prove that $\zeta(t)$ is non-increasing with increasing $t$: Let $x\in\Omega^=(t)$, and define $\hat{x}\in[0,1]^{A\times\{1,\ldots,t+1\}}$ such that
for all $(u,v)\in A$, $\hat{x}_{uv}^k=x_{uv}^k$ ($k\leq t$) and $\hat{x}_{uv}^{t+1}=0$.
Then $\hat{x}\in\Omega^=(t+1)$.
An analogous extension of some $z\in[0,1]^{\{1,\ldots,t\}}$ satisfying $z_k\geq\sum\limits_{u \in N(v)}x_{uv}^k (k=1,\ldots,t)$, proves that $\zeta(t+1)\leq\zeta(t)$.

For $t\in\mathbb{Z}_+$ such that $\Omega^=(t)\neq\emptyset$, $t\geq t^{\ast}$ thus implies $\zeta(t)\leq\zeta(t^{\ast})$.
Since the only lower bound on $z_k$ in the definition of $\zeta(t)$ is $\max_{v\in V\setminus S}\sum\limits_{u \in N(v)}x_{uv}^k\leq 1$,
we get $\zeta(t)\leq\zeta(t^{\ast})\leq t^{\ast}$.
The proof is complete by observing that $t^{\ast}\leq\tau(G,S)$ follows from $\Omega^=\left(\tau(G,S)\right)\neq\emptyset$.
\end{proof}

\begin{remark} \label{rem:lpweak}
Proposition \ref{prop:lpweak} suggests to solve a sequence of instances of the continuous relaxation of problem \eqref{mod:basic:dec:obj}--\eqref{mod:basic:dec:dim},
and stop by the first value of $t$ for which the optimal objective function value is $\left|V\setminus S\right\|$.
Such an approach yields a lower bound ($t^{\ast}$) on $\tau(G,S)$,
which is no weaker than the bound achieved by solving an extended optimization model targeting minimum communication time.
\end{remark}

\begin{remark} \label{rem:otheropt}
Remark \ref{rem:lpweak} applies also if the definition of $\zeta(t)$ is modified to e.g.
$\min\left\{z: x\in\Omega^=(t), z\geq\sum\limits_{k=1}^tk\sum\limits_{u \in N(v)}x_{uv}^k ~~(v\in V\setminus S)\right\}$,
or other formulations reflecting minimization of communication time.
\end{remark}

\subsection{Combinatorial relaxations} \label{sec:lbcombrel}

Monotonous degree sequences of $G$ can be exploited more extensively than what was demonstrated in Section \ref{sec:lbanalyt}. 

Lower bounds on the broadcast time $\tau(G,S)$ are obtained by omitting one or more of the conditions imposed in Definition \ref{def:broadcasttime}.
For the purpose of strongest possible bounds, the relaxations thus constructed can be supplied with conditions that are redundant in the problem definition.

Recall from Section \ref{sec:def} that $D=(V,A)$ is the communication forest corresponding to $V_0,\ldots,V_t$, and $\pi$.
Conditions \ref{def:boundary}--\ref{def:unique} of Definition \ref{def:broadcasttime} imply that
\begin{enumerate}
\setcounter{enumi}{4}
  \item for all $v\in V$, $\delta_D^+(v)+\delta_D^-(v)\leq\delta_G(v)$. \label{def:degree}
\end{enumerate}

\noindent
A lower bound on $\tau(G,S)$ is then given by the solution to:
\begin{problem}[\textsc{Node Degree Relaxation}]\label{prob:degree}
Find the smallest integer $t\geq 0$ for which there exist
a sequence $V_0\subseteq\dots\subseteq V_t$ of node sets and a function $\pi:V\setminus S\to V$,
satisfying conditions \ref{def:boundary} and \ref{def:parent}--\ref{def:degree}.
\end{problem}

%Except from the node degrees, explicit information about $G$ is not part of the input to Problem \ref{prob:degree},
%and two nodes $u,v\in V\setminus S$, where $\delta_G(u)=\delta_G(v)$, are considered identical.

In order to develop an algorithm for the \textsc{Node Degree Relaxation}, we derive some optimality conditions.
Thus, let $S=\left\{v_1,\dots,v_{\sigma}\right\}$ and $V\setminus S=\left\{v_{\sigma+1},\ldots,v_n\right\}$, where $\delta_G(v_{\sigma+1})\geq\delta_G(v_{\sigma+2})\geq\dots\geq\delta_G(v_n)$,
and denote $d_i=\delta_G(v_i)$ ($i=1,\ldots,n$).

A degree sequence of a graph is commonly defined as a non-increasing sequence of degrees of nodes.
We however require, that first $\sigma$ positions correspond to degrees of sources, 
and the remaining $n-\sigma$ positions are degrees of the non-sources in a non-increasing order.
Given a graph with a degree sequence 
$Q=(d_1,\dots,d_n)$, we determine a lower bound by finding a smallest broadcast time among all possible graphs with degree sequence $Q$.
A modified verison of Problem \ref{prob:degree} takes a fixed $t$ and asks for a sequence $V_0,\dots,V_t$ minimizing $|V_t|$ while satisfying \ref{def:boundary} and \ref{def:parent}--\ref{def:degree}.
As the requirement \ref{def:edge} is relaxed, Problem~\ref{prob:degree} can be solved by iterative solution of its modified version for an increasing $t$.
As soon as $|V_t|=n$, $t$ is the optimal solution to Problem~\ref{prob:degree}.
Lemma \ref{lemma:degorder} states a sufficient optimality condition condition for the modified Problem~\ref{prob:degree}, and thereby also for Problem~\ref{prob:degree} itself.

\begin{lemma}
\label{lemma:degorder}
Maximum number of informed nodes within a given time limit $t\leq \bar{t} - \sigma$ in Problem~\ref{prob:degree} is achieved when nodes in $V\setminus S$ are informed 
in the order of their decreasing degree.
\end{lemma}
\begin{proof}

A node with degree $d_i$ informed in the $j$-th time step can inform at most $\min\{d_i-1,t-j\}$ nodes.
Consider integers  $k$ and $l$ such that $\sigma< k < \ell\leq n$ and nodes $v_k$, $v_\ell$ with degrees $d_k$, $d_\ell$, respectively.
Assume without loss of generality, that $v_k$ and $v_l$ are informed in the $k'$-th and $\ell'$-th time step, respectively, and that $k'\leq\ell'$.
Then, $d_k-1\geq d_\ell-1$ and $t-k' \geq t-\ell'$.
Let $K_1$ and $L_1$ be the number of nodes that can be informed by $v_k$ and $v_l$, respectively:
$$
K_1=\min\{d_k-1,t-k'\}, ~~~ L_1=\min\{d_\ell-1,t-\ell'\}.
$$
If the order in which $v_k$ and $v_l$ are informed is switched, i.e. when $v_k$ and $v_\ell$ is informed in the $\ell'$-th and $k'$-th time step, respectively, 
we use $K_2$ and $L_2$ to denote the maximum number of nodes informed by $v_k$ and $v_\ell$:
$$
K_2=\min\{d_k-1,t-\ell'\}, ~~~ L_2=\min\{d_\ell-1,t-k'\}.
$$
We now investigate what values the difference 
\begin{equation}
\label{eq:degbounds}
(K_1+L_1)-(K_2+L_2)
\end{equation}
can attain.
Observe that $K_1\geq L_2$ and $K_2\geq L_1$. 
If $K_2>L_1$, there are two distinct cases:
Either $K_2=d_k-1=K_1$, and then also $L_2=d_l-1=L_1$.
	Or $K_2=t-\ell'\leq K_1$ implying $L_1 =d_\ell-1$, and since $d_\ell-1\leq t-\ell'\leq t-k'$, we have $L_2=d_\ell-1$ as well.
We conclude that \eqref{eq:degbounds} always takes a non-negative value.
%Therefore, if the nodes are informed in the order of their decreasing degrees,
%the number of informed nodes within the give time step is no worse than when the nodes are informed in any other order.
Assume that the nodes are informed in an arbitrary order.
As long as it is possible to swap the order in which two nodes are informed so that a node with a higher degree becomes informed before a node with a lower degree,
thus whenever such $k$ and $l$ as above exist, swapping the order can not decrease the number of informed nodes within the given time limit.
The desired order of nodes can always be achieved by a sequence of swaps.
\end{proof}
For the relaxed problem, informing nodes in the desired order is clearly possible.
If $\mathcal{Q}$ is the set of graphs with degree sequence $Q$, a straightforward algorithm determining
$\min\{\tau(G,S): G\in \mathcal{Q}\}$,
and thus an optimal solution to Problem~\ref{prob:degree} can be developed.

Alg. \ref{alg:dreg} operates with the set $I(t)$ that contains nodes informed within the first $t$ time steps.
The function $a_i(t)$ determines the number of nodes informed by node $v_i$ within $t$ time steps, and finally,
the set $F(t)$ consists of nodes that in period $t$ can inform other nodes.
That is, $v\in F(t)$ if $v$ informed less than $\delta_G(v)$ other nodes, and $v$ itself is also informed.

\begin{algorithm}
\KwData{$I(0)=\{1,\dots,\sigma\}, a_1(0)=\dots = a_{n}(0)=0.$}
\For{$t=1,2,\dots$} {
	$F(t)=\{i\in I(t-1):a_i(t-1)<d_i\},$\\
	$I(t)=\{1,\dots,|I(t-1)|+|F(t)|\},$\\
	$a_i(t)=a_i(t-1)+
	\begin{cases}
		1, i\in F(t)\cup \left(I(t)\setminus I(t-1)\right) \\
		0, \text{ otherwise. }\\
	\end{cases}$\\
	\If{$|I(t)|=n$} {\Return $t$}
		
}
\caption{Lower bound exploiting distribution of degrees}
\label{alg:dreg}
\end{algorithm}


\begin{proposition}
Alg.~\ref{alg:dreg} returns a lower bound on $\tau(G,S)$.
\label{cor:deg}
\end{proposition}
\begin{proof}
		Follows from Lemma \ref{lemma:degorder} and the subsequent discussion. 
\end{proof}

