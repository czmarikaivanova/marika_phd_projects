\section{Lower bounds}
Strong lower bounds on the minimum objective function value are of vital importance to combinatorial optimization algorithms.
In this section, we study three types of lower bounds on the broadcast time $\tau(G,S)$.

\subsection{Analytical lower bounds} \label{sec:lbanalyt}
%An optimal solution is obtained by solving a sequence of decision problems with varying $t$. 
%It is therefore desirable to determine tight lower and upper bounds in order to arrive in the optimum after solving as few decision problems as possible.
Obvious bounds for a general graph instance are given by
\begin{observation}
For an instance $(G,S)$ of Problem \ref{prob:min},
\begin{equation}
\left\lceil\log\frac{n}{|S|}\right\rceil\leq \tau\{G,S\} \leq n-|S|.
\label{eq:loglb}
\end{equation}
\label{obs:loglb}
\end{observation}

In the following, we use $m$-step Fibonacci numbers \cite{noe05}, a generalization of the well known (2-step) Fibonacci numbers, defined by letting, 
$F^{(m)}_k=0$ for $k\leq 0$, $F^{(m)}_1=1$, and 
other terms according to the linear recurrence relation 
\begin{align*}
F^{(m)}_k &=\sum\limits_{i=1}^m F^{(m)}_{k-i}, &\text{ for } k\geq 2.
\end{align*}

% Possibly remove the subscript s in T_S
Assume that $G$ is a $d$-regular graph with a unique source $s$.
Any broadcast forest consists of a single tree $T_s$ rooted at $s$.
We investigate the number of leaves in $T_s$, and exploit this number to derive a lower bound on $\tau\{G,\{s\}\}$.

If the orientation of arcs in $T_s$ is disregarded, $|L(T^1_s)|=|L(T^2_s)|=2$.
For $i\geq 3$, $|L(T^i_s)|$ is no more than the number of nodes with degree below $d$ in $T^{i-1}_s$, 
because in a $d$-regular graph, only nodes with degree lower than $d$ can inform new uninformed nodes.
It can also be interpreted as the sum of the number of leaves in $T^{i-d+1}_s,\dots,T^{i-1}_s$, which leads to %the following formula
\begin{equation*}
\label{eq:leafrec}
|L(T^i_s)|=\sum\limits_{j=i-d+1}^{i-1} |L(T^j_s)|.
\end{equation*}  
This formula equals to the recursive definition of Fibonacci sequence of order $d-1$.
As each of the two base cases, $|L(T^1_s)|$ and $|L(T^2_s)|$, equals the double of the base cases of the Fibonacci sequence, the number of leaves in time step $i$ is calculated as
\begin{equation*}
\label{eq:fibleaf}
|L(T^i_s)|=2 F^{(d-1)}_i.
\end{equation*}  
The number of nodes in $T^i_s$ can be expressed as the sum of nodes newly informed in time steps $1,\dots,i$, that is, the sum of leaves in $T^1_s,\dots,T^i_s$. Thus,
\begin{equation}
\label{eq:fibcnt}
|V_{T^i_s}|=|V_i|=2\sum\limits_{j=1}^i F^{(d-1)}_j.
\end{equation}

\begin{proposition}
For a $d$-regular graph on $n$ nodes and $|S|$ sources, a lower bound on the delay is 
\begin{equation*}
\label{lem:lbreg1}
\underline{t}=\left\lceil\min\{k:2s\sum\limits_{j=1}^k F^{(d-1)}_j\geq n\}\right\rceil.
\end{equation*}
\label{prop:lbfib}
\end{proposition}
\begin{proof}
In order to inform $n$ nodes in the best possible scenario, the signal has to be relayed sufficiently many time steps so that the maximum possible number of informed nodes becomes $n$.
The maximum number of nodes informed within $i$ time steps is given by \eqref{eq:fibcnt}.
We therefore need to set the upper limit of the summation in \eqref{eq:fibcnt} so that the right-hand side exceeds $n$.
The reason why the result is divided by $|S|$ is that the best case scenario with several source nodes assumes that the signals initiated in individual sources are spread evenly.
\qed
\end{proof}

\subsection{Continuous relaxations of integer programming models} \label{sec:lblprel}

DH: \emph{To be written: Demonstrate why LP-bounds are expected to be weak (no stronger than the analytical bounds).}

\subsection{Combinatorial relaxations} \label{sec:lbcombrel}

Monotonous degree sequences of $G$ can be exploited more extensively than what was demonstrated in Section \ref{sec:lbanalyt}. 

Lower bounds on the broadcast time $\tau(G,S)$ are obtained by omitting one or more of the conditions imposed in Definition \ref{def:broadcasttime}.
For the purpose of strongest possible bounds, the relaxations thus constructed can be supplied with conditions that are redundant in the problem definition.

Recall from Section \ref{sec:def} that $D=(V,A)$ is the communication forest corresponding to $V_0,\ldots,V_t$, and $\pi$.
Conditions \ref{def:boundary}--\ref{def:unique} of Definition \ref{def:broadcasttime} imply that
\begin{enumerate}
\setcounter{enumi}{4}
  \item for all $v\in V$, $\delta_D^+(v)+\delta_D^-(v)\leq\delta_G(v)$. \label{def:degree}
\end{enumerate}

\noindent
A lower bound on $\tau(G,S)$ is then given by the solution to:
\begin{problem}[\textsc{Node Degree Relaxation}]\label{prob:degree}
Find the smallest integer $t\geq 0$ for which there exist
a sequence $V_0\subseteq\dots\subseteq V_t$ of node sets and a function $\pi:V\setminus S\to V$,
satisfying conditions \ref{def:boundary} and \ref{def:parent}--\ref{def:degree}.
\end{problem}

Except from the node degrees, explicit information about $G$ is not part of the input to Problem \ref{prob:degree},
and two nodes $u,v\in V\setminus S$, where $\delta_G(u)=\delta_G(v)$, are considered identical.

In order to develop an algorithm for the \textsc{Node Degree Relaxation}, we derive some optimality conditions of its single-source version.
Thus, let $|S|=\sigma$, $S=\left\{v_1,\dots,v_{\sigma}\right\}$ and $V\setminus S=\left\{v_{\sigma+1},\ldots,v_n\right\}$, where $\delta_G(v_{\sigma+1})\geq\delta_G(v_{\sigma+2})\geq\dots\geq\delta_G(v_n)$,
and denote $d_i=\delta_G(v_i)$ ($i=1,\ldots,n$).

Given a degree sequence 
\footnote{A degree sequence is commonly defined as a non-increasing sequence of degrees of nodes.
We however require, that first $\sigma$ positions correspond to degrees of sources, 
and the remaining $n-\sigma$ positions are degrees of the non-sources in a non-increasing order.} 
$Q=(d_1,\dots,d_n)$, we determine a LB by finding a smallest broadcast time of a graph that can be represented by $Q$.
Lemma \ref{lemma:degorder} characterizes such graphs.

\begin{lemma}
\label{lemma:degorder}
Maximum number of informed nodes within a given time limit $t\leq \bar{t} - \sigma$ in Problem~\ref{prob:degree} is achieved when nodes in $V\setminus S$ are informed 
in the order of their decreasing degree.
\end{lemma}
\begin{proof}

A node with degree $d_i$ informed in the $i$-th time step can inform at most $\min\{d_i-1,t-i\}$ nodes.
Consider integers  $k$ and $l$ such that $|S|< k < \ell\leq n$ and nodes $v_k$, $v_\ell$ with $\deg(v_k)=d_k$ and $\deg(v_\ell)=d_\ell$.
Then, $d_k-1\geq d_\ell-1$ and $t-k > t-\ell$.
If nodes are informed in the order of their decreasing degree, let $K_1$ and $L_1$ be the number of nodes that can be informed by $v_k$ and $v_l$, respectively:
$$
K_1=\min\{d_k-1,t-k\}, ~~~ L_1=\min\{d_\ell-1,t-\ell\}.
$$
If the order in which $v_k$ and $v_l$ are informed is switched, i.e. when $v_k$ and $v_\ell$ is informed in the $\ell$-th and $k$-th time step, respectively, 
we use $K_2$ and $L_2$ to denote the maximum number of nodes informed by $v_k$ and $v_\ell$:
$$
K_2=\min\{d_k-1,t-\ell\}, ~~~ L_2=\min\{d_\ell-1,t-k\}.
$$
We now investigate what values the difference 
\begin{equation}
\label{eq:degbounds}
(K_1+L_1)-(K_2+L_2)
\end{equation}
can attain.
Observe that $K_1\geq L_2$ and $K_2\geq L_1$. 
If $K_2>L_1$, there are two distinct cases:
Either $K_2=d_k-1=K_1$, and then also $L_2=d_l-1=L_1$.
Or $K_2=t-\ell\leq K_1$ implying $L_1 =d_\ell-1$, and since $d_\ell-1<t-\ell<t-k$, we have $L_2=d_\ell-1$ as well.
We conclude that \eqref{eq:degbounds} always takes a non-negative value.
%Therefore, if the nodes are informed in the order of their decreasing degrees,
%the number of informed nodes within the give time step is no worse than when the nodes are informed in any other order.
As long as it is possible to swap the order in which two nodes are informed so that a node with a higher degree becomes informed before a node with a lower degree,
thus whenever such $k$ and $l$ as above exist, swapping the order can not decrease the number of informed nodes within the given time limit.
\qed
\end{proof}
For the relaxed problem, informing nodes in the desired order is clearly possible.
If $\mathcal{Q}$ is the set of graphs represented by $Q$, a straightforward algorithm determining
$\min\{\tau(G,S): G\in \mathcal{Q}\}$,
and thus an optimal solution to Problem~\ref{prob:degree} can be developed.

Alg. \ref{alg:dreg} operates with the set $I(t)$ that contains nodes informed within the first $t$ time steps.
The function $a_i(t)$ determines the number of nodes informed by node $v_i$ within $t$ time steps, and finally,
the set $F(t)$ contains fertile nodes in time step $t$.
A node $v$ is called fertile in time step $t$ if $v$ informed less than $\delta_G(v)$ other nodes, and $v$ itself is also informed.

\begin{algorithm}
$\text{Let } I(0)=\{1,\dots,\sigma\}, a_1(0)=\dots = a_{n}(0)=0.$\\
\For{$t=1,2,\dots$} {
	$F(t)=\{i\in I(t-1):a_i(t-1)<d_i\},$\\
	$I(t)=\{1,\dots,|I(t-1)|+|F(t)|\},$\\
	$a_i(t)=a_i(t-1)+
	\begin{cases}
		1, i\in F(t)\cup \left(I(t)\setminus I(t-1)\right) \\
		0, \text{ otherwise. }\\
	\end{cases}$
}
$\text{Let } \tau(d_1,\dots,d_n,\sigma)=\min\{t=0,1,\dots:|I(t)|=n\}$
\caption{Lower bound exploiting distribution of degrees}
\label{alg:dreg}
\end{algorithm}


\begin{corollary}
Alg.~\ref{alg:dreg} finds a lower bound to Problem~\ref{prob:min}.
\label{cor:deg}
\end{corollary}

%A lower bound on $\tau(G,S)$ is then given by the solution to:
%\begin{problem}[\textsc{Node Degree Relaxation}]\label{prob:degree}
%Find the smallest integer $t\geq 0$ for which there exist
%a sequence $V_0\subseteq\dots\subseteq V_t$ of node sets and a function $\pi:V\setminus S\to V$,
%satisfying conditions \ref{def:boundary} and \ref{def:parent}--\ref{def:degree}.
%\end{problem}
%
%Except from the node degrees, explicit information about $G$ is not part of the input to Problem \ref{prob:degree},
%and two nodes $u,v\in V\setminus S$, where $\delta_G(u)=\delta_G(v)$, are considered identical.
%
%In order to develop an algorithm for the \textsc{Node Degree Relaxation}, we derive some optimality conditions of its single-source version.
%Thus, let $S=\left\{v_1\right\}$ and $V\setminus S=\left\{v_2,\ldots,v_n\right\}$, where $\delta_G(v_2)\geq\delta_G(v_3)\geq\dots\geq\delta_G(v_n)$,
%and denote $d_i=\delta_G(v_i)$ ($i=1,\ldots,n$).

%Consider an optimal communication tree $T$. That is, $T$ is a subtree of $G$ rooted at the unique source $v_1$ such that
%$\tau\left(G,\{v_1\}\right)=\tau\left(T,\{v_1\}\right)$.
%Let $\tau_i=\tau\left(T[i],\{v_i\}\right)$ denote the broadcast time of node $v_i$ in the subtree $T[i]$ of $T$ rooted at $v_i$.
%The optimal broadcast time $\tau\left(T,\{v_1\}\right)$ is hence no less than the sum of the time $\lceil\log\beta(v_i)\rceil$ it takes to inform $v_i$,
%and the subsequent time $\tau_i$ needed to inform all nodes in $T[i]$.
%Taking the largest of all such lower bounds, we get:
%\begin{equation}
%  \tau\left(T,\{v_1\}\right)=\max\left\{\lceil\log\beta(v_i)\rceil+\tau_i: i=2,\ldots,n\right\}. \label{eq:tau}
%\end{equation}
%
%\begin{lemma} \label{lem:degree}
%Problem \ref{prob:degree} has an optimal solution where, for all $i=2,\ldots,n-1$, $\delta_T^+(v_i)\geq\delta_T^+(v_{i+1})$ and $\tau_i\geq\tau_{i+1}$.
%\end{lemma}
%\begin{proof}
%Consider an optimal broadcast tree $T$, and let $j\in\{2,\ldots,n-1\}$ be the largest value of $i$ for which $\tau_i<\tau_{i+1}$.
%\emph{By utilizing (\ref{eq:tau}, go on to argue that a simple swap does not increase $\tau\left(\right)$}
%\emph{Slightly more tricky: Argue that moving subtrees can fix the first inequality. Swap again to restore the second}.
%\end{proof}

