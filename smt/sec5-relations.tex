\section{Relations Between the Models}
\label{sec:comp}

In this section, we analyze and compare the lower bounds provided by optimal solutions with the continuous relaxations of the models formulated in Section \ref{sec:ILP}.
For any integer programming models $\mathcal{M}$ and $\mathcal{M}'$ of SMT, we say that $\text{LP}(\mathcal{M})$ is \emph{at least as strong as} $\text{LP}(\mathcal{M}')$ if
$z\left(\text{LP}(\mathcal{M})\right)\geq z\left(\text{LP}(\mathcal{M}')\right)$ for all SMT-instances,
and that it is \emph{stronger} if there also exists an instance in which the inequality is strict.

\subsection{Comparing $\mathcal{X}_1$ and $\mathcal{F}_1$}
For all $(i,j)\in A$, define $F_{ij}^*=\max_{s\in D_0}F_{ij}^s$.
To prove that $\text{LP}(\mathcal{F}_1)$ is at least as strong as $\text{LP}(\mathcal{X}_1)$, the following claim is instrumental.
 
\begin{lemma} \label{lem:lpf1feas}
All feasible solutions $(F,g,\pi)$ to $\text{LP}(\mathcal{F}_1)$ satisfy $\sum_{k\in V_i}g_{ki}=1$ ($i\in D_0$) and $\sum_{k\in V_i}g_{ki}\leq 1$ ($i\in V\setminus D$).
\end{lemma}
\begin{proof}
Let $i\in D_0$. Utilizing first (\ref{con:pf1:fitt=xit}), next flow conservation (\ref{con:pf1:flow}) at $i=s$, and finally (\ref{con:pf1:noflowFromT}), we get
$\sum_{j\in V_i}g_{ji}=\sum_{j\in V_i}F_{ji}^i = 1+\sum_{j\in V_i}F_{ij}^i=1.$
For $i\in V\setminus D$, the claim is identical to (\ref{con:pf1:B}). \qed
\end{proof}
 
\begin{lemma} \label{lem:lpf1opt}
In all SMT-instances, there exists an optimal solution $(F,g,\pi)$ to $\text{LP}(\mathcal{F}_1)$ satisfying
\begin{subequations}
\begin{align}
 & \min\{F_{ij}^s, F_{ji}^s\} = 0 & \{i,j\}\in E, s\in D_0, \text{and} \label{eq:lemlpf1a} \\
 & g_{ij} = F_{ij}^* & (i,j)\in A. \label{eq:lemlpf1b}
\end{align}
\end{subequations}
\end{lemma}

\begin{proof}
Obviously, an optimal solution to $\text{LP}(\mathcal{F}_1)$ exists, so let $(\bar{F},\bar{g},\pi)$ denote one such solution.
For all $(i,j)\in A$ and $s\in D_0$, let $F_{ij}^s=\bar{F}_{ij}^s-\min\{\bar{F}_{ij}^s, \bar{F}_{ji}^s\}$, and let $g$ be chosen according to (\ref{eq:lemlpf1b}).
Consequently, (\ref{eq:lemlpf1a})--(\ref{eq:lemlpf1b}) are satisfied, and the objective function values of $(F,g,\pi)$ and $(\bar{F},\bar{g},\pi)$ are equal.

To prove optimality of $(F,g,\pi)$, it suffices to prove feasibility.
The capacity constraints (\ref{con:pf1:xfrel}) follow directly from (\ref{eq:lemlpf1b}).
Because $\bar{F}$ satisfies the flow conservation constraints (\ref{con:pf1:flow}), also $F$ does, since
$F_{ji}^s-F_{ij}^s=\bar{F}_{ji}^s-\bar{F}_{ij}^s$.
Further, $g_{ij}=F_{ij}^*\leq\bar{F}_{ij}^*\leq\bar{g}_{ij}$, proving that also (\ref{con:pf1:B}) and (\ref{con:pf1:yvar})--(\ref{con:pf1:yvar0}) are satisfied.
Finally, (\ref{con:pf1:noflowFromT})--(\ref{con:pf1:fitt=xit}) follow directly from $\bar{F}_{si}^s=0$ and $\bar{F}_{is}^s=\bar{g}_{is}$,
and (\ref{con:pf1:xi0=0}) follow from $g_{i0}\leq \bar{g}_{i0}=0$.
Thus, $(F,g,\pi)$ is feasible in $\text{LP}(\mathcal{F}_1)$.\qed
\end{proof}

\begin{lemma} \label{lem:lpf1gijgji}
If $(F,g,\pi)$ is feasible in $\text{LP}(\mathcal{F}_1)$ with (\ref{eq:lemlpf1a})--(\ref{eq:lemlpf1b}) satisfied, then
\begin{equation}
\begin{aligned}
 & g_{ij}+g_{ji} \leq 1 & \text{ for all } \{i,j\}\in E. \label{eq:lemlpf1c}
\end{aligned}
\end{equation}
\end{lemma}
\begin{proof}
Assume $g_{ij}+g_{ji}>1$ for some $\{i,j\}\in E$.
Because $g_{i0}=0$, this implies $i\neq s_0\neq j$.
By (\ref{eq:lemlpf1b}), there exist $s,t\in D_0$ such that $F_{ij}^s+F_{ji}^t>1$.
According to (\ref{eq:lemlpf1a}), $F_{ij}^s>1-F_{ji}^t\geq 0$ shows that $F_{ji}^s=0$.
Flow conservation at node $i$ thus implies $1<F_{ij}^s+F_{ji}^t\leq\sum_{k\in V_i\setminus\{j\}}F_{ki}^s+F_{ji}^t
\leq\sum_{k\in V_i\setminus\{j\}}g_{ki}+g_{ji}$, contradicting Lemma \ref{lem:lpf1feas}.\qed
\end{proof}

\begin{prop}
\label{prop:f1strx1}
$\text{LP}(\mathcal{F}_1)$ is at least as strong as $\text{LP}(\mathcal{X}_1)$. 
\end{prop}
\begin{proof}
Consider an optimal solution $(F,g,\pi)$ to $\text{LP}(\mathcal{F}_1)$ satisfying the conditions of Lemma \ref{lem:lpf1opt}.
We prove that there exist corresponding $X\in[0,1]^{A\times D}$ and $y\in[0,1]^E$ such that $(X,y,\pi)$ is a feasible solution to $\text{LP}(\mathcal{X}_1)$.
Let
\begin{subequations}
\begin{align}
 & X_{ij}^0=g_{ij} &  (i,j)\in A, \label{eq:trangX0}\\
 & X_{ij}^s=g_{ij}+F_{ji}^s-F_{ij}^s &  (i,j)\in A, s\in D_0, \label{eq:trangX}\\
 & y_{ij}=g_{ij}+g_{ji} &  \{i,j\}\in E. \label{eq:trangy}
\end{align}
\end{subequations}

\noindent
For distinct destinations $i,s\in D_0$, Lemma \ref{lem:lpf1feas} yields
$$\sum_{j\in V_i}X_{ji}^s=\sum_{j\in V_i}\left(g_{ji}+F_{ij}^s-F_{ji}^s\right)=\sum_{j\in V_i}g_{ji}=1,$$
proving that (\ref{con:dd:arrowFromDest}) is satisfied for $i\neq s_0\neq s$. 
By applying (\ref{con:pf1:xi0=0}), we get for $i=s_0\neq s$, that
$\sum_{j\in V_i}X_{ji}^s$ $=\sum_{j\in V_0}\left(g_{j0}+F_{0j}^s-F_{j0}^s\right)$ $=\sum_{j\in V_0}F_{0j}^s=1$,
where the latter identity is obtained by summing all flow conservation constraints (\ref{con:pf1:flow}) in which variables are superscripted by destination $s$.
Likewise, for $i\neq s_0=s$, we get 
$\sum_{j\in V_i}X_{ji}^s=\sum_{j\in V_i}g_{ji}=1$.
Hence, $X$ satisfies (\ref{con:dd:arrowFromDest}).

Also for $i\in V\setminus D$, we have
$\sum_{j\in V_i}X_{ji}^s=\sum_{j\in V_i}g_{ji}$, and constraints (\ref{con:dd:arrowFromNonDestB}) follow from (\ref{con:pf1:B}).
Constraints (\ref{con:dd:oneDir}) follow directly from the choice of values of $X$ and $y$,
and (\ref{con:dd:startInSource}) follows from (\ref{con:pf1:noflowFromT})--(\ref{con:pf1:fitt=xit}).
Further, $y_{ij}\in[0,1]$ holds according to Lemma \ref{lem:lpf1gijgji}, and $X_{ij}^s\in[0,1]$ follows from the capacity constraint (\ref{con:pf1:xfrel}).

It remains to prove (\ref{con:dd:arrowFromNonDestA}).
First, flow conservation at node $i$ is used to show
\[
  X_{ij}^s - \sum_{k\in V_{i}\setminus\{j\}}X^s_{ki} = g_{ij}+F_{ji}^s-F_{ij}^s - \sum_{k\in V_{i}\setminus\{j\}}\left(g_{ki}+F_{ik}^s-F_{ki}^s\right)
  = g_{ij}-\sum_{k\in V_{i}\setminus\{j\}}g_{ki}.
\]
It follows from condition (\ref{eq:lemlpf1b}) in Lemma \ref{lem:lpf1opt} that 
for some $t\in D_0$,
\[
  X_{ij}^s - \sum_{k\in V_{i}\setminus\{j\}}X^s_{ki}
  = F_{ij}^{t}-\sum_{k\in V_{i}\setminus\{j\}}g_{ki}
  \leq F_{ij}^{t}-\sum_{k\in V_{i}\setminus\{j\}}F_{ki}^{t}.
\]
If $F_{ij}^{t}=0$, the right hand side of the inequality is obviously non-positive.
To prove the same when $F_{ij}^{t}>0$, observe that (\ref{eq:lemlpf1a}) implies $F_{ji}^{t}=0$, which yields
\[
  F_{ij}^{t}-\sum_{k\in V_{i}\setminus\{j\}}F_{ki}^{t}
  = F_{ij}^{t}-\sum_{k\in V_{i}}F_{ki}^{t}
  \leq \sum_{k\in V_i}F_{ik}^{t}-\sum_{k\in V_{i}}F_{ki}^{t} = 0.
\]

Since the objective function values of $(F,g,\pi)$ and $(X,y,\pi)$ are identical in their respective relaxations, the proof is complete.
\qed\end{proof}

\begin{remark}
\label{rem:f1strx1}
Section \ref{sec:expcomplp} reports instances proving that $\text{LP}(\mathcal{F}_1)$ is \emph{stronger than} $\text{LP}(\mathcal{X}_1)$. 
\end{remark}

\subsection{Comparing $\mathcal{X}_2$ and $\mathcal{F}_2$}
For all $(i,j)\in A$ where $i,j\in V\setminus D$, let
$$\sigma_{ij}(F,g) = \min\left\{g_{ij} - F^*_{ij}, \sum_{k\in V_{i}}\left(g_{ik}-g_{ki}\right)\right\}.$$
That is, $\sigma_{ij}(F,g)$ is the smallest among slacks in constraints (\ref{con:pf1:xfrel}) at arc $(i,j)$ and constraint (\ref{con:pf1:flowX}) at node $i$.
 
\begin{lemma} \label{lem:lpf2}
In all SMT-instances, there exists an optimal solution $(F,g,\pi)$ to $\text{LP}(\mathcal{F}_2)$ satisfying (\ref{eq:lemlpf1a}),
\begin{subequations}
\begin{align}
 & \min\left\{g_{ij} - F_{ij}^*, g_{ji} - F_{ji}^*\right\} = 0 & \{i,j\}\in E, & \label{eq:lemlpf2a} \\
 & g_{ij} = F_{ij}^* & (i,j)\in A: \{i,j\}\cap D\neq\emptyset, & ~\text{and} \label{eq:lemlpf2b} \\
 & \sigma_{ij}(F,g)=0 & (i,j)\in A: \{i,j\}\subseteq V\setminus D. & \label{eq:lemlpf2c}
\end{align}
\end{subequations}
\end{lemma}

\begin{proof}
By following the initial step of the proof of Lemma \ref{lem:lpf1opt},
it is shown that $\text{LP}(\mathcal{F}_2)$ has an optimal solution $(F,\bar{g},\pi)$ satisfying (\ref{eq:lemlpf1a}).
For all $(i,j)\in A$, let $\delta_{ij}=\min\left\{\bar{g}_{ij}-F_{ij}^*, \bar{g}_{ji}-F_{ji}^*\right\}$,
and let $g_{ij}=\bar{g}_{ij}-\delta_{ij}$.
Hence, $(F,g)$ satisfies (\ref{eq:lemlpf2a}).
By exploiting $g\leq\bar{g}$, $g_{ij}-g_{ji}=\bar{g}_{ij}-\bar{g}_{ji}$, and feasibility of $(F,\bar{g},\pi)$,
it is seen straightforwardly that also $(F,g,\pi)$ is feasible.

For $j\in D_0$ (for $j=s_0$), (\ref{eq:lemlpf2b}) follows directly from (\ref{con:pf1:fitt=xit}) (from (\ref{con:pf1:xi0=0})).
Consider an $i\in D$ and $j\in V_i\setminus D$ such that $g_{ij}>F_{ij}^*$.
Reducing $g_{ij}$ to $F_{ij}^*$, while keeping all other variables unchanged, preserves feasibility and the objective function value.
Thus, $(F,g)$ is henceforth assumed to satisfy (\ref{eq:lemlpf2b}).

To prove that also (\ref{eq:lemlpf2c}) can be satisfied,
consider the \emph{slack-induced digraph} $H[F,g]=(V\setminus D,A[F,g])$ of non-destination nodes, where
% $A_H=\left\{(i,j)\in A:g_{ij} = g_{ji} + \sum_{k\in V_{i}\setminus\{j\}}\left(g_{ki}-g_{ik}\right) > F_{ij}^*\right\}$.
$A[F,g]=\left\{(i,j)\in A: i,j\in V\setminus D, F_{ij}^* < g_{ij}\right\}$.
Assume $C=(i_0,\ldots,i_{c-1}, i_c)$ is a cycle of length $c$ in $H$ ($i_c=i_0$),
and let $\varepsilon = \min\left\{g_{i_li_{l+1}}-F_{i_li_{l+1}}^*: l=0,\ldots,c-1\right\}>0$.
Define $g_{i_li_{l+1}}'=g_{i_li_{l+1}}-\varepsilon$ ($l=0,\ldots,c-1$), and $g_{ij}'=g_{ij}$ for arcs $(i,j)$ not in $C$.
Because $\sum_{j\in V_i}g_{ji}'-\sum_{j\in V_i}g_{ij}' = \sum_{j\in V_i}g_{ji}-\sum_{j\in V_i}g_{ij}$,
$g'$ satisfies (\ref{con:pf1:flowX}).
Since $g'\leq g$, it follows that $(F,g',\pi)$ is feasible in $\text{LP}(\mathcal{F}_2)$, with the same objective function value as $(F,g,\pi)$.
% Let $H'=(V,A_{H'})$ be defined analogously to $H$, but with $g$ replaced by $g'$.
Then, $A[F,g']\subset A[F,g]$, and for some $l=0,\ldots,c-1$, $(i_l,i_{l+1})\in A[F,g]\setminus A[F,g']$.
An induction argument hence proves the existence of an optimal solution to $\text{LP}(\mathcal{F}_2)$ for which the corresponding slack-induced digraph is acyclic.
It is henceforth assumed that $(F,g,\pi)$ is such a solution.

Assume arc $(j_0,j_1)$ violates (\ref{eq:lemlpf2c}).
We show that the capacity can be reduced to $\tilde{g}\leq g$ such that
\begin{itemize}
  \item $(F,\tilde{g},\pi)$ is feasible in $\text{LP}(\mathcal{F}_2)$,
  \item $\sigma_{ij}(F,\tilde{g})>0$ only for arcs $(i,j)$ where $\sigma_{ij}(F,g)>0$, and
  \item either $\sigma_{j_0j_1}(F,\tilde{g})=0$ or $A[F,\tilde{g}]\subsetneq A[F,g]$.
\end{itemize}
\noindent
Finitely many reductions, and replacements of $g$ by $\tilde{g}$, are then sufficient to achieve $\sigma_{j_0j_1}(F,g)=0$, which completes the proof.

Let $(j_1,\ldots,j_p)$ be a path with $p\geq 1$ nodes in $H[F,g]$, such that $j_p$ has no out-neighbors in $H[F,g]$.
Node $j_p$ exists since the digraph is acyclic.
Define the maximum capacity reduction as
\[ \sigma^*=\min\left\{\sigma_{j_0j_1}(F,g),\min\left\{g_{j_l,j_{l+1}}-F^*_{j_l,j_{l+1}}: l=1,\ldots,p-1\right\}\right\}>0.\]
Let $\tilde{g}_{j_lj_{l+1}}=g_{j_lj_{l+1}}-\sigma^*$ for $l=0,1,\ldots,p-1$, and let $\tilde{g}_{ij}=g_{ij}$ for all other arcs $(i,j)\in A$.
Then, $(F,\tilde{g},\pi)$ is feasible since, in comparison to $(F,g,\pi)$,  the left hand side of inequality (\ref{con:pf1:flowX}) is
increased by no more than the current slack at node $i=j_0$, reduced at node $i=j_p$, and unchanged at all other nodes $i\in V\setminus D$.
For all arcs $(i,j)$ between non-destination nodes, $\sigma_{ij}(F,\tilde{g})\leq\sigma_{ij}(F,g)$, except from arcs for which $i=j_p$.
But since $(j_p,j)\not\in A[F,g]$, we have that $\sigma_{j_pj}(F,\tilde{g})\leq\tilde{g}_{j_pj}-F^*_{j_pj}=g_{j_pj}-F^*_{j_pj}=0$.
%$\sigma_{j_pj}(F,g)>0$.
% and the capacity reduction introduces no new violations of (\ref{eq:lemlpf2a})--(\ref{eq:lemlpf2c}).
It is finally observed that $\sigma_{j_0j_1}(F,\tilde{g})=0$ if $\sigma^*=\sigma_{j_0j_1}(F,g)$,
and that $\left(j_l,j_{l+1}\right)\in A[F,g]\setminus A[F,\tilde{g}]$ if $\sigma^*=g_{j_l,j_{l+1}}-F^*_{j_l,j_{l+1}}$ for some $l=1,\ldots,p-1$.
\qed\end{proof}

\begin{lemma} \label{lem:lpf2gijgji}
If $(F,g,\pi)$ is a feasible solution to $\text{LP}(\mathcal{F}_2)$ satisfying the conditions in Lemma \ref{lem:lpf2}, then (\ref{eq:lemlpf1c}) is satisfied.
Moreover,
\begin{equation}
\begin{aligned}
 & g_{ij} \leq \sum_{k\in V_i\setminus\{j\}}g_{ki} & (i,j)\in A. \label{eq:lemlpf2d}
\end{aligned}
\end{equation}
\end{lemma}

\begin{proof}
To prove (\ref{eq:lemlpf2d}), we only consider arcs $(i,j)$ where $g_{ij}>0$, as the inequality to be proved obviously holds otherwise.

Assume first that $g_{ij}=F_{ij}^*$.
Then, there exists a destination node $t\in D_0\setminus\{i\}$ such that $F_{ij}^t=g_{ij}$.
Because (\ref{eq:lemlpf1a}) gives $F_{ji}^t=0$, flow conservation (\ref{con:pf1:flow}) at node $i$ implies 
$g_{ij}=F_{ij}^t\leq \sum_{k\in V_i\setminus\{j\}}F_{ki}^t\leq \sum_{k\in V_i\setminus\{j\}}g_{ki}$.

Assume next that $g_{ij}>F_{ij}^*$.
Conditions (\ref{eq:lemlpf2a}) and (\ref{eq:lemlpf2b}) of Lemma \ref{lem:lpf2} imply, respectively,
$g_{ji}=F_{ji}^*$ and $i,j\in V\setminus D$.
Because $g_{ij}>F_{ij}^*$, 
condition (\ref{eq:lemlpf2c}) of Lemma \ref{lem:lpf2} gives
$0=\sigma_{ij}(F,g) = \sum_{k\in V_{i}}\left(g_{ik}-g_{ki}\right)$. Thus,
\begin{equation}
  g_{ij} = g_{ji} + \sum_{k\in V_{i}\setminus\{j\}}\left(g_{ki}-g_{ik}\right). \label{eq:lemlpf2f}
\end{equation}
If $g_{ji}=0$, (\ref{eq:lemlpf2d}) follows directly from (\ref{eq:lemlpf2f}).
Otherwise, there exists a $t\in D_0$ such that $g_{ji}=F_{ji}^t$ and $F_{ij}^t=0$.
Flow conservation at $i$ then shows $g_{ji}=F_{ji}^t\leq\sum_{k\in V_{i}\setminus\{j\}}F_{ik}^t\leq\sum_{k\in V_{i}}g_{ik}$.
Adding this inequality to (\ref{eq:lemlpf2f}) proves (\ref{eq:lemlpf2d}).
% $g_{ij} \leq \sum_{k\in V_i\setminus\{j\}}g_{ik} + \sum_{k\in V_{i}\setminus\{j\}}\left(g_{ki}-g_{ik}\right) = \sum_{k\in V_{i}\setminus\{j\}}g_{ki}$.

To prove (\ref{eq:lemlpf1c}), assume $g_{ij}+g_{ji}>1$ for some $\{i,j\}\in E$.
By (\ref{eq:lemlpf2d}), this implies $1<g_{ij}+g_{ji}\leq\sum_{k\in V_{i}}g_{ki}$, contradicting Lemma \ref{lem:lpf1feas}.
\qed\end{proof}

\begin{prop}
\label{prop:f2strx2}
$\text{LP}(\mathcal{F}_2)$ is at least as strong as $\text{LP}(\mathcal{X}_2)$. 
\end{prop}
\begin{proof}
Consider an optimal solution $(F,g,\pi)$ to $\text{LP}(\mathcal{F}_2)$ satisfying the conditions of Lemma \ref{lem:lpf2}.
We prove that $(X,y,\pi)$ as defined by (\ref{eq:trangX0})--(\ref{eq:trangy}) is a feasible solution to $\text{LP}(\mathcal{X}_2)$.

The proof that $(X,y,\pi)$ satisfies constraints (\ref{con:dd:arrowFromDest})--(\ref{con:dd:arrowFromNonDestB}), (\ref{con:dd:oneDir})--(\ref{con:dd:startInSource}),
and (\ref{con:dd:yvar}),
follows the proof of Prop.\ \ref{prop:f1strx1}.
To prove (\ref{con:dd:arrowFromNonDestA}), let $i\in V\setminus D$, $j\in V_i$, and $s\in D$.
Then, transformations (\ref{eq:trangX0})--(\ref{eq:trangX}) yield
$X_{ij}^s - \sum_{k\in V_{i}\setminus \{j\}}X_{ki}^s
= g_{ij} - \sum_{k\in V_{i}\setminus \{j\}}g_{ki}$, and (\ref{con:dd:arrowFromNonDestA}) follows from (\ref{eq:lemlpf2d}) in Lemma \ref{lem:lpf2gijgji}.
Because Lemma \ref{lem:lpf2gijgji} also states that $g_{ij}+g_{ji}\leq 1$, we further have that $X_{ij}^s\in[0,1]$ and $y_{ij}\in[0,1]$.

Constraints (\ref{con:dd:extraCon}) hold because
$\sum_{j\in V_i}\left(X_{ji}^s-X_{ij}^s\right)=\sum_{j\in V_i}\left(g_{ji}-g_{ij}\right)\leq 0$
according to constraints (\ref{con:pf1:flowX}).
Finally, (\ref{con:vi:sumYImpSumXTrans}) implies
$\sum_{j\in V_i}X_{ji}^s-\sum_{j\in V_i\setminus\{s\}}\pi_{ij}^s=\sum_{j\in V_i}g_{ji}-\sum_{j\in V_i\setminus\{s\}}\pi_{ij}^s=0$,
which proves that $(X,\pi)$ satisfies (\ref{con:vi:sumYImpSumX}).
\qed\end{proof}

\begin{remark}
\label{rem:f2strx2}
Section \ref{sec:expcomplp} reports instances proving that $\text{LP}(\mathcal{F}_2)$ is \emph{stronger than} $\text{LP}(\mathcal{X}_2)$. 
\end{remark}
