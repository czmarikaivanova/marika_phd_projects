\documentclass[12pt]{article}
\usepackage{a4wide,url}
\usepackage{amsfonts}
\usepackage{amsmath}
\usepackage{color}

\begin{document}
\section*{Response Letter}
We are grateful to the reviewers for their time, effort, and constructive remarks.
All remarks have been accounted for in the revised version, as detailed in our response below.

\section*{Reviewer 2}
Our response in \textcolor{blue}{blue} follows the corresponding remark from the reviewer in \textcolor{red}{red}.

\begin{itemize}
\item \textcolor{red}{The abstract should be shorter.} \textcolor{blue}{
We have revised the abstract, and reduced its length from 22 to 15 lines.
}
\item \textcolor{red}{The literature review has not been organized in a proper format. The references are so old. New related works need to be explained. The authors should use a separate Section for this purpose and compare the novelty of the present approach with them.} \textcolor{blue}{
Following the recommendation of the reviewer, we have reorganized the literature review, and devoted a new section (Section 2 in the revised manuscript)
to this topic. We have also included a review of more recent and relevant literature (references Bein and Zheng (2010), Bhukya and Singh (2014)).
In Section 1.2, we now state explicitly the contributions from the present work, and explain how they relate to the current literature.
}
\item \textcolor{red}{They also should consider a separate section on the network model and assumptions.} \textcolor{blue}{
We have restructured the former Section 2 (now Section 3) into three subsections.
The first is devoted to mathematical notation and underlying assumptions, the second addresses the network model,
while the problem definition is given in the third subsection.
}
\item \textcolor{red}{The diagrams of figures are not distinguishable in black and white print. The legends should be corrected.} \textcolor{blue}{
While paying particular attention to their appearance in black and white print, we have given all figures,
including their legends, a complete overhaul.
}
\item \textcolor{red}{All equations should be terminated with "." or "," as well as sentences.} \textcolor{blue}{
Done.
}
\item \textcolor{red}{The legends, markers and other details of the figures are so small.} \textcolor{blue}{
Legends, markers, and other details are now given in larger size.
}
\item \textcolor{red}{A fundamental weak point is that there are no numerical results comparing the proposed scheme with the other new works;} \textcolor{blue}{
We agree that a comparison with alternative contemporary methods largely benefits the paper.
To that end, we have included a comparison with a recently published metaheuristic algorithm, based on iterative combination of elite solutions Pajor et al. (2018).
The method is described in the new Section 7, and results from experiments with the method are presented in the new Section 8.3.
}
\item \textcolor{red}{The simulation graphs need more expiation and justification.} \textcolor{blue}{
For better justification of the numerical results illustrated graphically in the former Section 6 (now Section 8),
we have added more explanations in the caption of Fig. 3.
}
\item \textcolor{red}{ The main aim of this paper is the energy, How is the energy issue modeled? Why is not the real models used in the paper?} \textcolor{blue}{
In the new Section 3.2 on the network model, we now explain more detailed how the energy is modeled.
At the end of the section, we also describe how our energy model relates to others in the literature.
Justification is added in terms of references to  Halgamuge et al. (2009), 
and to other relevant and frequently cited works mentioned in Section 2 (Wieselthier et al. (2000), Das et al. (2003), Altinkemer et al. (2005),
Yuan (2005), Yuan et al. (2008), Bauer et al. (2008), Montemanni and Leggieri (2011) and Leggieri et al. (2008), etc.).
See also our response to the second remark of Reviewer 3.
}
\item \textcolor{red}{Overall the novelty of the proposed approach in unclear and the paper is hard to follow
The authors have not considers the wireless problems in the paper.} \textcolor{blue}{
The novelty of and the contributions from the paper are now explicitly stated in Section 2.
We also explain how the paper relates to other works on energy minimization in wireless networks.
}
\end{itemize}

\section*{Reviewer 3}
Our response in \textcolor{blue}{blue} follows the corresponding remark from the reviewer in \textcolor{red}{red}.

\begin{itemize}
\item \textcolor{red}{The paper provides two new ILP formulations for the Shared Multicast Tree (SMT) problem in wireless ad hoc networks.
To address the problem of ILP scalability, the paper also provides a study and proofs of related theorems that shed light on the quality of
solutions achievable by  LP  bounds.
As the number and also provides a new constraint generation scheme.
Numerical results demonstrate the usefulness of the approach. The paper is well written.} \textcolor{blue}{
We are grateful to the reviewer for her/his encouraging remarks, and we appreciate the (s)he finds the paper well written.
}
\item \textcolor{red}{Overall, the methodology makes contribution in the field of optimization as the approach can be used to address other similar problems
that can be modeled as ILP which is not scalable.
However, this paper aims to address the problem of SMT in wireless ad hoc networks and it is
really motivated by the important problem of saving energy in sensor networks.
There I see a certain minor weakness.
The model considered ignores details of energy consumption is sensor network
(see for example Halgamuge et al., Progress in Electromagnetics Research, 12, 2009, or other sources).
It will be good to my opinion to point out this weakness and provide certain qualitative comments to benefit the practitioners,
on how the work can be used in practice, or what is still needed to be done so the work can be useful.} \textcolor{blue}{
We thank the reviewer for suggesting this very useful reference, a citation of which has been included in the revised version of the paper.
We fully agree that there are gaps between the coarse energy model applied in our work and more detailed energy models for energy consumption is sensor networks,
e.g.\ those analyzed by Halgamuge et al.\ \cite{halgamuge}.
These gaps, and their possible implications on practical usability of our approaches, are now discussed in the new Section 3.2.
}
\end{itemize}

\noindent
Final comment: 
Newly added parts of the paper are also in \textcolor{blue}{blue} font.
In several instances the text is restructured, but the sentences are not altered. In these cases the font color remains black.
Such updates should be apparent due to new highlighted sections.

We thank the reviewers again for their constructive remarks, which largely helped us to improve the presentation of our findings.

\end{document}
