\section{Computing near-optimal solutions} \label{sec:heur}

Instances of size that prohibits computation of the optimal solution should be approached by some fast, but possibly inexact method.
Imposing a time bound on an IP-solver, and applying it to any of the models introduced in Section \ref{sec:ILP}, readily gives such a method.
Alternatively, a heuristic method dedicated to SMT can be developed.

To the best of our knowledge, the scientific literature offers few heuristic methods for SMT besides those studied by \citet{ivanova16isco}.
However, recent progress by \citet{pajor18} on heuristic methods for the related minimum Steiner tree problem can smoothly be exploited in order to generate analogous methods for SMT.
The main algorithmic components in the most basic heuristic are a construction algorithm, a local improvement algorithm,
and an algorithm for merging two Steiner trees.

A set $\mathcal{T}=\left(T_1,\ldots,T_{|\mathcal{T}|}\right)$ of at most $\gamma$ distinct Steiner trees,
ordered by non-decreasing costs, is maintained throughout the heuristic \citep{pajor18}.
Construction and local improvement are applied in order to generate a new tree $T$ of presumably low cost.
Diversity in $\mathcal{T}$ is achieved by perturbing the edge costs randomly before construction and local improvement are applied.
If $\left|\mathcal{T}\right|<\gamma$, $T$ is included in $\mathcal{T}$.
If $\left|\mathcal{T}\right|=\gamma$, and the cost of $T$ is below the cost of $T_{\gamma}$, $T_{\gamma}$ is replaced by $T$ in $\mathcal{T}$.
Another Steiner tree, $\tilde{T}$, is generated by merging $T$ with a randomly chosen $T_r\in\mathcal{T}$ ($r=1,\ldots,\gamma$),
and $\tilde{T}$ replaces $T_{\gamma}$ in $\mathcal{T}$ if its cost is smaller.

\begin{algorithm}
\While{the time assigned to the heuristic is not expired} {
  \textbf{for} $\{i,j\}\in E$ \textbf{do} $\hat{p}_{ij}\leftarrow\text{random number uniformly distributed on
  $\left[\frac{1}{2}p_{ij},\frac{3}{2}p_{ij}\right]$}$\;
  $T\leftarrow\text{\textit{SMTHeur}$(G,D,\hat{p})$}$\;
  \textbf{if} $|\mathcal{T}|<\gamma$ \textbf{then} $\mathcal{T}\leftarrow \mathcal{T}\cup \{T\}$\;
  \Else {
    \textbf{if} $c(T)<c(T_{\gamma})$ \textbf{then} $\mathcal{T}\leftarrow \mathcal{T}\setminus\{T_{\gamma}\}$, $\mathcal{T}\leftarrow \mathcal{T}\cup \{T\}$\;
    $r\leftarrow\text{random number uniformly distributed on $\{1,\ldots,\gamma\}$}$\;
    \textbf{for} $\{i,j\}\in E_T\cap E_{T_r}$ \textbf{do} $\tilde{p}_{ij}\leftarrow p_{ij}$\;
    \For{$\{i,j\}\in \left(E_T\cup E_{T_r}\right)\setminus \left(E_T\cap E_{T_r}\right)$}{
      $\nu\leftarrow\text{random number uniformly distributed on $\{100,\ldots,500\}$}$\;
      $\tilde{p}_{ij}\leftarrow \nu p_{ij}$\;
    }
    \textbf{for} $\{i,j\}\in E\setminus E_T\setminus E_{T_r}$ \textbf{do} $\tilde{p}_{ij}\leftarrow 1000p_{ij}$\;
    $\tilde{T}\leftarrow\text{\textit{SMTHeur}$(G,D,\tilde{p})$}$\;
    \textbf{if} $c(\tilde{T})<c(T_{\gamma})$ \textbf{then} $\mathcal{T}\leftarrow \mathcal{T}\setminus\{T_{\gamma}\}$, $\mathcal{T}\leftarrow \mathcal{T}\cup \{\tilde{T}\}$\;
  }
}
\Return $T_1$

\caption{Outline of the heuristic method}
\label{alg:meta}
\end{algorithm}

Taking advantage of the recently published construction and local improvement algorithms for SMT \citep{ivanova16isco},
Alg.\ \ref{alg:meta} shows how the method by \citet{pajor18} is adapted to our problem.
The procedure \textit{SMTHeur}$(G,D,p)$ produces a multicast tree $T$ by first applying the construction algorithm to the input $(G,D,p)$,
and next applying local improvement to the constructed tree.
Following \citet{pajor18}, the perturbed cost $\hat{p}_{ij}$ of edge $\{i,j\}$ is put equal to a random number with expected value $p_{ij}$.
Further, in the merge process, adjusted weights $\tilde{p}_{ij}$ are determined in order to 
promote edges that appear in both $T$ and the randomly selected tree $T_r\in\mathcal{T}$. 
The cost of edges appearing in exactly one of the trees are thus magnified.
For the purpose of diversification, the factor by which the cost is multiplied is chosen randomly.
To discourage the appearance of edges outside both $T$ and $T_r$ in the merge $\tilde{T}$,
their cost is magnified by a factor dominating the factors drawn for edges that occur in one of the trees.

A few minor simplification over the approach taken by \citet{pajor18} are made in Alg.\ \ref{alg:meta}:
Rather than selecting the tree to leave $\mathcal{T}$ randomly, we consistently remove $T_{\gamma}$ when a new tree is added.
Also, for each new tree $T$ constructed, Alg.\ \ref{alg:meta} makes only one attempt to merge it with some $T_r\in\mathcal{T}$.
This contrasts the method in the cited work, where $\tilde{T}$ is chosen as the best tree found in a sequence of merges.

