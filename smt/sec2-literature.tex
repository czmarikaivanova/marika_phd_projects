\section{Literature Overview}
\label{sec:literature}
The combinatorial optimization literature is already rich on Integer Programming (IP) approaches to power minimization in wireless networks.
Since the work of \citet{Wieseltier00onthe}, in particular the MEB and MEM problems have been studied extensively.
Noteworthy contributions to this research include the IP models studied by \citet{das03},
\citet{altinkemer05}, \citet{leggieri08}, \citet{yuan05}, \citet{yuan08}, \citet{bauer08}, and \citet{montemanni11}.
\citet{cagalj02} proved that MEB, and thereby MEM, is NP-hard.
Numerous contributions to fast, but possible inexact methods for MEM have thus appeared, and the interested reader is referred to \citep{hsiao13} for an overview.

The SBT problem has also received some attention in the scientific literature.
It was introduced by \citet{Papadimitriou06SBT}, who also proved that the problem is NP-hard.
In the same article, an approximation algorithm for SBT was developed and analyzed experimentally.
Building on this work, \citet{Haugland12Dual} presented an integer programming model for SBT, along with an associated decomposition approach.

The SBT and SMT problems resemble the more frequently studied \textsc{Range Assignment Problem} (RAP) in the sense that they all ask for an undirected source-independent tree of minimum power.
However, RAP concerns cases where a connected topology must be established by means of \emph{bidirectional} links,
with the feature that power must be assigned to both the sending and the receiving device in each link.
This scenario induces a spanning tree problem, where costs are incurred only for edges that are the most expensive ones incident to any of their end nodes.
Simple instances \citep{Haugland12Dual} show that because of the difference in objective functions, the optimal solutions to RAP and SBT can differ substantially.
\citet{kirousis97} and \citet{clementi99} conduct
hardness results for instances of RAP embedded in three- and two-dimensional Euclidean space, respectively.
Later, IP models have been developed by e.g.\ \citet{althaus03}, \citet{montemanni04}, \citet{das05}, and \citet{Haugland11Compact}.

For the same reasons that MEB has been generalized to MEM, it is natural and desirable to extend SBT to its multicast equivalent.
The \textsc{Minimum Shared Multicast Tree} (SMT) problem is thus at the forefront of this paper,
and neither MEB, MEM, nor RAP will be considered further.
Focus is henceforth on instances in which some of the nodes neither initiate new messages nor need messages from others.
Utilizing such \emph{non-destination} nodes to forward messages can in many cases contribute to substantial reductions in the total power consumption.
Non-destination nodes thus have the potential to play the role of Steiner nodes in the multicast tree.
All devices that can initiate a transmission, referred to as \emph{destination} nodes, also have to receive every message.

Some preliminary work on SMT has recently been published.
\citet{ivanova16isco} introduces the problem, and develops an IP model.
Moreover, the author studies inexact construction and improvement methods, for which computational results are reported.
