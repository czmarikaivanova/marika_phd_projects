\section{Literature Overview}
\label{sec:literature}

The combinatorial optimization literature is already rich on Integer Programming (IP) approaches to power minimization in wireless networks.
Since the work of \citet{Wieseltier00onthe}, in particular the MEB and MEM problems have been studied extensively.
Noteworthy contributions to this research include the IP models studied by \citet{das03},
\citet{altinkemer05}, \citet{yuan05}, \citet{yuan08}, \citet{bauer08}, \citet{montemanni11} and \citet{leggieri08}. 
\textcolor{blue}{
The latter paper proposes an IP formulation of MEM based on a set covering model together with a preprocessing procedure reducing the number of constraints.}
\citet{cagalj02} prove that MEB, and thereby MEM, is NP-hard.
Numerous contributions to fast, but possible inexact methods for MEM have thus appeared, and the interested reader is referred to \citep{hsiao13} for an overview. 
\textcolor{blue}{Another study concerning MEM is the work by \cite{guo}, who investigate energy conservation while using adaptive antennas. 
Unlike omnidirectional antennas, adaptive antennas can concentrate their transmission power in a smaller angle.}

\textcolor{blue}{\citet{bein10} study energy efficiency in wireless networks where all nodes can initiate a message, and the communication takes place in a single all-to-all broadcast tree.
A heuristic algorithm generating all-to-all broadcast trees is proposed by \citet{bhukya14}, along with an experimental evaluation confirming its merit.}

The SBT problem has also received some attention in the scientific literature.
It was first introduced by \citet{Papadimitriou06SBT}, who also proved that the problem is NP-hard.
In the same article, an approximation algorithm for SBT is developed and analyzed experimentally.
Building on this work, \citet{Haugland12Dual} present an integer programming model for SBT, along with an associated decomposition approach.

Simple instances \citep{Haugland12Dual} show that because of the difference in objective functions, the optimal solutions to RAP and SBT can differ substantially.
\citet{kirousis97} and \citet{clementi99} conduct
hardness results for instances of RAP embedded in three- and two-dimensional Euclidean space, respectively.
Later, IP models have been developed by e.g.\ \citet{althaus03}, \citet{montemanni04}, \citet{das05}, and \citet{Haugland11Compact}.

Some preliminary work on SMT has recently been published.
\citet{ivanova16isco} introduces the problem, and develops an IP model.
Moreover, the author studies inexact construction and improvement methods, for which computational results are reported.
