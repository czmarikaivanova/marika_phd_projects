\section{Stronger Formulations}
\label{sec:strong}

Aiming for stronger LP bounds than those provided by the relaxations of the models introduced in Section\ \ref{sec:ILP},
this section is devoted to extensions of $\mathcal{X}_2$ and $\mathcal{F}_2$.
Variables with four node indices are added to both models.
Constraints connecting new and original variables then contribute to the improved bounds.

\subsection{Extension of $\mathcal{X}_2$ by destination-to-destination variables [$\mathcal{X}_3$]}
\label{sec:smtx3}

Let $A(D)=A\cap(D\times D)$ be the set of ordered pairs of distinct destination nodes.
For a tree $T$ solving Problem \ref{def:problem}, define the variables $x\in\{0,1\}^{A\times A(D)}$, where
$x_{ij}^{st}=1$ iff $(i,j) \in A$ is an arc on the path from $s$ to $t$ in arborescence $T^s$.
% \newline\newline
% $x_{ij}^{st}=
% \begin{cases}
%     1 & \text{if $(i,j) \in A$ is an arc on the path from $s$ to $t$ in $T$, $(s,t)\in A(D)$},\\
%     0 & \text{otherwise}.
% \end{cases}$
% \newline\newline
Model $\mathcal{X}_2$ is strengthened by constraints on $x$, which yields model $\mathcal{X}_3$: 
\newline
\newline
%%%%%%%%%%%%%%%%%%%%%%%%%%%%%%%%%%%%%%%
%                                     % 
%             MODEL X3                %
%             (4 idx vars)            %
%%%%%%%%%%%%%%%%%%%%%%%%%%%%%%%%%%%%%%%
\begin{subequations}[resume]\label{eq:smt-x3}
\begin{align}
\notag\makebox[0pt][l]{$\displaystyle{} \min \sum\limits_{(i,j) \in A} p_{ij} \sum\limits_{s \in D} \pi^s_{ij}, $}\\
\text{s.t.} \notag \\
(\ref{con:dd:arrowFromDest}) - (\ref{con:dd:vardim}) &\text{ and }(\ref{con:dd:yvar}) - (\ref{con:vi:sumYImpSumX}), \notag\\
\label{con:mf:flowNormal} \sum\limits_{\substack{ j\in V_i}}x^{st}_{ij}-\sum\limits_{\substack{j\in V_i }}x^{st}_{ji} & = 0 & (s,t)\in A(D), i\in V\setminus\{s,t\},\\
\label{con:mf:flowDest} \sum\limits_{\substack{j\in V_t}}x^{st}_{jt}    & = 1  & (s,t)\in A(D),\\
\label{con:mf:fcap} x^{st}_{ij} &\leq X^{s}_{ij}& (i,j)\in A, (s,t)\in A(D),\\
\label{con:mf:fsym} x^{st}_{ij} &=  x^{ts}_{ji}& (i,j)\in A, (s,t)\in A(D),\\ 
\label{con:vi:sumFImpSumY} \sum_{k\in W_{ij}}x^{st}_{ik} & \leq \sum_{k\in W_{ij}} \pi^{s}_{jk} & (i,j)\in A, (s,t)\in A(D),\\
\label{con:mf:xydim} x&\in\{0,1\}^{A\times A(D)}. 
\end{align}~
\end{subequations}

Variable $x_{ij}^{st}$ is also interpreted as the \emph{flow of commodity $(s,t)$} along arc $(i,j)$.
Correspondingly, (\ref{con:mf:flowNormal})--(\ref{con:mf:flowDest}) can be viewed as flow conservation constraints,
stating that there exists a path from $s$ to $t$ in $T^s$.
Next, constraints (\ref{con:mf:fcap}) express the obvious relation between $X$ and $x$, that $(i,j)$ is an arc on the path from $s$ to $t$ only if it exists in $T^s$.
The flow symmetry equations (\ref{con:mf:fsym}) state that the path from $s$ to $t$ contains $(i,j)$ iff the reverse path contains the reverse arc $(j,i)$.
Finally, constraints (\ref{con:vi:sumFImpSumY}) reflect the fact that if the path from $s$ to $t$ intersects $(i,j)$,
then the most expensive arc leaving $i$ in $T^s$ has cost $p_{ij}$ or more.

% In terms of multicast trees, $x_{ij}^{st}=1$ if and only if $(i,j)$ carries a signal that is initiated in $s$ and destined for $t$.
% Consider nodes $i,j\in V$ and a pair $(s,t)$ of destinations.
% If a message from $s$ to $t$ is sent through $(j,k)$ such that $p_{jk}\geq p_{ji}$, then the message from $s$ must be relayed by $j$ using power level at least $p_{ji}$.
% This is expressed by valid inequality (\ref{con:vi:sumFImpSumY}).

\subsection{Extension of $\mathcal{F}_2$ by variables of joint flow [$\mathcal{F}_3$]}

Similarly to the extension $\mathcal{X}_3$ of $\mathcal{X}_2$ by $x$-variables, the model $\mathcal{F}_2$ is extended by variables with four node indices.
Analogously to $A(D)$, let $E(D_0)=E\cap (D_0\times D_0)$ be the set of \emph{unordered} pairs of destination nodes other than $s_0$.
Define variables $f\in\{0,1\}^{A\times E(D)}$, where
$f_{ij}^{st}=f_{ji}^{ts}$ takes the value 1 iff the path in $T^0$ from $s_0$ to $s$ intersects with the path from $s_0$ to $t$ at arc $(i,j)$.
% \newline\newline
% $f^{st}_{ij}=f^{ts}_{ij}=
% \begin{cases}
%     1 & \text{if the paths in $T^0$ from $s_0$ to respectively $s$ and $t$ intersect} \\
%       & \text{at $(i,j)$,} \\
%     0 & \text{otherwise.}
% \end{cases}$
% \newline\newline
%
Variables $f_{ij}^{st}$ can also be regarded as the \emph{joint flow} from $s_0$ to $s$ and $t$ \citep{Polzin}.
In the cited work, they are successfully used to strengthen formulations for the \textsc{Minimum Steiner Tree} problem.
The new variables enable formulation of an extended model denoted $\mathcal{F}_3$: 

%%%%%%%%%%%%%%%%%%%%%%%%%%%%%%%%%%%%%%%
%                                     %
%             MODEL F3                %
%                                     %
%%%%%%%%%%%%%%%%%%%%%%%%%%%%%%%%%%%%%%%
\begin{subequations}[resume]
\begin{align}
\notag\min & \sum\limits_{(i,j) \in A} p_{ij} \sum\limits_{s \in D} \pi^s_{ij},\\ \notag
\text{s.t.}& && \\
\notag(\ref{con:pf1:flow})-(\ref{con:vi:sumYImpSumXTrans}) & ~\text{and} ~(\ref{con:vi:Y1}), \\
\label{con:pf2:flowHook} \sum\limits_{\substack{ j \in V_i }}\left(f^{st}_{ij}-f^{st}_{ji}\right) & \leq 
  \begin{cases}
    1, & i = s_0, \\
    0, & i\in V_0,
  \end{cases} \quad \{s,t\}\in E(D_0), \\
\label{con:pf2:startInSource}  f^{st}_{ij} & \leq F^s_{ij} \quad (i,j)\in A, \{s,t\}\in E(D_0),\\
\label{con:pf2:stopInDest}  f^{st}_{ij} & \leq F^t_{ij} \quad (i,j)\in A, \{s,t\}\in E(D_0),\\
\label{con:pf2:stronger}  F^{s}_{ij}+F^{t}_{ij}-f^{st}_{ij} &\leq g_{ij} \quad (i,j)\in A, \{s,t\}\in E(D_0),\\ 
\label{con:pf2:sumFImpSumY} \sum\limits_{k\in W_{ij}}\left(F^{t}_{ik} + F^{s}_{ki} - f^{st}_{ik} - f^{st}_{ki}\right) & \leq \sum\limits_{k\in W_{ij}}  \pi^{s}_{ik} \quad (i,j)\in A,\{s,t\}\in E(D_0),\\
\label{con:pf2:sumFImpSumY2} \sum\limits_{k\in W_{ij}}F^{t}_{ik} & \leq \sum\limits_{k\in W_{ij}}  \pi^{0}_{ik} \quad (i,j)\in A, t\in D_0,\\
\label{con:pf2:dim} f&\in\{0,1\}^{A\times E(D_0)}.
\end{align}~
\end{subequations}

By (\ref{con:pf2:flowHook}), it is ensured that the joint flow is non-increasing by increasing distance from the root $s_0$.
While constraints (\ref{con:pf2:startInSource})--(\ref{con:pf2:stopInDest}) ensure $f_{ij}^{st}=0$ if $F_{ij}^s=0$ or $F_{ij}^t=0$,
inequalities (\ref{con:pf2:stronger}) impose $f_{ij}^{st}=1$ if $F_{ij}^s=F_{ij}^t=1$.
By virtue of (\ref{con:pf2:stopInDest}), constraints (\ref{con:pf2:stronger}) replace the weaker capacity constraints (\ref{con:pf1:xfrel}).
Finally, (\ref{con:pf2:sumFImpSumY})--(\ref{con:pf2:sumFImpSumY2}) are valid inequalities justified by:
%%%%%%%%%%%%%%%%%%%%%%%%%%%%%%%%%%%%%%%
%                                     %
%         Proposition  		      %
%   (transformation x_{ij}^{st})      %
%                                     %
%%%%%%%%%%%%%%%%%%%%%%%%%%%%%%%%%%%%%%%
\begin{prop}\label{prop:transX}
Assume $(F,f,g,\pi)$ satisfies (\ref{con:pf1:flow})--(\ref{con:vi:sumYImpSumXTrans}), (\ref{con:vi:Y1}), (\ref{con:pf2:flowHook})--(\ref{con:pf2:stronger}), and (\ref{con:pf2:dim}).
Let $T^0\subseteq G_g$ be the corresponding arborescence spanning $D$ (see Lemma \ref{lem:modelcorrect}),
and let $T^s$ be the arborescence obtained by redirecting the arcs in $T^0$ such that they point away from $s\in D$. 
Then, for all nodes $i$ and $j$ ($i\neq j$) spanned by $T^0$,
$$
F^t_{ij}+F^s_{ji}-f^{st}_{ij}-f^{st}_{ji} = 
	\begin{cases}
		1, ~\text{if $(i,j)$ is an arc on the path from~$s$ to $t$ in $T^s$,} \\
		0, ~\text{otherwise.}
	\end{cases}
$$
\end{prop}
%
\begin{proof}
Because $g_{ij}+g_{ji}\leq 1$, we have $F^t_{ij}+F^s_{ji}\leq 1$.
Constraints (\ref{con:pf2:startInSource})--(\ref{con:pf2:stopInDest}) thus yield $f^{st}_{ij}+f^{st}_{ji}\leq 1$, and
$F^t_{ij}+F^s_{ji}-f^{st}_{ij}-f^{st}_{ji}\in \{0,1\}$ follows.
Inequalities (\ref{con:pf2:stronger}) in combination with (\ref{con:pf2:startInSource})--(\ref{con:pf2:stopInDest}) also imply $f^{st}_{ij}=F^t_{ij}F^s_{ji}$.

The path from $s$ to $t$ in $T^s$ contains $(i,j)$ iff $(i,j)\in A_{F^t}\setminus A_{F^s}$ or $(j,i)\in A_{F^s}\setminus A_{F^t}$.
This condition is equivalent to $F_{ij}^t=1-F_{ij}^s=1$ or $F_{ji}^s=1-F_{ji}^t=1$,
which holds iff $F_{ij}^t(1-F_{ij}^s)+F_{ji}^s(1-F_{ji}^t)=F_{ij}^t-f_{ij}^{st}+F_{ji}^s-f_{ji}^{st}=1$.
\qed
\end{proof}

Constraints (\ref{con:pf2:sumFImpSumY}) reflect the observation that if $(i,j)$ is on the path from $s$ to $t$ in $T^s$,
then the cost of the most expensive arc leaving $i$ in $T^s$ is at least $p_{ij}$.
Validity of (\ref{con:pf2:sumFImpSumY2}) is explained analogously, as $F_{ij}^t=1$ iff $(i,j)$ is on the path from $s_0$ to $t$ in $T^0$.

\subsection{Constraint Generation Scheme}
\label{sec:cg}
Owing to the large number of constraints in the stronger models, $\text{LP}(\mathcal{X}_3)$ and $\text{LP}(\mathcal{F}_3)$ are impractical for computing lower bounds, even in fairly small instances.
To enable computation of strong lower bounds, we first solve a reduced version of the LP, where some of the constraints are omitted.
Some of the relaxed constraints that are violated in the solution are next added, and the process is repeated until no violations exist.
This approach is known as a \emph{constraint generation} (CG) scheme.

\subsubsection{Implementation}%

Preliminary computational experiments indicate that solving $\text{LP}(\mathcal{F}_k)$ ($k=1,2$)
in order to compute a stronger bound than the one output from $\text{LP}(\mathcal{X}_k)$ is also more time-consuming.
The disproportion in running time is even more considerable in a comparison between $\text{LP}(\mathcal{F}_3)$ and $\text{LP}(\mathcal{X}_3)$ (see Section \ref{sec:expsmall}).
For these practical reasons, we apply the CG idea to $\text{LP}(\mathcal{X}_3)$.

Consider an optimal solution $(X,y,\pi)$ to $\text{LP}(\mathcal{X}_2)$, and let $X$ and $\pi$ be fixed.
Whether there exists an $x\in\left[0,1\right]^{A\times A(D)}$ satisfying (\ref{con:mf:flowNormal})-(\ref{con:mf:fcap}) can be checked by solving a maximum flow problem for each $(s,t)\in A(D)$.
The constraints state that it should be feasible to send one unit of flow from source $s\in D$ to destination $t\in D_s$,
such that the flow on arc $(i,j)$ is no more than $X_{ij}^s$.
Further, the symmetry constraints (\ref{con:mf:fsym}) implicitly impose the upper bound $X_{ji}^t$ on the same flow.
Finally, the side constraints (\ref{con:vi:sumFImpSumY}) impose upper bounds on the total flow on subsets of arcs leaving a given node.
By applying the symmetry $x_{ij}^{st}=x_{ji}^{ts}$ also to (\ref{con:vi:sumFImpSumY}), the extended maximum flow model, denoted $\mathcal{MF}$, is completed as
(the $x$-variables do not have the superscripts $s$ and $t$ as $\{s,t\}$ is fixed within each model instance):

%%%%%%%%%%%%%%%%%%%%%%%%%%%%%%%%%%%%%%%
%                                     %
%          Model 2MF                  %
%        (13a) - (13f)                %
%                                     %
%%%%%%%%%%%%%%%%%%%%%%%%%%%%%%%%%%%%%%%
\begin{subequations}\label{eq:mf}
\begin{align}
\label{objective:maxf} &\max\sum\limits_{i \in V_s} x_{si} \\ 
\text{s.t.}  \notag   \\
\label{con:maxf:flowNormal}  \sum\limits_{\substack{ j\in V_i}}\left(x_{ij}-x_{ji}\right) & = 0 & i\in V\setminus\{s,t\},\\
\label{con:maxf:fcapst}   x_{ij} &\leq \min\{X_{ij}^s, X_{ji}^t\}     &  (i,j)\in A,  \\ 	 
\label{con:maxf:sumFImpSumYst} \sum_{k\in W_{ij}}x_{ik} & \leq \sum_{k\in W_{ij}} \pi^{s}_{ik} & (i,j)\in A,\\
\label{con:maxf:sumFImpSumYts} \sum_{k\in W_{ij}}x_{ki} & \leq \sum_{k\in W_{ij}} \pi^{t}_{ik} & (i,j)\in A,\\
\label{con:maxf:fdim}x&\in\left[0,1\right]^{A}. 
\end{align}~
\end{subequations}  
  
For $Q\subseteq E(D)=\left\{\{s,t\}\in E: s,t\in D\right\}$, let $\mathcal{X}_2+Q$ denote the extension of $\mathcal{X}_2$ by constraints (\ref{con:mf:flowNormal})--(\ref{con:vi:sumFImpSumY}) for all $\{s,t\}\in Q$.
In the CG procedure described in Algorithm \ref{alg:cg},
we first solve $\text{LP}(\mathcal{X}_2)$ to obtain solution vectors $X$ and $\pi$.
For each $\{s,t\}\in E(D)$, we then check whether the relaxed constraints (\ref{con:mf:flowNormal})--(\ref{con:vi:sumFImpSumY}) can be satisfied,
given the values of $X$ and $\pi$.
This question is answered by solving $\mathcal{MF}$
since the objective function in $\mathcal{MF}$ takes optimal value 1 iff the constraints are satisfiable.
In each iteration,
relaxed constraints corresponding to a selection of pairs $\{s,t\}$ for which the maximum flow is below 1, are added to the original model.
How to make this selection is discussed next.

%%%%%%%%%%%%%%%%%%%%%%%%%%%%%%%%%%%%%%%
%                                     %
%         Algorithm 1                 %
%                                     %
%%%%%%%%%%%%%%%%%%%%%%%%%%%%%%%%%%%%%%%
\begin{algorithm}
$Q\leftarrow\emptyset$\\
\Do {$\Delta Q\neq\emptyset$}{
	$(X,y,\pi)\leftarrow\text{optimal solution to $\text{LP}(\mathcal{X}_2+Q)$}$\\
	\For{$\{s,t\}\in E(D)\setminus Q$} {
		$v_{st}\leftarrow \text{optimal objective function value in $\mathcal{MF}$}$\\
	}
	$Q'\leftarrow\left\{\{s,t\}\in E(D)\setminus Q: v_{st}<1\right\}$\\
	Select some $\Delta Q\subseteq Q'$ such that $\Delta Q\neq\emptyset$ if $Q'\neq\emptyset$\\
	$Q\leftarrow Q\cup\Delta Q$\\
}
 \caption{Constraint generation}
\label{alg:cg}
\end{algorithm}

\subsubsection{Constraint selection}

Preliminary experiments have demonstrated that in many instances, the optimal values of $X$ and $\pi$ in $\text{LP}(\mathcal{X}_2)$ are too restrictive to allow for unit maximum flow.
For most $\{s,t\}\in E(D)$, the objective function of $\mathcal{MF}$ takes a value below 1 in optimum.
Adding the constraints corresponding to all $\{s,t\}$ for which this occurs thus involves computations almost as extensive as solving $\text{LP}(\mathcal{X}_3)$ directly.

Selection of pairs $\{s,t\}$ to be included in $\Delta Q$ in Algorithm \ref{alg:cg} is based on the following intuitive considerations:
\begin{description}
  \item[\textbf{Adjacency matters}:] Extending $Q$ by $\{s,t\}$ is more likely to imply $v_{uw}=1$ in the following iteration if $\{s,t\}\cap\{u,w\}\neq\emptyset$ ($\{u,w\}\in E(D)\setminus Q$).
  \item[\textbf{Extent of violation matters}:] The growth of $z\left(\text{LP}(\mathcal{X}_2+Q)\right)$ increases by decreasing value of $v_{st}$.
\end{description}

Balancing the needs for rapid growth in $z\left(\text{LP}(\mathcal{X}_2+Q)\right)$ and small value of $|Q|$,
we pursue the CG scheme as follows:
In the graph $(D,Q')$, where $Q'=\{\{s,t\}\in E(D):v_{st}<1\}$,
%edges correspond to pairs $\{s,t\}$ at which $v_{st}<1$, 
let the edge weights be $1-v_{st}$.
Let $\Delta Q$ be a matching of maximum weight in $(D,Q')$.
Thus, for each $\{u,w\}\in E(D)\setminus Q$ where constraints (\ref{con:mf:flowNormal})--(\ref{con:vi:sumFImpSumY}) are not satisfiable given the current values of $X$ and $\pi$,
$Q$ is extended by an adjacent edge.

