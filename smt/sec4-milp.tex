\section{IP Formulations}
\label{sec:ILP}

A basic element of every IP formulation for SMT is a set of constraints modelling a Steiner tree.
We investigate two such constraint sets formulated in terms of variables with up to three node indices, and make necessary extensions required by the objective function of SMT.
The resulting models are subsequently strengthened by valid inequalities.

\subsection{Formulation based on broadcast trees}

The first model extends the SBT formulation introduced by \citet{Haugland12Dual} by non-destination nodes, in order to formulate the multicast version of the problem \citep{ivanova16isco}.

\subsubsection{Underlying Steiner tree formulation}

Feasible solutions to SMT are characterized by a Steiner tree $T$ covering $D$.
Define for each edge $\{i,j\}\in E$ and each destination node $s\in D$, the binary variables $y\in\{0,1\}^E$ and $X\in\{0,1\}^{A\times D}$,
where $y_{ij}=1$ iff $\{i,j\} \in E_T$,
and $X^{s}_{ij}=1$ iff $(i,j)\in A^s$.
% \[y_{ij}=
% \begin{cases}
%     1 & \text{if $\{i,j\} \in E_T$},\\
%     0 & \text{otherwise},
%   \end{cases}
% \]
% 
% \[X^{s}_{ij}=
% \begin{cases}
%     1 & \text{if $(i,j)$ is an arc in the arborescence $T^s$},\\
%     0 & \text{otherwise}.
%   \end{cases}
% \]
To induce a graph with an acyclic connected component spanning $D$, the following constraints are imposed:
%%%%%%%%%%%%%%%%%%%%%%%%%%%%%%%%%%%%%%%
%                                     % 
%     STEINER TREE MODEL  X0          %
%                                     %
%%%%%%%%%%%%%%%%%%%%%%%%%%%%%%%%%%%%%%%
\begin{subequations}\label{mod:x0}
\begin{align}
\label{con:dd:arrowFromDest} \sum\limits_{j\in V_i}X^s_{ji} & = 1 & i,s\in D,i\neq s,\\
\label{con:dd:arrowFromNonDestB} \sum\limits_{j\in V_{i}}X^s_{ji} & \leq 1 & i\in V \setminus D, s\in D,\\
\label{con:dd:arrowFromNonDestA} X^s_{ij} & \leq \sum\limits_{k\in V_{i}\setminus \{j\}}X^s_{ki} & i\in V \setminus D,(i,j)\in A, s\in D,\\
\label{con:dd:oneDir} X^s_{ij} + X^s_{ji} & = y_{ij} & \{i,j\}\in E, s\in D,\\
\label{con:dd:startInSource} X^s_{is} & = 0 &  s\in D, (i,s)\in A,&\\
\label{con:dd:vardim}y \in \{0,1\}^{E}, X & \in \{0,1\}^{A\times D}.
\end{align}~
\end{subequations}

Constraints (\ref{con:dd:arrowFromDest}) ensure that all destination nodes but $s$ have a unique entering arc in $T^s$.
Inequalities (\ref{con:dd:arrowFromNonDestB})--(\ref{con:dd:arrowFromNonDestA})
say that non-destination nodes do not have more than one entering arc, and that they do have one if they have leaving arcs.
Constraints (\ref{con:dd:oneDir}) state that the arc set of $T^s$ consists of directed versions of the edge set of $T$,
and (\ref{con:dd:startInSource}) ensure that $s$ is the root of $T^s$.
More formally, the effect of the constraints can be stated as follows:

\begin{lemma} \label{lem:x0valid}
If $(X,y)$ satisfies (\ref{con:dd:arrowFromDest})--(\ref{con:dd:vardim}), then $G_y$ has a tree spanning $D$ as one of its connected components.
Conversely, for any tree $T\subseteq G$ spanning $D$, there exists a pair $(X,y)$ satisfying (\ref{con:dd:arrowFromDest})--(\ref{con:dd:vardim}) such that $T$
is a connected component of $G_y$.
\end{lemma}
\begin{proof}
By (\ref{con:dd:startInSource}) and (\ref{con:dd:arrowFromDest})--(\ref{con:dd:arrowFromNonDestB}),
the in-degree of $s\in D$ is 0 in $G_{X^s}$, whereas it is at most one at all other nodes.
Hence, the connected components of $G_{X^s}$ are directed cycles, arborescences, and isolated nodes,
and one of the components is an arborescence rooted at $s$.
Constraints (\ref{con:dd:oneDir}) imply that $G_y$ consists of undirected versions of the same components,
showing that no cycle in $G_y$ contains destination nodes.
Let $t\in D$, and let $P$
be the maximal path in $G_{X^s}$ that terminates in $t$.
Existence and uniqueness of this path is assured since $t$ belongs to an acyclic connected component.
According to (\ref{con:dd:arrowFromNonDestA}), each non-destination in $P$ has an entering arc in $P$,
while (\ref{con:dd:arrowFromDest}) ensures that the same is true for every destination node, except $s$.
It follows that the first node of $P$ is $s$.
Thus, destinations $s$ and $t$ belong to a joint, acyclic connected component of $G_y$.

The second claim of the lemma is obvious.\qed
\end{proof}


\subsubsection{SMT model [$\mathcal{X}_1$]} \label{sec:x1}

To reflect that only expensive edges in $T$ contribute to the objective function of Problem \ref{def:problem}, we introduce the variables:
\newline\newline
$\pi^s_{ij}=
\begin{cases}
    1 & \text{if $(i,j)$ is the most expensive arc leaving node $i$ in $T^s$},\\
    0 & \text{otherwise},
  \end{cases}$
\newline\newline
leading to the following formulation $\mathcal{X}_1$ of SMT:

%%%%%%%%%%%%%%%%%%%%%%%%%%%%%%%%%%%%%%%
%                                     % 
%             MODEL X1                %
%           (3 idx vars)              %
%%%%%%%%%%%%%%%%%%%%%%%%%%%%%%%%%%%%%%%
\begin{subequations}[resume]\label{eq:smt-x2}
\begin{align}
\label{objective:mf} \makebox[0pt][l]{$\displaystyle{} \min \sum\limits_{(i,j) \in A} p_{ij} \sum\limits_{s \in D} \pi^s_{ij} $}\\
\notag\text{s.t.}\\
\notag(\ref{con:dd:arrowFromDest}) &- (\ref{con:dd:vardim})\\
\label{con:dd:yvar} X^s_{ij} & \leq \sum_{k\in W_{ij}}\pi^s_{ik} & s\in D, (i,j)\in A,\\
\label{con:dd:vardimy} \pi & \in \{0,1\}^{A\times D}.
\end{align}~
\end{subequations}

Constraints (\ref{con:dd:yvar}) say that if $(i,j)\in A_{T^s}$, then the cost of the most expensive arc in $T^s$ leaving $i$ is at least $p_{ij}$.
Since $\sum_{s\in D}\pi_{ij}^s=n_{ij}(T)$ is the number of arborescences in which $(i,j)$ is the most expensive arc leaving $i$,
(\ref{objective:mf}) thus agrees with the objective of Problem \ref{def:problem}.

\begin{prop}
\label{prop:x1valid}
If $(X,y,\pi)$ is an optimal solution to $\mathcal{X}_1$, then $G_y$ consists of isolated non-destination nodes and a tree that
solves Problem \ref{def:problem}.
\end{prop}
\begin{proof}
Follows directly from $p\geq 0$ and Lemma \ref{lem:x0valid}.
\qed
\end{proof}

Model $\mathcal{X}_1$ is a slightly modified version of the SMT model introduced by \citet{ivanova16isco},
which contains a set of constraints disallowing non-destination leaves, and a weaker version of constraints (\ref{con:dd:arrowFromNonDestA}).
Formulation (\ref{objective:mf})--(\ref{con:dd:vardimy}) is \emph{minimal},
in the sense that each set of constraints is necessary in order to give a valid formulation.
In the next section, we strengthen the formulation by adding redundant valid inequalities.

\subsubsection{Valid inequalities [$\mathcal{X}_2$]} \label{sec:x2}

Adding the following valid inequalities to $\mathcal{X}_1$ gives another SMT-model, denoted $\mathcal{X}_2$:
%%%%%%%%%%%%%%%%%%%%%%%%%%%%%%%%%%%%%%%
%                                     % 
%             MODEL X2                %
%      (3 idx vars + VIs )            %
%%%%%%%%%%%%%%%%%%%%%%%%%%%%%%%%%%%%%%%
\begin{subequations}[resume]
\begin{align}
\label{con:dd:extraCon} \sum\limits_{j\in V_{i}}X^s_{ji} & \leq \sum\limits_{j\in V_{i}}X^s_{ij}  & 	i\in V\setminus D, s\in D,\\
\label{con:vi:Y1} \sum\limits_{j\in V_s}  \pi^{s}_{sj} & =1 & s\in D,\\
\label{con:vi:sumYImpSumX} \sum\limits_{j\in V_i\setminus\{s\} }\pi^{s}_{ij} & = \sum\limits_{j\in V_i}  X^{s}_{ji} & i\in V\setminus D, s\in D.
\end{align}
\end{subequations}
Constraints (\ref{con:dd:extraCon}) ensure that the number of arcs leaving a non-destination in $T^s$ is no less than the number of entering arcs.
Their effect is to disallow non-destination leaves.
Equations (\ref{con:vi:Y1}) say that in arborescence $T^s$, exactly one arc $(s,j)$ leaving the root $s$ is declared to be largest.
Validity of the constraints follows immediately because the root has at least one leaving arc.
As constraints (\ref{con:vi:sumYImpSumX}) state, if an arc enters non-destination node $i$ in $T^s$, then $i$ also has at least one out-going arc,
exactly one of which is the most expensive.
The root $s$ is excluded from the summation set on the left of (\ref{con:vi:sumYImpSumX}) since no arcs in $T^s$ enter $s$.

\begin{remark} \label{rem:x2}
The sets (\ref{con:dd:extraCon})--(\ref{con:vi:sumYImpSumX}) of valid inequalities are \emph{independent} in the following sense:
There exist instances where the removal of either set implies that $z(\text{LP}(\mathcal{X}_2))$ takes a smaller value.
This includes the instance where $|V|=9$, $|D|=5$, all power requirements (arc costs) $p_{ij}$ equal the square of the Euclidean distance between nodes $i$ and $j$,
and the nodes are located in the plane with respective coordinates (destinations mentioned first)
(47, 65), (8, 65), (26, 12), (78, 27), (8, 8), (73, 66), (13, 55), (91, 44), and (76, 74).
\end{remark}

\subsection{Formulation based on network flows}

There are many formulations for the \textsc{Minimum Steiner Tree} problem \citep{goemans93catalog} that can serve as a basis for modelling SMT.
We consider the multi-commodity network flow model developed by \citet{Polzin} (denoted $P_{F}$ in their work).
A given destination $s_0\in D$ plays a particular role in the models.
For the sake of simplified notation, $s_0$ as subscript (superscript) is henceforth replaced by subscript (superscript) 0.

\subsubsection{Underlying Steiner tree formulation}

To identify a Steiner arborescence $T^0=(V^0,A^0)$ rooted at $s_0$, we define variables $g\in\{0,1\}^A$ and $F\in\{0,1\}^{A\times D_0}$,
where $g_{ij}=1$ iff $(i,j)\in A$, and $F_{ij}^s=1$ iff  the path in $T^0$ from $s_0$ to destination $s$ contains arc $(i,j)$.
% the following variables are introduced for each arc $(i,j)\in A$ and destination $s\in D_0$:
% \newline\newline
%   $g_{ij}=
% \begin{cases}
%     1 & \text{if $(i,j) \in A^0$},\\
%     0 & \text{otherwise},
% \end{cases}$
% \newline\newline
%   $F^{s}_{ij}=
% \begin{cases}
%     1 & \text{if the path from $s_0$ to $s$ in $T^0$ contains $(i,j)$},\\
%     0 & \text{otherwise}.
% \end{cases}$
% \newline\newline
%
Variables $g_{ij}$ and $F_{ij}^s$ are traditionally referred to as the \emph{design variable} of arc $(i,j)$ and the
\emph{flow of commodity $s$} along $(i,j)$, respectively.
The constraints of the \emph{multi-commodity flow formulation}
of the \textsc{Minimum Steiner Tree} problem are:
%%%%%%%%%%%%%%%%%%%%%%%%%%%%%%%%%%%%%%%
%                                     % 
%     STEINER ARBORESCENCE F0         %
%        (from Polzin)                %
%%%%%%%%%%%%%%%%%%%%%%%%%%%%%%%%%%%%%%%
\begin{subequations}
\begin{align}
\label{con:pf1:xfrel} F^{s}_{ij} & \leq g_{ij} & s\in D_0, (i,j)\in A, \\
\label{con:pf1:flow} \sum\limits_{\substack{ j \in V_i }}F^{s}_{ji}-\sum\limits_{\substack{j\in V_i}}F^{s}_{ij} &= 
  \begin{cases}
    1, & i=s,\\
    0, & i\in V\setminus \{s_0, s\},
  \end{cases} & s\in D_0,\\
\label{con:pf1:B}  \sum_{j\in V_i}g_{ji}&\leq 1 & i\in V\setminus D,\\
\label{con:pf1:noflowFromT} F_{si}^s&=0 & s\in D_0, i\in V_s,\\
\label{con:pf1:fitt=xit} F_{is}^s&=g_{is} & s\in D_0, i\in V_s,\\
\label{con:pf1:xi0=0} g_{i0}&=0 & i\in V_0,\\
\label{con:pf1:dim}g \in \{0,1\}^{A},F&\in\{0,1\}^{A \times D_0}.
\end{align}~
\end{subequations}

The \emph{flow conservation} constraints (\ref{con:pf1:flow}) ensure that $T^0$ is connected,
and the \emph{capacity} constraints (\ref{con:pf1:xfrel}) state that flow of any commodity is allowed only if $(i,j)\in A^0$.
Multi-commodity flow formulations appear abundantly in the network design literature, and proof of the following claim is therefore omitted:

%%%%%%%%%%%%%%%%%%%%%%%%%%%%%%%%%%%%%%%
%                                     %
%         Proposition  		      %
% (F1 models an arborescence)         %
%                                     %
%%%%%%%%%%%%%%%%%%%%%%%%%%%%%%%%%%%%%%%
\begin{lemma}
\label{lem:modelcorrect}
Assume $(F,g)$ satisfies (\ref{con:pf1:xfrel})--(\ref{con:pf1:dim}).
Then some connected component $T^0\subseteq G_{g}$ is an $s_0$-rooted arborescence spanning $D$,
and some connected component of $G_{F^s}\subseteq T^0$ is a path from $s_0$ to $s\in D_0$.
Conversely, for any $s_0$-rooted arborescence $T^0$ spanning $D$, there exists a pair $(F,g)$ satisfying (\ref{con:pf1:xfrel})--(\ref{con:pf1:dim}) such that $T^0$
is a connected component of $G_g$.
\end{lemma}
 
To model the objective function of SMT, the following result is useful:
%%%%%%%%%%%%%%%%%%%%%%%%%%%%%%%%%%%%%%%
%                                     %
%         Proposition  		      %
%   (transformation X_{ij}^s)         %
%                                     %
%%%%%%%%%%%%%%%%%%%%%%%%%%%%%%%%%%%%%%%
\begin{prop}\label{prop:transx}
Assume $(F,g)$ satisfies (\ref{con:pf1:xfrel})--(\ref{con:pf1:dim}),
and let $T^s=(V^s,A^s)$ be the arborescence obtained by redirecting the arcs in $T^0$ such that they point away from $s\in D_0$. Then
$$
g_{ij} - F^s_{ij}+F^s_{ji} = 
	\begin{cases}
		1, ~\text{if $(i,j)\in A^s$}, \\
		0, ~\text{otherwise.}
	\end{cases}
$$
\end{prop}
%
\begin{proof}
Because $T^0$ is acyclic, it follows from the capacity constraints (\ref{con:pf1:xfrel}) that $g_{ij}+F^s_{ji}\leq 1$.
The same constraints also imply $g_{ij} - F^s_{ij}\geq 0$.
Hence, $g_{ij} - F^s_{ij}+F^s_{ji}\in\{0,1\}$.

Arborescence $T^s$ contains arc $(i,j)$ iff $(i,j)$ is in $A^0$ but not on a path to $s$, or its reverse arc $(j,i)$ is on such a path.
By Lemma \ref{lem:modelcorrect}, this disjunction is equivalent to $g_{ij}-F_{ij}^s=1$ or $F_{ji}=1$.
The result then follows since the two conditions are mutually exclusive.
\qed
\end{proof}

% $\mathcal{F}_1$ based on $\mathcal{F}_0$.
% For this purpose, it is necessary to find a way to represent the constraint (\ref{con:dd:yvar}) in the $\mathcal{F}_1$ space.
% The $\pi$-variables from $\mathcal{X}_1$ have to be used in $\mathcal{F}_1$, because they appear in the objective function which remains unchanged.

\subsubsection{SMT model [$\mathcal{F}_1$]}

Recall from Section \ref{sec:x1} that $\pi_{ij}^s$ determines whether $(i,j)$ is the most expensive arc leaving node $i$ in arborescence $T^s$.
Taking advantage of Prop.\ \ref{prop:transx}, we can thus formulate Problem \ref{def:problem} in terms of variables $F$, $g$, and $\pi$,
resulting in model $\mathcal{F}_1$:
%%%%%%%%%%%%%%%%%%%%%%%%%%%%%%%%%%%%%%%
%                                     % 
%             MODEL F1                %
%        (3 idx vars)                 %
%%%%%%%%%%%%%%%%%%%%%%%%%%%%%%%%%%%%%%%
\begin{subequations}[resume]
\begin{align}
\notag\min & \sum\limits_{(i,j) \in A} p_{ij} \sum\limits_{s \in D} \pi^s_{ij}\\ 
\notag\text{s.t.} && \\
\notag(\ref{con:pf1:xfrel}) - (\ref{con:pf1:dim})\\
\label{con:pf1:yvar} g_{ij} - F^s_{ij}+F^s_{ji} & \leq \sum_{k\in W_{ij}}\pi^s_{ik}  & s\in D_0, (i,j)\in A,\\
\label{con:pf1:yvar0} g_{ij} & \leq \sum_{k\in W_{ij}}\pi^0_{ik}   &  (i,j)\in A,\\
\label{con:pf1:dimy} \pi&\in \{0,1\}^{A\times D}.
\end{align}~
\end{subequations}

Like (\ref{con:dd:yvar}), constraints (\ref{con:pf1:yvar}) ensure that if $(i,j)\in A^s$, then $\pi_{ik}^s=1$ for some
arc $(i,k)$ at least as expensive as $(i,j)$.
As $g_{ij}$ determines whether $(i,j)\in A^0$, the corresponding constraint for destination $s_0$ is simplified to (\ref{con:pf1:yvar0}).
Validity of $\mathcal{F}_1$ is thus stated as a corollary to Prop.\ \ref{prop:transx}:

\begin{cor}
If $(F,g,\pi)$ is an optimal solution to $\mathcal{F}_1$, then $G_g$ consists of isolated non-destination nodes and an arborescence $T^0$.
The tree obtained by disregarding arc directions in $T^0$ is an optimal solution to Problem \ref{def:problem}.
\end{cor}

\begin{remark} \label{rem:f1a}
Like $\mathcal{X}_1$, also $\mathcal{F}_1$ is a minimal formulation.
Constraints (\ref{con:pf1:B}) prevent non-destinations from having multiple entering arcs.
In a \textsc{Minimum Steiner Tree} formulation, where a linear function of $g$ is minimized subject to constraints
(\ref{con:pf1:xfrel})--(\ref{con:pf1:flow}) and (\ref{con:pf1:noflowFromT})--(\ref{con:pf1:dim}),
constraints (\ref{con:pf1:B}) are implied by optimality.
The necessity of (\ref{con:pf1:B}) in formulation $\mathcal{F}_1$ of SMT is demonstrated by the optimal solution to $\mathcal{F}_1$ after removal of (\ref{con:pf1:B})
in the following instance: $V=\{s_0,a,\ldots,i\}$ with coordinates
(3, 61),
(79, 80),
(32, 54),
(93, 21),
(12, 70),
(93, 60),
(72, 78),
(76, 80),
(39, 84), and
(82, 80), respectively, edge costs equal to the square of the Euclidean distance between end points, and $D=\{s_0,a,\ldots,f\}$.
Figure \ref{fig:BProof} depicts the optimal solution to SMT, and the solution to $\mathcal{F}_1$ after removal of (\ref{con:pf1:B}).
%The optimal solution with objective value 25156 to the depicted instance obtained by solving SMT-$\mathcal{X}_1$ is shown in Fig. \ref{fig:BorigSMT}.
%The solution in Fig \ref{fig:Bpf2} yielded by solving SMT-$\mathcal{F}_1$ without the constraint (\ref{con:pf1:B}) has objective value 25148, but is not a feasible solution to Problem \ref{def:problem}, because of the cycle $(g,h,i,d,g)$.
%The non-existence of such a cycle in a solution given by model $\mathcal{X}_1$ is ensured by constraints (\ref{con:dd:arrowFromDest}), (\ref{con:dd:arrowFromNonDestB}) and (\ref{con:dd:oneDir}).
%A detailed proof of this claim can be found in \cite{ivanova16isco}.
%A transmission commenced in node $c$ is sent via arc $(g,f)$.
%As a consequence of the link between $g$ and $d$ in Fig. \ref{fig:Bpf2}, the node $d$ also receives the message.
%This link is absent in Fig. \ref{fig:BorigSMT}, and so $i$ has to relay the signal using arc $(i,d)$, causing the higher total objective value.
\end{remark}
\begin{remark} \label{rem:f1b}
The obviously valid inequalities (\ref{con:pf1:noflowFromT})--(\ref{con:pf1:fitt=xit}) are not necessary in the \textsc{Minimum Steiner Tree} formulation,
but have to be included in the formulation $\mathcal{F}_1$ of SMT in order to disallow nodes in $D_0$ to have multiple entering arcs.
Removal of (\ref{con:pf1:noflowFromT}) or (\ref{con:pf1:fitt=xit}) implies that the optimal solution to the instance where
$V=\{s_0,a,\ldots,k\}$ with coordinates
(43, 15),
(89, 88),
(56, 12),
(44, 66),
(60, 42),
(91, 48),
(64, 8),
(64, 34),
(68, 9),
(41, 53),
(85, 67), and
(93, 49), respectively,
edge costs defined as in Remark \ref{rem:f1a}, and $D=\{s_0,a,\ldots,h\}$,
is not a feasible solution to SMT.
\end{remark}

% \begin{figure}[!htb]
%     \centering
%     \begin{subfigure}[b]{0.4\textwidth}
%         \includegraphics[width=\textwidth]{conBNec}
%         \caption{Optimal SMT-solution with total cost 25156.\newline~}
%         \label{fig:BorigSMT}
%     \end{subfigure}
%     \hfill
%     \begin{subfigure}[b]{0.4\textwidth}
%         \includegraphics[width=\textwidth]{conBNec2}
%         \caption{Optimal solution $G_g$ to $\mathcal{F}_1$ after omission of constraints (\ref{con:pf1:B}).
%                  Its total cost is 25148, but as it is acyclic, it is infeasible in SMT.}
%         \label{fig:Bpf2}
%     \end{subfigure}
%     \caption{Instance with $D=\{s_0,a,\ldots,e\}$ demonstrating necessity of constraints (\ref{con:pf1:B}) in $\mathcal{F}_1$.
% 		The edge labels denote costs.
% 		For better legibility, some edges are contracted in the illustration.} 
%     \label{fig:BProof}
% \end{figure}

%%%%%%%%%%%%%%%%%%%%%%%%%%%%%%%%%%%%%%%%%%%%%%%%%%%%%%
%      Figure                                        %
%    explaining why some constraints are necessary   %
%%%%%%%%%%%%%%%%%%%%%%%%%%%%%%%%%%%%%%%%%%%%%%%%%%%%%%
\newcommand{\drawnodes}{
\begin{scope}[every node/.style={circle,draw=black,fill=gray!40,minimum size=1.5em, inner sep=2, scale=.8}]
 \node (s0) at (11,10) {$s_0$};
 \node (a) at (9,11) {$a$};
 \node (b) at (8,8) {$b$};
 \node[fill=none] (f) at (5,11) {$f$};
 \node[fill=none] (g) at (4.5,5) {$g$};
 \node[fill=none] (h) at (4,3) {$h$};
 \node[fill=none] (i) at (4.5,0.8) {$i$};
 \node (c) at (5,2) {$c$};
 \node (d) at (8,1) {$d$};
 \node (e) at (12,0.5) {$e$};
\end{scope}
}
\begin{figure}[!htb]
\centering
\begin{subfigure}[b]{0.48\textwidth}
\begin{tikzpicture}[scale=.6]
 \drawnodes
 \begin{scope}[above=5pt, every edge/.style={draw=black,thick}, every node/.style={scale=.8}]
 \draw  (s0) edge node[right=5pt,above=-1pt]{162} (a);
 \draw  (a) edge node{925} (f);
 \draw  (f) edge node[right=2pt]{949} (b);
 \draw  (f) edge node[right=-1pt]{1125} (g);
 \draw  (g) edge node[left=-1pt]{20} (h);
 \draw  (h) edge node[right=3pt,above=-2pt]{9} (c);
 \draw  (c) edge node[left=3pt,above=-3pt]{9} (i);
 \draw  (i) edge node{521} (d);
 \draw  (d) edge node{1521} (e);
 \end{scope}
\end{tikzpicture}
\caption{Optimal SMT-solution with total cost 25156.\newline~}
\label{fig:BorigSMT}
\end{subfigure}
\hfill
\begin{subfigure}[b]{0.48\textwidth}
\begin{tikzpicture}[scale=.6]
\drawnodes
 \begin{scope}[->,above=5pt, every edge/.style={draw=black,thick}, every node/.style={scale=.8}]
 \draw  (s0) edge node[right=5pt,above=-1pt]{162} (a);
 \draw  (a) edge node{925} (f);
 \draw  (f) edge node[right=2pt]{949} (b);
 \draw  (f) edge node[right=-1pt]{1125} (g);
 \draw  (g) edge node[left=-1pt]{20} (h);
 \draw  (g) edge node[right=-1pt]{53} (c);
 \draw  (d) edge node[right=2pt]{765} (g);
 \draw  (h) edge node[left=6pt,above=-6pt]{36} (i);
 \draw  (i) edge node{521} (d);
 \draw  (d) edge node{1521} (e);
 \end{scope}
\end{tikzpicture}
\caption{Optimal solution $G_g$ to $\mathcal{F}_1$ after omission of constraints (\ref{con:pf1:B}).
         Its total cost is 25148, but the cycle shows that it is infeasible in SMT.}
\label{fig:Bpf2}
\end{subfigure}
\caption{Instance demonstrating necessity of constraints (\ref{con:pf1:B}) in $\mathcal{F}_1$.
         Edge labels denote costs.
         For better legibility, some edges are contracted in the illustration.} 
\label{fig:BProof}
\end{figure}

\subsubsection{Valid inequalities [$\mathcal{F}_2$]}

Valid inequalities analogous to those leading to model $\mathcal{X}_2$ in Section \ref{sec:x2} are added to $\mathcal{F}_1$, resulting in a model denoted $\mathcal{F}_2$.
Analogously to (\ref{con:dd:extraCon}), to enforce non-destination nodes included in the spanning arborescence to have at least one child node,
the following constraints are introduced \citep{Polzin}:
\begin{subequations}[resume]
\begin{flalign}
\label{con:pf1:flowX}  \sum\limits_{\substack{ j\in V_i }}g_{ji}-\sum\limits_{\substack{j\in V_i}}g_{ij}    & \leq 0    \qquad\qquad i\in V\setminus D.
\end{flalign}
\end{subequations}

According to Prop.\ \ref{prop:transx}, if $i\in V\setminus D$ and $g_{ji} - F^s_{ji}+F^s_{ij}=1$ for some $j\in V_i$ and $s\in D_0$,
then $i$ is a Steiner node used to establish connection between $s$ and some other destination.
Since $i$ also has a leaving arc in $T^s$, we have $\pi_{ik}^s=1$ for some $k\neq i,s$.
This proves validity of the constraint $\sum_{j\in V_i}\left(g_{ji} - F^s_{ji}+F^s_{ij}\right)=\sum_{j\in V_i\setminus\{s\}}\pi_{ij}^s$
Application of flow conservation (\ref{con:pf1:flow}) yields the valid constraints
%%%%%%%%%%%%%%%%%%%%%%%%%%%%%%%%%%%%%%%
%                                     %
%             MODEL F2                %
%                                     %
%%%%%%%%%%%%%%%%%%%%%%%%%%%%%%%%%%%%%%%
\begin{subequations}[resume]
\begin{flalign}
\label{con:vi:sumYImpSumXTrans} \sum\limits_{j\in V_i\setminus\{s\}}\pi^{s}_{ij} & = \sum\limits_{j\in V_i}  g_{ji}  \quad\quad   i\in V\setminus D, s\in D. 
\end{flalign}
\end{subequations}

\noindent
Finally, constraints (\ref{con:vi:Y1}) are valid,
completing formulation $\mathcal{F}_2$ as model $\mathcal{F}_1$ extended by (\ref{con:pf1:flowX})--(\ref{con:vi:sumYImpSumXTrans}) and (\ref{con:vi:Y1}).
%
\begin{remark} \label{rem:f2}
With $s_0$ equal to the destination node in position (47, 65), the instance given in Remark \ref{rem:x2} also proves independence of the three sets
(\ref{con:vi:Y1}), (\ref{con:vi:sumYImpSumXTrans})--(\ref{con:pf1:flowX}) of valid inequalities.
\end{remark}
