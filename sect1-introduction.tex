\section{Introduction}
\label{intro}
The minimum broadcast time (MBT) problem consists of a set of communication nodes with a subset of source nodes. 
The task is to disseminate a signal to every node in a shortest possible time (broadcast time), while abiding by communication rules.
An \emph{informed} node is a node that has received the signal.
Otherwise, a node is \emph{uninformed}.
At the beginning, the set of informed nodes is exactly the set of sources.
An informed node can send the signal to an uninformed node if the two nodes are located within a communication vicinity of each other.

The time is divided into a finite number of time steps.
Every informed node can, at each time step, forward the signal to at most one uninformed neighbor node.
Therefore, the number of informed nodes can at most be doubled from one step to the next.
%\subsection{Motivation and Related Work}
%This communication protocol differs from various wireless communication models where a signal can be relayed to all nodes within a visibility range of a sender.
This communication protocol appears in various practical application such as communication among computer processors or telephone networks.
%The applications are however not confined to wired networks.
Situations where the signals have to cover large distances typically assume sending the signal to one neighbor at a time.
This is common in satellite communication.

\subsection{Literature overview}

Deciding whether an instance of MBT has a solution with delay at most $t$ has been shown to be NP-complete \cite{slater81}. 
For bipartite planar graphs with maximum degree 3, NP-completeness persists even if $t=2$ or if there is only one source \cite{jansen95}.
When $t=2$, the problem also remains NP-complete for cubic planar graphs \cite{middendorf93}, grid graphs with maximum degree 3,
complete grid graphs, chordal graphs, and for split graphs \cite{jansen95}. 
The single-source variant of the decision version of MBT is NP-complete for grid graphs with maximum degree 4, and for chordal graphs \cite{jansen95}.
The problem is known to be polynomial in trees \cite{slater81}.
Whether the problem is polynomial or NP-complete for split graphs with a single source was stated as an open questions in \cite{jansen95}, and has to the best of our knowledge not been answered yet.

A number of inexact methods, for both general and special graph classes, have been proposed in the literature during the last three decades.
One of the first works of this category \cite{scheuermann84} 
introduces an exact dynamic programming algorithm based on generating all maximum matchings in an induced bipartite graph.
Additional contributions of \cite{scheuermann84} are heuristic approaches for near optimal broadcasting.
From more recent works we mention \cite{hasson04}, which describes a meta heuristic algorithm for MBT, and provides a comparison with other existing methods.
The communication model is considered in an existing satellite navigation system in \cite{chu17}, where a greedy inexact methods is proposed together with a non-linear mathematical model.
Examples of additional efficient heuristics can be found e.g. in \cite{harutyunyan06,harutyunyan14,wang10,jimborean13}.

Approximation algorithms for MBT are studied in \cite{kortsarz95}. 
The authors argue that methods presented in \cite{scheuermann84} provide no guarantee on the performance, and show that the $n$-node wheel is an example of an unfavourable instance.
They introduce an $\mathcal{O}(\sqrt{n})$-additive approximation algorithm for broadcasting in general graphs.
They further provide approximation algorithms for several graph classes with small separators with approximation ratio proportional to the separator size times $\log n$.
Another algorithm with $\mathcal{O}\left(\frac{\log n}{\log \log n}\right)$-approximation\footnote{All logarithms in this paper are of base 2} ratio is given in~\cite{elkin03}.
Most of the works cited above consider a single source.

A related problem extensively studied in the literature is the minimum broadcast graph problem \cite{grigni91,mcgarvey16}. 
A broadcast graph supports a broadcast from any node to all other nodes in optimal time $\lceil\log n\rceil$.
For a given integer $n$, the problem is to find a broadcast graph of $n$ nodes such that e.g.\ the number of edges, or the maximum node degree, in the graph is minimized.
The authors of \cite{mcgarvey16} study ILP models for $c$-broadcast graphs, a generalization of the problem that allows transmission of the signal to at most $c$ neighbours in a single time step.


