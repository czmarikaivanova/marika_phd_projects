% comment
% This is a general template file for the LaTeX package SVJour3mm
% for Springer journals.          Springer Heidelberg 2010/09/16
%
% Copy it to a new file with a new name and use it as the basis
% for your article. Delete % signs as needed.
%
% This template includes a few options for different layouts and
% content for various journals. Please consult a previous issue of
% your journal as needed.
%
%%%%%%%%%%%%%%%%%%%%%%%%%%%%%%%%%%%%%%%%%%%%%%%%%%%%%%%%%%%%%%%%%%%
%
% First comes an example EPS file -- just ignore it and
% proceed on the \documentclass line
% your LaTeX will extract the file if required
%\begin{filecontents*}{example.eps}
%!PS-Adobe-3.0 EPSF-3.0
%%BoundingBox: 19 19 221 221
%%CreationDate: Mon Sep 29 1997
%%Creator: programmed by hand (JK)
%%EndComments
%gsave
%newpath
%  20 20 moveto
%  20 220 lineto
%  220 220 lineto
%  220 20 lineto
%closepath
%2 setlinewidth
%gsave
%  .4 setgray fill
%grestore
%stroke
%grestore
%\end{filecontents*}
%
\RequirePackage{fix-cm}
%
%\documentclass{svjour3}                     % onecolumn (standard format)
%\documentclass[smallcondensed]{svjour3}     % onecolumn (ditto)
\documentclass[smallextended]{svjour3}       % onecolumn (second format)
\usepackage[T1]{fontenc}
%\documentclass[twocolumn]{svjour3}          % twocolumn
%
\smartqed  % flush right qed marks, e.g. at end of proof
%
\usepackage{graphicx}
\usepackage{tkz-graph}
\usepackage{amsmath, amssymb}
\usepackage{graphicx}
\usepackage{tikz}
\usepackage{amsfonts}
\usepackage{subcaption}
\usepackage{hyperref}
%\usepackage{pdflscape} % Landscape pages
\usepackage{color}	% Different font colors
\usepackage{enumerate}
\usepackage{enumitem}
\usepackage{mathtools}
\usepackage{color}
\usepackage[linesnumbered]{algorithm2e}
\SetKwRepeat{Do}{do}{while}

\captionsetup{compatibility=false}

\newtheorem{observation}[theorem]{\textbf{Observation}}

\tikzset{
every edge/.style={fill=none} 
every node/.style={shape=circle,fill=gray!40,draw }}



%\newtheorem{proposition}{Proposition}
%\newtheorem{obsservation}{Observation}
%\newtheorem{corollary}{Corollary}
%newtheorem{lemma}{Lemma}
%\newtheorem{problem}{Problem}
%
% \usepackage{mathptmx}      % use Times fonts if available on your TeX system
%
% insert here the call for the packages your document requires
%\usepackage{latexsym}
% etc.
%
% please place your own definitions here and don't use \def but
% \newcommand{}{}
%
% Insert the name of "your journal" with
% \journalname{myjournal}
%
\begin{document}

\title{Insert your title here%\thanks{Grants or other notes
%about the article that should go on the front page should be
%placed here. General acknowledgments should be placed at the end of the article.}
}
\subtitle{Do you have a subtitle?\\ If so, write it here}

%\titlerunning{Short form of title}        % if too long for running head

\author{First Author         \and
        Second Author %etc.
}

%\authorrunning{Short form of author list} % if too long for running head

\institute{F. Author \at
              first address \\
              Tel.: +123-45-678910\\
              Fax: +123-45-678910\\
              \email{fauthor@example.com}           %  \\
%             \emph{Present address:} of F. Author  %  if needed
           \and
           S. Author \at
              second address
}

\date{Received: date / Accepted: date}
% The correct dates will be entered by the editor


\maketitle

\begin{abstract}
\keywords{}
% \PACS{PACS code1 \and PACS code2 \and more}
% \subclass{MSC code1 \and MSC code2 \and more}
\end{abstract}

\section{Introduction}
\label{intro}
The minimum broadcast time (MBT) problem consists of a set of communication nodes with a subset of source nodes. 
The task is to disseminate a signal to every node in a shortest possible time (delay), while abiding by communication rules.
An \emph{informed} node is a node that has received the signal.
Otherwise, a node is \emph{uninformed}.
At the beginning, the set of informed nodes is exactly the set of sources.
An informed node can send the signal to an uninformed node if the two nodes are located within a communication vicinity of each other.

The time is divided into a finite number of time steps.
Every informed node can, at each time step, forward the signal to at most one uninformed neighbor node.
Therefore, the number of informed nodes can at most be doubled from one step to the next.
%\subsection{Motivation and Related Work}
%This communication protocol differs from various wireless communication models where a signal can be relayed to all nodes within a visibility range of a sender.
This communication protocol appears in various practical application such as communication among computer processors or telephone networks.
%The applications are however not confined to wired networks.
Situations where the signals have to cover large distances typically assume sending the signal to one neighbor at a time.
This is common in satellite communication.

{Literature overview}

Deciding whether an instance of MBT has a solution with delay at most $t$ has been shown to be NP-complete \cite{slater81}. 
For bipartite planar graphs with maximum degree 3, NP-completeness persists even if $t=2$ or there is only one source \cite{jansen95}.
When $t=2$, the problem also remains NP-complete for cubic planar graphs \cite{middendorf93}, grid graphs with maximum degree 3,
complete grid graphs, chordal graphs, and for split graphs \cite{jansen95}. 
The single-source variant of the decision version of MBT is NP-complete for grid graphs with maximum degree 4, and for chordal graphs \cite{jansen95}.
The problem is known to be polynomial in trees \cite{slater81}.
Whether the problem is polynomial or NP-complete for split graphs with a single source was stated as an open questions in \cite{jansen95}, and has to the best of our knowledge not been answered yet.

A number of inexact methods, for both general and special graph classes, have been proposed in the literature during the last three decades.
One of the first works of this category \cite{scheuermann84} 
introduces an exact dynamic programming algorithm based on generating all maximum matchings in an induced bipartite graph.
Additional contributions of \cite{scheuermann84} are heuristic approaches for near optimal broadcasting.
From more recent works we mention \cite{hasson04}, which describes a meta heuristic algorithm for MBT, and provides a comparison with other existing methods.
The communication model is considered in an existing satellite navigation system in \cite{chu17}, where a greedy inexact methods is proposed together with a non-linear mathematical model.
Examples of additional efficient heuristics can be found e.g. in \cite{harutyunyan06,harutyunyan14,wang10,jimborean13}.

Approximation algorithms for MBT are studied in \cite{kortsarz95}. 
The authors argue that methods presented in \cite{scheuermann84} provide no guarantee on the performance, and show that the $n$-node wheel is an example of an unfavourable instance.
They introduce an $\mathcal{O}(\sqrt{n})$-additive approximation algorithm for broadcasting in general graphs.
They further provide approximation algorithms for several graph classes with small separators with approximation ratio proportional to the separator size times $\log n$.
Another algorithm with $\mathcal{O}\left(\frac{\log n}{\log \log n}\right)$-approximation\footnote{All logarithms in this paper are of base 2} ratio is given in~\cite{elkin03}.
Most of the works cited above consider a single source.

A related problem extensively studied in the literature is the minimum broadcast graph problem \cite{grigni91,mcgarvey16}. 
A broadcast graph supports a broadcast from any node to all other nodes in optimal time $\lceil\log n\rceil$.
For a given integer $n$, the problem is to find a broadcast graph of $n$ nodes such that e.g.\ the number of edges, or the maximum node degree, in the graph is minimized.
The authors of \cite{mcgarvey16} study ILP models for $c$-broadcast graphs, a generalization of the problem that allows transmission of the signal to at most $c$ neighbours in a single time step.

\section{Network Model and Definitions}

The communication network is represented by a connected graph $G=(V,E)$ and a subset $S\subseteq V$ referred to as the set of sources, where $|V|=n$. 
% Let $V\left[F\right]=\{v\in V:\text{ some edge in } F \text{ is incident to } v\}, F\subseteq E$.
% For disjoint node sets $V_1,V_S\subseteq V$, we denote by $G\left[V_1,V_2\right]$ the bipartite graph $(V_1\cup V_2,E')$, where  $E'=\{\{u,v\}\in E: u\in V_1,v\in V_2\}$. 

\begin{definition} \label{def:broadcasttime}
The \emph{broadcast time} $\tau(G,S)$ of a node set $S\subseteq V$ in $G$ is defined as the smallest integer $t\geq 0$ for which there exist
a sequence $V_0\subseteq\dots\subseteq V_t$ of node sets and a function $\pi:V\setminus S\to V$, such that:
\begin{enumerate}
  \item $V_0=S$ and $V_t=V$, \label{def:boundary}
  \item for all $v\in V\setminus S$, $\{v,\pi(v)\}\in E$, \label{def:edge}
  \item for all $i=1,\ldots,t$ and all $v\in V_i$, $\pi(v)\in V_{i-1}$, and \label{def:parent}
  \item for all $u,v\in V_i\setminus V_{i-1}$, $\pi(u)=\pi(v)$ only if $u=v$. \label{def:unique}
\end{enumerate}
% % \begin{equation*}
% \tau(G,S)=
% \begin{cases}
% 	0, S=V,\\
% 	1+\min\{\tau(G,S\cup V\left[M\right]):M \text{ is a matching in } G\left[S, V\setminus S\right]\}, S\neq V.
% \end{cases}
% \label{eq:btime}
% \end{equation*}
\end{definition}
% %Broadcasting is defined as a sequence of sets $(S=V_0\subseteq\dots\subseteq V_k = V)$ where each $V_i$ represents the nodes informed after time step $i$, $0\leq i\leq t$.
% %For each node $v\in V_i\setminus V_{i-1}$, there exists a single node $\pi(v)\in V_{i-1}$ adjacent to $v$, which forwarded the signal to $v$.
% %Also, for every two nodes $u,v\in V_i\setminus V_{i-1}$ we have $\pi(u)=\pi(v)$.
% %The value $k$ is referred to as \emph{delay}.

Referring to Section \ref{intro}, the node set $V_i$ is the set of nodes that are informed in time step $i$.
Initially, only the sources are informed ($V_0=S$), whereas all nodes are informed after $t$ time steps ($V_t=V$),
and the set of informed nodes is monotonously non-decreasing ($V_{i_1}\subseteq V_i$ for $i=1,\ldots,t$).
The parent function $\pi$ maps each node to the node from which it received the signal.
Conditions \ref{def:edge}--\ref{def:parent} of Definition \ref{def:broadcasttime} thus reflect that the sender is a neighbor node in $G$,
and that it is informed at an earlier time step than the recipient node.
Because each node can send to at most one neighbor node in each time step,
condition \ref{def:unique} states that $\pi$ maps the set of nodes which are informed for the first time in step $i$ to distinct parent nodes.

The optimization problem in question is formulated as follows:
\begin{problem}[\textsc{Minimum Broadcast Time}]\label{prob:min}
Given $G=(V,E)$ and $S\subseteq V$, find $\tau(G,S)$.
% a sequence $(S=V_0\subseteq\dots\subseteq V_{\tau(G,S)}=V)$ and a function $\pi:V\setminus S\to V$, such that for each $v\in V\setminus S:\{v,\pi(v)\}\in E$, and for each $u,v\in V_i\setminus V_{i-1}: \pi(u)=\pi(v)\Leftrightarrow u=v$.
\end{problem}

\begin{definition} \label{def:broadcastgraph}
For any $V_0,\ldots,V_t$ and $\pi$ satisfying the conditions of Definition \ref{def:broadcasttime},
the corresponding \emph{communication forest} is
the digraph $D=(V,A)$, where $A=\left\{\left(v,\pi(v)\right): v\in V\right\}$.
Each connected component of $D$ is a \emph{communication tree}.
\end{definition}

\noindent
It is easily verified that the communication trees are indeed arborescences, rooted at distinct sources, with arcs pointing away from the source.
Let $T_s=\left(V_{T_s},A_{V_s}\right)$ denote the communication tree rooted at source $s\in S$, and define
the function $\alpha:V\to S$ such that $v\in V_{T_{\alpha(v)}}$ for all $v\in V$.
That is, $\alpha$ associates a node $v$ with the source in which the communication tree containing $v$ is rooted.
For any communication tree $T_s$, let $T^i_s$ be the subtree of $T$ induced by the set of nodes with distance at most $i$ edges from $s$.
Analogously, let $D^i=\left\{T_s^i: s\in S\right\}$.

% Given a feasible sequence of node sets and $\pi$, we derive the following broadcast trees \cite{grigni91}:
% For $s\in S$, a broadcast tree $T_s=(V_{T_s},E_{T_s})$ is a time-labeled directed subgraph of $G$ describing a broadcast originated in $s$ by the rules:
% \begin{enumerate}
% \item $T_s$ is rooted at $s$ with arcs directed towards the leaves.
% \item each node $v\in V_i$ is labeled with an integer $\tau(v)=i$,
% \item for $v\in V\setminus S, v\in V_{T_s}\Leftrightarrow \pi(v)\in V_{T_s}$, 
% \item $E_{T_s}=\{\{u,v\}: u,v\in V_{T_s}, \pi(v)=u\}$, and
% \item for two sources $s,s'\in S, s\neq s'$ we have that $V_{T_s}\cap V_{T_{s'}}=\emptyset$. 
% \end{enumerate}

% Any feasible sequence of node sets and $\pi$ yield a set of trees $T=\{T_s:s\in S\}$ that forms a partition of $G$ into trees referred to as \emph{broadcast forest}.

% Given a broadcast forest $T$, the delay is determined as $\max_{v\in V}\{\tau(v)\}$.
%A set of trees $T=\{T_s:\cup_{s\in S}V_{T_s}=V\}$ forms a partition of $G$ into trees referred to as \emph{broadcast forest}.
%$T$ can be derived from a given sequence of sets of nodes defining broadcasting and the mapping $\pi$.
% For any integer $i$, let $T^i_s$ be the subtree of $T_s$ obtained by pruning all nodes $v\in V_{T_s}$ with $\tau(v)>i$.
% Analogously, we define $T^i=\{T^i_s:s\in S\}$. 


%We also define the set $A=\{(i,j),(j,i):\{i,j\}\in E\}$ that consists of all arcs that can be derived by directing edges in $E$.
The degree of node $v$ in graph $G$ is denoted by $\text{deg}_G(v)$.
For the diameter of $G$ we use the symbol $\Delta_G$.
The set of neighbors of $v\in V$ in $G$ is denoted by $N_G(v)$.
Whenever there is no risk of confusion, the subscript $G$ is omitted.
Likewise, $\text{deg}^-_D(v)$ and $\text{deg}^+_D(v)$ denote, respectively, the in-degree and the out-degree of digraph $D$,
and $N_D^+(v)=\left\{u\in V:(v,u)\in A\right\}$ and $N_D^-(v)=\left\{u\in V:(u,v)\in A\right\}$ denote the neighbor sets of node $v$ in $D$.
The node set $L(D)=\left\{v\in V: N_D^+(v)=\emptyset\right\}$ is referred to as the \emph{leaves} in $D$.

%For convenience, we consider the following definition of broadcast trees \cite{grigni91}:
%For $s\in S$, a broadcast tree $T_s$ with node set $V_{T_s}$ is a time-labeled directed subgraph of $G$ describing a broadcast originated by $s$ by the following rules:
%\begin{enumerate}
%\item $T_s$ is rooted at $s$ with arcs directed towards the leaves.
%\item Each node $v$ is labeled with an integer $\tau(v)$, where $\tau(s)=0$.
%\item Whenever $v$ is a parent of $u$ in $T_s$ $\tau(v)<\tau(u)$.
%\item Whenever $v$ and $u$ are siblings in $T_s$, $\tau(v)\neq \tau(u)$. 
%\end{enumerate}
%The number of leaves in $T_s$ is denoted by $L(T_s)$.
\section{Exact methods} \label{sec:exact}


In this section, we formulate an ILP model for Problem \ref{prob:min}, and discuss possible solution strategies. 

\subsection{Broadcast time model}
Given an integer $t\geq\tau(G,S)$, define the variables ($(u,v)\in \overrightarrow{E}$, $k=1,\ldots,t$) 
$$ x_{uv}^k=
\begin{cases} 
1, \text{ if } v\in V_k\setminus V_{k-1} \text{ and } \pi(v)=u,\\ 
0, \text{ otherwise},
\end{cases}
z_{k}=\begin{cases}
1, \text{ if } k\leq t,\\
0, \text{ otherwise}.
\end{cases}
$$
\noindent
Variable $x_{uv}^k$ thus represents the decision whether or not the signal is to be transmitted from node $u$ to node $v$ in period $k$,
while $z_k$ indicates whether transmissions take place as late as in period $k$.

An upper bound $t$ on the broadcast time $\tau(G,S)$ is easily available.
Because $G$ is connected, the cut between any set $V_i$ of informed nodes and its complement is non-empty,
and therefore at least one more node can be informed in each period.
It follows that $\tau(G,S)\leq n-\sigma$.
The bound is tight in the worst case instance where $S=\{v_1\}$, and $G$ is a path with $v_1$ as one of its end nodes.
Problem \ref{prob:min} is then formulated as follows: 
\begin{subequations}\label{mod:basic}
\begin{align}
\label{mod:basic:obj} \min \sum\limits_{k=1}^{t}z_k \\ 
\text{s. t.~~~} \label{mod:basic:singlein} \sum\limits_{k=1}^{t}\sum\limits_{v\in N(u)}x_{vu}^k  = 1 && u\in V \setminus S,\\
%\label{mod:basic:uniqueTout} \sum\limits_{v\in N(u)}x_{uv}^k & \leq 1  & u\in V,k=1,\dots,\bar{t},\\
\label{mod:basic:tIncreases} \sum\limits_{v\in N(u)}x_{uv}^k \leq\sum\limits_{\ell=1}^{k-1}\sum\limits_{w\in N(u)} x_{wu}^{\ell}  && u\in V\setminus S, k=2,\dots,t,\\
%\label{mod:basic:tIncreases} x_{uv}^k &\leq\sum\limits_{\ell=1}^{k-1}\sum\limits_{w\in N(u)\setminus\{v\}} x_{wu}^{\ell}  & \{u,v\}\in E, u\not\in S, k=2,\dots,\bar{t},\\
%\label{mod:basic:tcrel} \sum\limits_{k=1}^{\bar{t}}k\cdot x_{uv}^k & \leq t^* &  (u,v)\in A,\\
\label{mod:basic:tcrel} \sum\limits_{v\in N(u)}x_{uv}^k  \leq z_k &&  u\in V,k=1,\dots,t,\\
\label{mod:basic:timing} z_k  \leq z_{k-1} &&  k=2,\dots,t,\\
%\label{mod:basic:tcrel} \sum\limits_{t=1}^{n-1}t\sum\limits_{j\in N(i)}x_{ij}^k & \leq c &  i\in V,\\
\label{mod:basic:positiveCost}x_{uv}^1  = 0 && (u,v)\in \overrightarrow{E}, u \in V\setminus S,\\
\label{mod:basic:dec:0toSource} x_{uv}^k  = 0  && (u,v)\in\overrightarrow{E}, v\in S, k=1,\ldots,t\\
\label{mod:basic:dim}x \in \{0,1\}^{\overrightarrow{E}\times \{1,\dots,t\}},z\in\{0,1\}^{\{1,\ldots,t\}}.&&
\end{align}~
\end{subequations}
By \eqref{mod:basic:singlein}, every non-source node $u$ receives the signal from exactly one adjacent node $v$ in some time step $k$.
%The requirement that a non-source node has a neighbor $v\in V_k$ such that $\pi(v)=u$ only if there exists a node $w\in V_{k-1}$ such that $\pi(u)=w$ is modeled by \eqref{mod:basic:tIncreases}. 
The requirement that a non-source node $u$ informs a neighbor $v$ in the $k$-th time step only if $u$ is informed by some adjacent node $w$ in an earlier time step is modeled by \eqref{mod:basic:tIncreases}. 
%Constraints \eqref{mod:basic:uniqueTout} enforce that for each node $u\in V$ and each subset $V_k$, there is at most one adjacent node $v\in V_k$ with $\pi(v)=u$.
Constraints \eqref{mod:basic:tcrel} enforce that each node $u\in V$ forwards the signal to at most one adjacent node $v$ in each time step.
It also sets correct values to the $z$-variable that appear in the objective function.
%The requirement that only informed nodes can relay a signal is modeled by \eqref{mod:basic:tIncreases}. 
%The maximum time step at which any transmission takes place is captured by \eqref{mod:basic:tcrel}, and finally, \eqref{mod:basic:positiveCost} states that a node that is not a source never transmits in the first time step.
%The length of the sequence of subsets is captured by \eqref{mod:basic:tcrel}, and finally, \eqref{mod:basic:positiveCost} state that if $\pi(v)\not\in S$ for some $j\in V$, then $v\not\in V_1$.
The valid inequalities \ref{mod:basic:timing} reflect that $k\leq\tau(G,S)$ only if $k-1\leq\tau(G,S)$.
Lastly, constraints \eqref{mod:basic:positiveCost} and \eqref{mod:basic:dec:0toSource} state, respectively, that non-source nodes do not transmit in the first time step,
and that sources never receive the signal.

\subsubsection{Decision version}
\label{sec:decbasic}
The nature of MBT suggests another modelling approach derived from Model \eqref{mod:basic}. 
For a given positive integer $t$, we maximize the number of nodes $v$ that receive a signal from some neighbor $u$ within $t$ time steps.
Hence, $\tau(G,S)$ is the smallest value of $t$ for which the maximum attains the value $n-\sigma$.
If the optimal objective function value is $n-\sigma$, the corresponding values of the decision variable induce a broadcast forest with broadcast time $t$.
Otherwise, if the optimal objective function value is smaller than $n-\sigma$, $t<\tau(G,S)$, and Model \eqref{mod:basic:dec} is solved again with $t$ replaced by $t+1$.
The computational efficiency of a search where $t$ is increased from some lower bound on $\tau(G,S)$ is largely dependent of the tightness of available upper and lower bounds on $\tau(G,S)$.
If an upper bound $\bar{t}\geq \tau(G,S)$ is known, and it is revealed that for $t=\bar{t}-1$ fewer than $n-\sigma$ can receive the signal in time $t$, it is concluded that $\tau(G,S)=\bar{t}$.

The decision version of a model for \textsc{Minimum Broadcast Time} takes the form
\begin{subequations}\label{mod:basic:dec}
\begin{align}
\label{mod:basic:dec:obj} \max \sum\limits_{v \in V\setminus S}\sum\limits_{u \in N(v)} \sum\limits_{k=1}^{t}x_{uv}^k \\ 
\text{s. t.~~~} \label{mod:basic:dec:atMost1in} \sum\limits_{k=1}^{t}\sum\limits_{v\in N(u)}x_{vu}^k  \leq 1 && u\in V \setminus S,\\
\eqref{mod:basic:tIncreases},  \eqref{mod:basic:positiveCost}, \eqref{mod:basic:dec:0toSource}, \\
\label{mod:basic:dec:dim}x \in \{0,1\}^{\overrightarrow{E}\times \{1,\dots,t\}}.&&
\end{align}~
\end{subequations}

In the transition from the optimization model \eqref{mod:basic}, constraint \eqref{mod:basic:singlein} is replaced by \eqref{mod:basic:dec:atMost1in}.
The former is an inequality in the decision version, because not all nodes are necessarily reached within the given time limit.



\section{Lower bounds}
In this section, we study lower bounds on the delay for several restrictions of input graphs.
%An optimal solution is obtained by solving a sequence of decision problems with varying $t$. 
%It is therefore desirable to determine tight lower and upper bounds in order to arrive in the optimum after solving as few decision problems as possible.
Obvious bounds for a general graph instance are given by
\begin{observation}
For an instance $(G,S)$ of Problem \ref{prob:min},
$$\left\lceil\log\frac{n}{|S|}\right\rceil\leq t^* \leq n-|S|.$$
\end{observation}

In the following, we use $m$-step Fibonacci numbers \cite{noe05}, a generalization of well known (2-step) Fibonacci numbers, defined by letting, 
$F^{(m)}_k=0$ for $k\leq 0$, $F^{(m)}_1=1$, and 
other terms according to linear recurrence relation 
\begin{align*}
F^{(m)}_k &=\sum\limits_{i=1}^m F^{(m)}_{k-i}, &\text{ for } k\geq 2.
\end{align*}

Consider a $d$-regular graph with a unique source $s$.
Any broadcast forest consists of a single tree $T_s$.
We investigate the number of leaves in $T_s$, and derive a lower bound on the delay for this graph class.

If the orientation of arcs in $T_s$ is disregarded, $L(T^1_s)=L(T^2_s)=2$.
For $i\geq 3$, $L(T^i_s)$ equals the number of nodes with degree $1,\dots,d-1$ in $T^{i-1}_s$, 
because in a $d$-regular graph, only nodes with degree lower than $d$ can inform new uninformed nodes.
It can also be interpreted as the sum of the number of leaves in $T^{i-d+1}_s,\dots,T^{i-1}_s$, which leads to %the following formula
\begin{equation*}
\label{eq:leafrec}
L(T^i_s)=\sum\limits_{j=i-d+1}^{i-1} L(T^j_s).
\end{equation*}  
This formula equals to the recursive definition of Fibonacci sequence of order $d-1$.
As each of the two base cases, $L(T^1_s)$ and $L(T^2_s)$, equals the double of the base cases of the Fibonacci sequence, the number of leaves in time step $i$ is calculated as
\begin{equation*}
\label{eq:fibleaf}
L(T^i_s)=2 F^{(d-1)}_i.
\end{equation*}  
The number of nodes in $T^i_s$ can be expressed as the sum of nodes newly informed in time steps $1,\dots,i$, that is, the sum of leaves in $T^1_s,\dots,T^i_s$. Thus,
\begin{equation}
\label{eq:fibcnt}
|V_{T^i_s}|=|V_i|=2\sum\limits_{j=1}^i F^{(d-1)}_j.
\end{equation}

\begin{proposition}
For a $d$-regular graph on $n$ nodes and $|S|$ sources, a lower bound on the delay is 
\begin{equation*}
\label{lem:lbreg1}
\underline{t}=\left\lceil\min\{k:2s\sum\limits_{j=1}^k F^{(d-1)}_j\geq n\}\right\rceil.
\end{equation*}
\end{proposition}
\begin{proof}
In order to inform $n$ nodes in the best possible scenario, the signal has to be relayed sufficiently many time steps so that the maximum possible number of informed nodes becomes $n$.
The maximum number of nodes informed within $i$ time steps is given by \eqref{eq:fibcnt}.
We therefore need to set the upper limit of the summation in \eqref{eq:fibcnt} so that the right-hand side exceeds $n$.
The reason why the result is divided by $|S|$ is that the best case scenario with several source nodes assumes that the signals initiated in individual sources are spread evenly.
\qed
\end{proof}

An additional knowledge of a degree sequence of $G$ can be exploited. 
\begin{lemma}
\label{lemma:degorder}
Let $(d_1,\dots,d_n)$ be a degree sequence and $t\leq \bar{t}-|S|$ a given delay.
Consider all possible instances of which graphs are described by this degree sequence.
The maximum number of nodes informed within $t$ time steps is achieved in an instance in which it is possible to inform  nodes in the order of their decreasing degree.

In other words, the instance with the shortest possible broadcast time is such that its nodes can be informed in the order of their decreasing degree.

\end{lemma}
\begin{proof}
A node with degree $d_i$ informed in $i$-th time step can inform at most $\min\{d_i-1,t-i\}$ nodes.
Consider $l,k$ such that $1\leq k < \ell\leq n$ and nodes $v_k$, $v_\ell$ with $\deg(v_k)=d_k$ and $\deg(v_\ell)=d_\ell$.
Then, $d_k-1>d_\ell-1$ and $t-k > t-\ell$.
If nodes are informed in the order of their decreasing degree, let $K_1$ and $L_1$ be the number of nodes that can be informed by $v_k$ and $v_l$, respectively:
$$
K_1=\min\{d_k-1,t-k\}, ~~~ L_1=\min\{d_\ell-1,t-\ell\}.
$$
If the order in which $v_k$ and $v_l$ are informed is switched, i.e. when $v_k$ and $v_\ell$ is informed in the $\ell$-th and $k$-th time step, respectively, 
we use $K_2$ and $L_2$ to denote the maximum number of nodes informed by $v_k$ and $v_\ell$:
$$
K_2=\min\{d_k-1,t-\ell\} ~~~ L_2=\min\{d_\ell-1,t-k\}.
$$
We now investigate what possible values can the expression 
\begin{equation}
\label{eq:degbounds}
(K_1+L_1)-(K_2+L_2)
\end{equation}
attain.
Observe that $K_1\geq L_2$ and $K_2\geq L_1$. 
If $K_2>L_1$, then $K_2=K_1=d_k-1$.
But in that case, $L_2=L_1=d_\ell-1$. 
We conclude that \eqref{eq:degbounds} always takes a non-negative value.
Therefore, if the nodes are informed in the order of their decreasing degrees,
the number of informed nodes within the give time step is no worse than when the nodes are informed in any other order.
\qed
\end{proof}

Alg.~\ref{alg:dreg} calculates a lower bound on the delay when given number of nodes, sources and a degree sequence in $G$ as an input.
The algorithm iteratively updates possible node degrees in $T^k$ in each time step $k$, and records the maximum possible number of nodes in $V_k$.
For the purpose of finding lower bounds, it is assumed that each node $v\in V_k$ with $\deg_{T^k}(v)<\deg_G(v)$ informs a new uninformed node.
The number of iterations is then the lowest possible delay for given input.
%The main idea is that in the most optimistic case, each node $v\in V_i$ with $\text{deg}_{F_i}(v)<\text{deg}_G(v)$ informs new not yet informed node.
%For the purpose of finding lower bounds, we can assume without loss of generality that $\text{deg}_{F_i}(u)\leq\text{deg}_{F_i}(v) \Leftrightarrow \text{deg}_G(u)\geq \text{deg}_G(v)$.
Once a node $v$ reaches its maximum degree, i.e., when $\text{deg}_{T^k}=\deg_G(v)$ for some $k$, $v$ does not inform any other node in the next time steps.

%For each $i=1,\dots d$, Alg. \ref{alg:dreg} keeps the  number of nodes with degree $i$.
%These values are updated iteratively using dynamic programming until the number of informed nodes reaches $n$. 

\begin{algorithm}
\KwData{$n,m,d_1,\dots,d_n\in \mathbb{N}, m\leq n,\newline 1\leq d_n\leq\dots\leq d_2\leq d_1$}
$a_1\leftarrow\dots\leftarrow a_{m}\leftarrow 1$\;%\tcp{this is a comment}
$a_{m+1}\leftarrow\dots\leftarrow a_{n}\leftarrow 0$\;%\tcp{this is a comment}
$c\leftarrow m$;%\tcp{every source informs a new node}
$~k\leftarrow 0$\;
\While {$c<n$} {
$k\leftarrow k+1$\;
$c'\leftarrow 0$\;
\For{$i=1,\dots,c$} {
	\If {$a_i<d_i$} {
		$a_i\leftarrow a_i + 1$\;
		$c'\leftarrow c' + 1$\;
		\If {$c+c'\leq n$} {
			$a_{c+c'}\leftarrow 1$;\tcp{Newly informed node} 
		}
	}
}
$c\leftarrow c + c'$\;
}
\Return $k$\;
%\Return $\lceil k/s \rceil$\;
 \caption{Lower bound exploiting distribution of degrees}
\label{alg:dreg}
\end{algorithm}

The input is assumed to be correct in the sense that a graph with the given degree sequence exists, and by definition, the degree sequence is ordered non-increasingly.
For each iteration $k$, Alg.~\ref{alg:dreg} stores degrees of nodes in $T^k$ in variables $a_1,\dots,a_n$.
Note that the forest $T^k$ is not actually constructed. 
The algorithm operates merely with potential degrees of nodes in $T^k$.
Next, variable $c$ keeps the value $|V_k|$, i.e., the number nodes informed within $k$ steps.
Finally, $c'$ stores $|V_k\setminus V_{k-1}|$, thus the number of nodes newly informed in time step $k$.

\begin{proposition}
If $G$ is an arbitrary graph with $n$ nodes, $m$ sources and node degrees $d_i$, $1\leq i\leq n$, Alg.~\ref{alg:dreg} calculates a lower bound on the delay in $G$.
\end{proposition}
\begin{proof}
The assumption that nodes with higher degree in $G$ are informed first in the optimal solution is justified by Lemma \ref{lemma:degorder}.
In each iteration $k$, degrees in $T^k$ of maximum possible number of nodes are increased in the order of their decreasing degree.
The current maximum possible number of informed nodes is stored in the variable $c$, and is updated at the end of the outre loop.
So, when the condition $c<n$ is tested on line 7, $c$ contains the correct value of number of nodes in $V_k$.
As soon as it is possible that at least $n$ nodes are informed, the algorithm returns the number of necessary iterations.
\qed
\end{proof}

\section{Upper bounds} \label{sec:ub}

A knowledge of an upper bound $\bar{t}$ affects the number of variables in all studied models. 
Particularly in the decision versions, the iterative approach can be terminated once the solution is found to be infeasible for broadcast time limit $\bar{t}-1$.
The algorithm presented in this section iteratively constructs broadcast forest $T=(V_T,A_T)$, where in the last iteration $V_T=V$.

\subsection{Restricted broadcast tree method}

The following idea is based on the observation that at every time step, the maximum number of nodes that can be informed equals the size of maximum matching between already informed nodes and the rest.
In the first iteration, the only informed nodes are the sources.
Once a maximum matching is found, the set of informed nodes is extended by the endpoints of the matching that were not yet informed.
This process is repeated until all nodes become informed.
The number of iteration necessary to inform all nodes is then the upper bound on the broadcast time.

A maximum matching can be found by an exact polynomial algorithm, or by solving an integer program.
Even though the second option is not a polynomial method, the solution time is negligible for the considered instance sizes.

Maximum matching can be found using integer programs presented earlier with maximum time step set to 1.
These models can be conveniently employed for an extension of this approach by increasing the maximum time step.
In our implementation, we used model \eqref{mod:basic:dec}.
A solution in each iteration gives a maximum number of newly informed nodes within the imposed time limit by finding a set of node disjoint broadcast trees rooted at nodes informed in previous iterations.
In this way we use the principle of rolling horizon method known from planning and scheduling.
For the next iteration, only some nodes are selected for extending the set of informed nodes, typically only the ones reachable in a single time step, thus a matching.
The steps are expressed by a pseudocode in Alg. \ref{alg:match}.
\begin{algorithm}[]
	\KwData{$G=(V,E), S\subseteq V, t\in \{1,\dots,n-\sigma\}$}
$V_T\leftarrow S, A_T \leftarrow \emptyset$\;
$\bar{t}\leftarrow 0$\;
\While{$V_T\neq V$} {
	$S\leftarrow V_T$\;
	$x\leftarrow$ optimal solution to model \eqref{mod:basic:dec}\;
%	Find a set of restricted broadcast trees $\{T_1,\dots,T_{|V_T|}\}$ with broadcast time at most $t_{\text{max}}$ rooted in nodes of $V_T$ by solving model \eqref{mod:genmatch}\;
%	$V_T\leftarrow V_T\cup \{v:v\in V_B:\beta(v)\leq p\}$\;
	$V_T\leftarrow V_T\cup \{v\in V\setminus V_T:x_{uv}^1=1, u\in V_T\}$\;
	$A_T\leftarrow A_T\cup \{(u,v)\in V_T\times V\setminus V_T: x_{uv}^1=1\}$\;
	$\bar{t}\leftarrow \bar{t}+1$\;
}
\Return $\bar{t}$\;
%\Return $\lceil k/s \rceil$\;
 \caption{A method for determining an upper bound based on iterative search for  trees}
\label{alg:match}
\end{algorithm}

For calculating an upper bound, it is not necessary to store node and arcs sets $V_T$ and $A_T$ in Alg \ref{alg:match},
but it is essential when the actual broadcast tree is desirable.
%It is also possible to use model \eqref{mod:genmatch} and obtain vector $y$.
%In that case lines 6 and 7 in Alg. \ref{alg:match} would be replaced by $V_T\leftarrow V_T\cup\{v\in V\setminus V_T:y_{2u}^v=1,u\in V_T\}$ and 
%$A_T\leftarrow A_T\cup \{(u,v)\in V_T\times V\setminus V_T:y_{2u}^v\}$, respectively.


%\begin{acknowledgements}
%If you'd like to thank anyone, place your comments here
%and remove the percent signs.
%\end{acknowledgements}

% BibTeX users please use one of
%\bibliographystyle{spbasic}      % basic style, author-year citations
%\bibliographystyle{spmpsci}      % mathematics and physical sciences
%\bibliographystyle{spphys}       % APS-like style for physics
%\bibliography{}   % name your BibTeX data base

% Non-BibTeX users please use
\begin{thebibliography}{}
%
% and use \bibitem to create references. Consult the Instructions
% for authors for reference list style.
%
\bibitem{chu17}
Chu, X., Chen, Y.,
Time division inter‐satellite link topology generation problem: Modeling and solution,
International Journal of Satellite Communications and Networking, 194 -- 206, 36 (2017)

\bibitem{cormen90}
Cormen, T. H., Leiserson, C. E., Rivest, R. L,
Introduction to Algorithms, 
MIT Press, 401 -- 402, 1990. 

\bibitem{elkin03}
Elkin, M., Kortsarz, G.,
Sublogarithmic approximation for telephone multicast: path out of jungle,
Symposium on Discrete Algorithms, 76 -- 85 (2003)

\bibitem{farley81}
Farley, A. M., Proskurowski, A.,
Broadcasting in Trees with Multiple Originators,
SIAM Journal on Algebraic Discrete Methods, 381 -- 386, 2, 4 (1981)

\bibitem{grigni91}
Grigni, M., Peleg, D.,
Tight bounds on minimum broadcast networks
Networks, 207-222, 4 (1991)

\bibitem{hasson04} 
Hasson, Y., Sipper, M.,
A Novel Ant Algorithm for Solving the Minimum Broadcast Time Problem,
International Conference on Parallel Problem Solving from Nature, 775 -- 780 (2004)

\bibitem{harutyunyan06}
Harutyunyan, H. A., Shao, B.,
An efficient heuristic for broadcasting in networks,
Journal of Parallel and Distributed Computing, 68 -- 76, 66, 1 (2006)

\bibitem{harutyunyan14}
Harutyunyan, H. A., Jimborean, C.,
New Heuristic for Message Broadcasting in Network,
IEEE 28th International Conference on Advanced Information Networking and Application, 517 -- 524, (2014)

\bibitem{jansen95}
Jansen, K., M\"uller, H.,
The minimum broadcast time problem for several processor networks, 
Theoretical Computer Science, 69 -- 85, 147 (1995)

\bibitem{kortsarz95}
Kortsarz, G., Peleg, D.,
Approximation algorithms for minimum-time broadcast
SIAM Journal on Discrete Mathematics, 401 -- 427, 8, 3 (1995)

\bibitem{mcgarvey16}
McGarvey, R. G., Rieksts, B. Q., Ventura, J. A., Ahn, N.,
Binary linear programming models for robust broadcasting in communication networks,
Discrete Applied Mathematics, 173 -- 84, 204, (2016)

\bibitem{middendorf93}
Middendorf, M.,
Minimum broadcast time is NP-complete for 3-regular planar graphs and deadline 2,
Information Processing Letters, 281 -- 287, 46 (1993)

\bibitem{noe05}
Noe, T. D., Post, J. V., 
Primes in Fibonacci n-step and Lucas n-Step Sequences,
J. Integer Seq. 8, Article 05.4.4, 2005.

\bibitem{scheuermann84}
Scheuermann, P., Wu, G.,
Heuristic Algorithms for Broadcasting in Point-to-Point Computer Networks,
IEEE Transactions on Computers, 804 -- 811, 33, 9 (1984)

\bibitem{slater81}
Slater, P. J., Cockayne, E. J., Hedetniemi, S.T.,
Information dissemination in Trees,
SIAM Journal on Computing, 692 -- 701, 10, 4 (1981)

\bibitem{wang10}
Wang, W.,
Heuristics for Message Broadcasting in Arbitrary Networks,
Master thesis, Concordia University, Montr\'eal, Qu\'ebec, 
Retrieved from http://citeseerx.ist.psu.edu/viewdoc/download?doi=10.1.1.633.5827\&rep=rep1\&type=pdf (2010)

\bibitem{jimborean13}
Jimborean, C.,
New Heuristics for Message Broadcasting in Arbitrary Networks,
Master thesis, Concordia University, Montr\'eal, Qu\'ebec, 
Retrieved from https://spectrum.library.concordia.ca/977717/1/Jimborean\_MCompSc\_F2013.pdf (2013)


% Format for Journal Reference
%Author, Article title, Journal, Volume, page numbers (year)
% Format for books
%\bibitem{RefB}
%Author, Book title, page numbers. Publisher, place (year)
% etc
\end{thebibliography}

\end{document}
% end of file template.tex

